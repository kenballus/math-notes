\documentclass[12pt]{article}
\usepackage[english]{babel}
\usepackage[utf8]{inputenc}
\usepackage{amsmath, amssymb,amsthm}
\usepackage{graphicx}
\usepackage{hyperref}
\usepackage{geometry}

\setlength{\topmargin}{0pt}
\setlength{\headsep}{0pt}
\textheight = 600pt

\title{Real Analysis \\ Homework 13}
\author{Ben Kallus, Noah Barton}
\date{Due Friday, October 16}

\begin{document}
\maketitle

\hrule
\bigskip

\noindent {\bf Acknowledgements:} You helped us in class with 22.

\bigskip
\hrule
\bigskip

\noindent{\bf 22. Claim:} If $(a_n)$ is a bounded sequence such that all of its convergent subsequences converge to $a$, then $(a_n)$ converges to $a$.
\begin{proof}
    Suppose $(a_n)$ does not converge to $a$. Then, there exists $\epsilon > 0$ such that for each $k \in \mathbb N$, there exists $n_k \geq k$ such that $|a_{n_k} - a| \geq \epsilon$. Consider $(b_k)$, the subsequence of $(a_n)$ defined by $$b_k = \begin{cases} a_{n_1} & k = 1, \\ a_{n_k+1} & k \geq 2. \end{cases}$$ Since $(a_n)$ is bounded, $(b_n)$ is also bounded. Thus, $(b_n)$ contains a convergent subsequence by the Bolzano-Weierstrass Theorem. Since all terms of $(b_n)$ are at least distance $\epsilon$ from $a$, its convergent subsequence cannot converge to $a$. Thus, we have arrived at a contradiction, so $(a_n)$ must converge to $a$.
\end{proof}

\newpage
\noindent{\bf 23.} Suppose $(a_n)$ and $(b_n)$ are Cauchy sequences.

\medskip
\noindent{\bf a. Claim:} $(a_n + b_n)$ is a Cauchy sequence.
\begin{proof}
    By the Cauchy Criterion, $(a_n)$ converges to $a$ and $(b_n)$ converges to $b$ for some $a,b \in \mathbb R$. Then, by the Algebraic Limit Theorem, $(a_n + b_n)$ converges to $a+b$. Thus, by the Cauchy Criterion, $(a_n + b_n)$ is a Cauchy sequence.
\end{proof}

\medskip
\noindent{\bf b. Claim:} $(a_n + b_n)$ is a Cauchy sequence.
\begin{proof}
    Let $\epsilon > 0$ be given. Since $(a_n)$ is a Cauchy sequence, there exists $N_1 \in \mathbb N$ such that for all $n_1,m_1 \geq N_1$, $|a_{n_1} - a_{m_1}| < \frac\epsilon2$. Similarly, there exists $N_2 \in \mathbb N$ such that for all $n_2,m_2 \geq N_2$, $|b_{n_2} - b_{m_2}| < \frac\epsilon2$. Let $N = \max\{N_1, N_2\}$. Then, for all $n,m \geq N$,
    \begin{align*}
        |(a_{n} + b_{n}) - (a_{m} + b_{m})| &= |a_{n} - a_{m} + b_{n} - b_{m}| \\
                                            &\leq |a_{n} - a_{m}| + |b_{n} - b_{m}| \\
                                            &< \frac\epsilon2 + \frac\epsilon2 \\
                                            &= \epsilon.
    \end{align*}
    Thus, $(a_n + b_n)$ is a Cauchy sequence.
\end{proof}

\end{document}