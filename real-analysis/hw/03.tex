\documentclass[12pt]{article}
\usepackage[english]{babel}
\usepackage[utf8]{inputenc}
\usepackage{amsmath, amssymb,amsthm}
\usepackage{graphicx}
\usepackage{hyperref}
\usepackage{geometry}

\setlength{\topmargin}{0pt}
\setlength{\headsep}{0pt}
\textheight = 600pt

\title{Real Analysis \\ Homework 3}
\author{Ben Kallus}
\date{Due Thursday, September 3}

\begin{document}
\maketitle

\hrule
\bigskip

\noindent {\bf Acknowledgements:}  I did not collaborate with anyone for this assignment.

\bigskip
\hrule

\bigskip
\noindent{\bf 7.} Suppose $A \subseteq \mathbb R$ and $s =$ sup $A$. Also, let $\epsilon > 0$ be a fixed positive real number.

\medskip
\noindent{\bf a. Claim:} $s - \epsilon$ cannot be an upper bound for $A$.
\begin{proof}
    Suppose (toward a contradiction) that $s- \epsilon$ is an upper bound for $A$. Then, since $s - \epsilon < s$, $s - \epsilon$ is a lesser upper bound that $s$. This is a contradiction, since $s$ is defined to be the supremum of $A$. Thus, $s - \epsilon$ cannot be an upper bound for $A$.
\end{proof}

\medskip
\noindent{\bf b. Claim:} There exists some $a \in A$ such that $a > s - \epsilon$.
\begin{proof}
    Since $s - \epsilon$ is not an upper bound for $A$, there exists some $a \in A$ such that $a > s - \epsilon$ by the definition of upper bound.
\end{proof}

\medskip
\noindent{\bf c. Claim:} $s$ is the supremum of $A$ if and only if, for every real number $\epsilon > 0$, $a > s - \epsilon$.
\begin{proof}
    ($\Rightarrow$) Suppose $s =$ sup $A$, and let $\epsilon > 0$ be given. Then by part (b), there must exist $a \in A$ such that $a > s - \epsilon$.
    
    ($\Leftarrow$) Now suppose $s$ is an upper bound for $A$ with the property that, for every $\epsilon > 0$, there exists $a \in A$ such that $a > s - \epsilon$. We want to prove that $s =$ sup $A$.
    
    Because we are assuming that $s$ is an upper bound for $A$, by Definition 1.3.2, we just need to prove that, for any $u \in \mathbb R$, if $u$ is an upper bound for $A$, then $s \leq u$. Suppose (toward a contradiction) that this is not true. In other words, suppose that $u < s$. Then, $u = s - \epsilon$ for some $\epsilon > 0$. Then, by the result of part (b), there exists $a \in A$ such that $a > u$. Therefore, $u$ is not an upper bound for $A$.
    
    Thus, $s$ is the supremum of $A$ if and only if, for every real number $\epsilon > 0$, $a > s - \epsilon$.
\end{proof}


\newpage
\noindent {\bf 8. Answer:}

Let $s \in \mathbb R$ be an upper bound for a set $A$. Then, $s =$ inf $A$ if and only if there exists $a \in A$ such that $s + \epsilon > a$ for all $\epsilon > 0$.

\end{document}
