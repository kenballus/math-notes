\documentclass[12pt]{article}
\usepackage[english]{babel}
\usepackage[utf8]{inputenc}
\usepackage{amsmath, amssymb,amsthm}
\usepackage{graphicx}
\usepackage{hyperref}
\usepackage{geometry}

\setlength{\topmargin}{0pt}
\setlength{\headsep}{0pt}
\textheight = 600pt

\title{Real Analysis \\ Homework 14}
\author{Ben Kallus, Noah Barton}
\date{Due Monday, October 19}

\begin{document}
\maketitle

\hrule
\bigskip

\noindent {\bf Acknowledgements:} None.

\bigskip
\hrule
\bigskip

\noindent{\bf 24. Claim:} If $\sum\limits_{k=1}^\infty b_k$ is a convergent series, and $0 \leq a_k \leq b_k$ for every $k \in \mathbb N$, then $\sum\limits_{k=1}^\infty a_k$ is a convergent series.
\begin{proof}
    Let $(s_n)$ be defined by $s_n = \sum\limits_{k=1}^n a_k$, and let $(t_n)$ be defined by $t_n = \sum\limits_{k=1}^n b_k$. Since $\sum\limits_{k=1}^\infty b_k$ is a convergent series, $(t_n)$ is a convergent sequence. Thus, it is bounded, so there exists $M > 0$ such that $M \geq t_k$ for all $k \in \mathbb N$. Note that $s_k \leq t_k$ for all $k \in \mathbb N$, because $0 \leq a_k \leq b_k$ for all $k \in \mathbb N$. Thus, $M \geq s_k$ for all $k \in \mathbb N$, so $(s_k)$ is also bounded. Observe that $(s_n)$ is increasing, because for all $n \in \mathbb N$, $s_n \leq s_n + a_{n+1} = s_{n+1}$. Thus, by the Monotone Convergence Theorem, $(s_n)$ converges. Thus, $\sum\limits_{k=1}^\infty a_k$ is a convergent series.
\end{proof}

\newpage
\noindent{\bf 25.} Consider the series $\sum\limits_{n=1}^\infty(-1)^{n+1}a_n$. Note that the $n^\text{th}$ partial sum is $$s_n=a_1 - a_2 + a_3 - a_4 + \hdots + (-1)^{n+1}a_n.$$

\noindent{\bf a. Claim:} For all $n \in \mathbb N$, $s_{2n} \leq a_1$.
\begin{proof}
    Since $(a_n)$ is decreasing, $a_i - a_{i+1} \geq 0$ for all $i \in \mathbb N$. Then, for all $n \in \mathbb N$,
    \begin{align*}
        s_{2n} &= a_1 - a_2 + a_3 - a_4 + a_5 + \hdots - a_{2n-2} + a_{2n-1} - a_{2n} \\
               &= a_1 - (a_2 - a_3) - (a_4 - a_5) + \hdots - (a_{2n-2} - a_{2n-1}) - a_{2n} \\
               &\leq a_1 - a_{2n} \\
               &\leq a_1.
    \end{align*}
    Thus, $s_{2n} \leq a_1$ for all $n \in \mathbb N$.
\end{proof}

\noindent{\bf b. Claim:} The sequence $(s_{2n})$ converges.
\begin{proof}
    Note that since $(a_n)$ is decreasing, $a_n - a_{n+1} \geq 0$ for all $n \in \mathbb N$. For the base case, observe that $s_2 = a_1 - a_2  \leq a_1 - a_2 + (a_3 - a_4) = s_4$. Begin the inductive step by assuming that $2_{2n} \leq 2_{2(n+1)}$ for some $n \in \mathbb N$. Then,
    \begin{align*}
        2_{2(n+1)} &= a_1 - a_2 + \hdots - a_{2n} + a_{2n+1} - a_{2n+2} \\
                   &\leq a_1 - a_2 + \hdots - a_{2n} + a_{2n+1} - a_{2n+2} + (a_{2n+3} - a_{2n+4}) \\
                   &= s_{2(n+1)}.
    \end{align*}
    Thus, by the principle of mathematical induction, $(s_{2n})$ is increasing. Note that $(s_{2n})$ is bounded since $a_1 \geq s_{2n}$ for all $n \in \mathbb N$. Therefore, by the Monotone Convergence Theorem, $(s_{2n})$ converges.
\end{proof}

\medskip
\noindent{\bf c. Claim:} The sequence $(s_{2n-1})$ and the sequence $(s_{2n})$ converge to the same value, $s$.
\begin{proof}
    Let $\epsilon > 0$ be given. Since $(s_{2n})$ converges to $s$, there exists $N_1 \in \mathbb N$ such that $|s_{2n} - s| < \frac\epsilon2$ for all $n \geq N_1$. Since $(a_n)$ converges to 0, there exists $N_2 \in \mathbb N$ such that $|a_n - 0| < \frac\epsilon2$ for all $n \geq N_2$. Define $N = \max\{N_1, N_2\}$. Note that $s_{2n-1} = s_{2n} + a_{2n}$ for all $n \in \mathbb N$. Then, for all $n \geq N$,
    \begin{align*}
        |s_{2n-1} - s| &= |s_{2n} + a_{2n} - s| \\
                       &\leq |s_{2n} - s| + |a_{2n}| \\
                       &< \frac\epsilon2 + \frac\epsilon2 \\
                       &= \epsilon.
    \end{align*} Thus, $(s_{2n-1})$ converges to $s$.
\end{proof}

\newpage
\noindent{\bf d. Claim:} The sequence $\sum\limits_{n=1}^\infty (-1)^{n+1} a_n$ converges.
\begin{proof}
    Let $\epsilon > 0$ be given, and let $s_n$ be the $n^\text{th}$ partial sum of $\sum\limits_{n=1}^\infty (-1)^{n+1} a_n$. Then, there exists $N_1$ such that $|s_{2n} - s| < \epsilon$ for all $n \geq N_1$. Similarly, there exists $N_2$ such that $|s_{2n-1} - s| < \epsilon$ for all $n \geq N_2$. Define $N = \max\{2N_1, 2N_2-1\}$. Then, for all even $n \geq N$, $s_n = s_{2n_1}$ for some $n_1 \geq N_1$, and for all odd $n \geq N$, $s_n = s_{2n_2-1}$ for some $n_2 \geq N_2$. Thus, for all $n \geq N$, $|s_n - s| < \epsilon$. Thus, $(s_n)$ converges to $s$, so $\sum\limits_{n=1}^\infty (-1)^{n+1} a_n$ converges.
\end{proof}
\end{document}