\documentclass[12pt]{article}		
\usepackage[english]{babel} 		
\usepackage[utf8]{inputenc}			
\usepackage{amsmath, amssymb,amsthm}
\usepackage{graphicx}				
\usepackage{hyperref}				

\setlength{\topmargin}{0pt}
\setlength{\headsep}{0pt}
\textheight = 600pt

\title{Real Analysis \\ Homework 1}
\author{Ben Kallus}
\date{Due Wednesday, August 26}

\begin{document}

\maketitle

\hrule
\bigskip

\noindent {\bf Acknowledgements:}  I did not collaborate with anyone for this assignment.

\bigskip
\hrule
\bigskip

\noindent 1.

\noindent a. {\bf Example:} $A_n = \{n\}$.

\bigskip
\noindent b. {\bf Example:}

$A_1 = \{3i - 2~|~i \in \mathbb N\}$,

$A_2 = \{3i - 1~|~i \in \mathbb N\}$,

$A_3 = \{3i~|~i \in \mathbb N\}$.

\bigskip
\noindent {\bf Extra Hard Example:}

Let $T$ be the set containing 1, as well as all natural numbers that are not powers of primes.

Let $p_n$ be the $n^\text{th}$ prime.

Then, $A_n = \{p_n, p_n^2, p_n^3, \hdots\}~\cup~\{$the $n^\text{th}$ smallest element of $T\}$.

\newpage
\noindent 2.

\noindent a. {\bf Claim:}  $\bigcap\limits_{n=1}^\infty A_n$ must be nonempty.

\begin{proof}
    Let $A_z$ be the final $A_i$ in the sequence. (If the sequence is infinite, then it must end in the repetition of a set. If this is the case, let $A_z$ be that set.)
    
    Let $a \in A_z$.
    
    If the sequence is infinite, then $a$ must be in all $A_i$ such that $i > z$, since those sets are all equal to $A_z$.
    
    If the sequence is finite, then there are not $A_i$ such that $i > z$, so we need not consider those sets.
    
    Since $A_z \subseteq A_{z-1} \subseteq \hdots \subseteq A_1$, $a$ must also be an element of all $A_i$ such that $i < z$.
    
    Thus, $a$ is in every set in the sequence, so $a \in \bigcap\limits_{n=1}^\infty A_n$.
\end{proof}

\medskip
\noindent b. {\bf Counterexample:} Suppose $A_n = \{n,n+1,n+2,\hdots\}$. Then  $\bigcap\limits_{n=1}^\infty A_n =\emptyset$, which is a finite set, since every integer $m$ is not an element of $A_{m+1}$.

\medskip
\noindent c. {\bf Counterexample:} Suppose $A_n = \{n,n+1,n+2,\hdots\}$. Then  $\bigcap\limits_{n=1}^\infty A_n =\emptyset$, since every integer $m$ is not an element of $A_{m+1}$.

\bigskip
\noindent 3. {\bf Claim:} Let $x \in \mathbb R$ and $x \geq 0$. Suppose that $x < \epsilon$ for every real number $\epsilon > 0$. Then, $x = 0$.

\begin{proof}
    Assume $x > 0$.
    
    Suppose that $x < \epsilon$ for every real number $\epsilon > 0$.
    
    Consider the case in which $\epsilon = x$. Then, $x \not < \epsilon$, which is a contradiction.
    
    Thus, it cannot be that $x > 0$.
    
    Thus, $x$ must equal 0.
\end{proof}

\end{document}