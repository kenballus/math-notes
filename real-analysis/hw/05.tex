\documentclass[12pt]{article}
\usepackage[english]{babel}
\usepackage[utf8]{inputenc}
\usepackage{amsmath, amssymb,amsthm}
\usepackage{graphicx}
\usepackage{hyperref}
\usepackage{geometry}

\setlength{\topmargin}{0pt}
\setlength{\headsep}{0pt}
\textheight = 600pt

\title{Real Analysis \\ Homework 5}
\author{Ben Kallus, Truong Pham}
\date{Due Thursday, September 10}

\begin{document}
\maketitle

\hrule
\bigskip

\noindent {\bf Acknowledgements:} I consulted with you about style for questions 11 and 12.

\bigskip
\hrule
\bigskip

\noindent{\bf 11. Claim:} The infimum of $\Big{\{}\frac{3n+2}{2n+1} : n \in \mathbb N\Big{\}}$ is $\frac 3 2$.

\begin{proof}
    Let $S = \Big{\{}\frac{3n+2}{2n+1} : n \in \mathbb N\Big{\}}$.
    
    Observe that \begin{align*}
        3(2n + 1) &= 6n + 3 \\
                  &< 6n + 4 \\
                  &= 2(3n + 2).
    \end{align*}
    Since $3(2n + 1) < 2(3n + 2)$, $$\frac32 < \frac{3n+2}{2n+1}.$$
    Thus, $\frac32$ is a lower bound for $S$.
    
    Observe that $\frac53 \in S$, so $S$ is nonempty. Suppose there exists $\epsilon > 0$ such that $\frac32 + \epsilon$ is a lower bound for $S$. Then, $$\frac32 + \epsilon \leq \frac{3n+2}{2n+1}.$$
    Thus,
    \begin{align*}
        \epsilon &\leq \frac{3n+2}{2n+1} - \frac32 \\
                 &\leq \frac{3n+2}{2n+1} - \frac{3(n + \frac12)}{2(n + \frac12)} \\
                 &\leq \frac{3n+2 - (3n + \frac32)}{2n+1} \\
                 &\leq \frac{\frac12}{2n+1} \\
                 &\leq \frac{1}{4n+2} \\
                 &\leq \frac{1}{n}.
    \end{align*}
    This inequality contradicts the Archimedian Principle.
    Thus, by the lower bound equivalent of Lemma 1.3.8, $\frac32$ is the infimum of $S$.
\end{proof}

\newpage
\noindent{\bf 12. Claim:} For all $n \in \mathbb N, n < 2^n$.

\begin{proof}
    For the base case, observe that $1 < 2^1$.
    
    Begin the inductive step by assuming $n < 2^n$ for some $n \in \mathbb N$. Observe that
    \begin{align*}
        n + 1 &\leq n + n \\
              &= 2n \\
              &< 2(2^n) \text{ by the inductive assumption,}\\
              &=2^{n+1}.
    \end{align*}
    
    Thus, by the principle of mathematical induction, $n < 2^n$ for all $n \in \mathbb N$.
\end{proof}

\newpage
\noindent{\bf 13. Claim:} The set $S = \{\frac p {2^q} : p \in \mathbb Z, q \in \mathbb N\}$ is dense in $\mathbb R$.

\begin{proof}
    Let $S = \{\frac p {2^q} : p \in \mathbb Z, q \in \mathbb N\}$. Let $(a,b)$ be an open interval in $\mathbb R$. We claim that there exists an element of $s$, $\frac{p}{2^q}$, such that
    \begin{align}
        a < \frac{p}{2^q} < b.
    \end{align}
    To see that such an element exists, first observe that (1) is equivalent to
    \begin{align}
        2^qa < p < 2^qb.
    \end{align}
    Next observe that, by the Archimedian Principle, there exists $q \in \mathbb N$ such that $\frac1q < b - a$. Then, by the result of the previous proof, $\frac1{2^q} < b - a$. Fix such a $q \in \mathbb N$, and let $p$ be the smallest integer that is strictly greater than $2^qa$. Then, $p = 2^qa + \epsilon$ for some $0 < \epsilon < 1$. Clearly $2^qa < p$, which is the first half of (2).
    Now, observe that $$\frac{1}{2^q} < b - a.$$
    Then, since $\epsilon < 1$, $$\epsilon\Big{(}\frac{1}{2^q}\Big{)} < b - a.$$
    Thus, $$a + \epsilon\Big{(}\frac{1}{2^q}\Big{)} < b.$$
    Then, multiplying both sides by $2^q$, $$2^qa + \epsilon < 2^qb.$$
    Therefore, $$p < 2^qb.$$
    Therefore, $S$ is dense in $\mathbb R$.
    
    
\end{proof}

\end{document}
