\documentclass[12pt]{article}
\usepackage[english]{babel}
\usepackage[utf8]{inputenc}
\usepackage{amsmath, amssymb,amsthm}
\usepackage{graphicx}
\usepackage{hyperref}
\usepackage{geometry}
\usepackage{xcolor}

\setlength{\topmargin}{0pt}
\setlength{\headsep}{0pt}
\textheight = 600pt

\title{Real Analysis \\ Homework 20}
\author{Ben Kallus and Jonathan Sills}
\date{Due Friday, November 13, 2020}

\begin{document}
\pagecolor{black}
\color{white}
\maketitle

\hrule
\bigskip

\noindent {\bf Acknowledgements:} 

\bigskip
\hrule
\bigskip

\noindent{\bf 5.} Claim: Let $f: \mathbb R \to \mathbb R$ defined by \[f(x) = \begin{cases} 1 & x \in \mathbb Q, \\ 0 & x \notin \mathbb Q. \end{cases}\] Then, $\lim\limits_{x \to 0}f(x)$ does not exist.
\begin{proof}
    Let the sequence $(a_n)$ be defined by $a_n = \frac1n$, and let the sequence $(b_n)$ be defined by $b_n = \frac1{n\sqrt2}$.
    We have previously shown that $\lim\limits_{n\to\infty} a_n = 0$.
    Thus, by the Algebraic Limit Theorem, $\lim\limits_{n\to\infty} b_n = 0$.
    Note that all terms of $(a_n)$ are rational, all terms of $(b_n)$ are irrational.
    Therefore, $f(a_n) = (1, 1, 1, \hdots)$, and $f(b_n) = (0, 0, 0, \hdots)$.
    Thus, $\lim\limits_{n\to\infty} a_n = 1$, and $\lim\limits_{n\to\infty} b_n = 0$.
    Note that 0 is a term of neither $(a_n)$ nor $(b_n)$.
    Thus, by Theorem 4.2.3, $\lim\limits_{x\to0}f(x)$ does not exist.
\end{proof}

\newpage
\noindent{\bf 6.} Claim: Let $f: \mathbb R \to \mathbb R$ defined by \[f(x) = \begin{cases} x & x \in \mathbb Q, \\ 0 & x \notin \mathbb Q. \end{cases}\] Then, $\lim\limits_{x \to 0}f(x) = 0$.
\begin{proof}
    Let $\epsilon > 0$ be given.
    Let $(x_n) \subseteq \mathbb R$ be a sequence such that $\lim\limits_{n \to \infty} x_n = 0$ and $x_n \neq 0$ for all $n \in \mathbb N$.
    Let $(y_n)$ be the subsequence of $(x_n)$ consisting of $(x_n)$'s rational terms.

    Suppose $(y_n)$ is an infinite sequence.
        Then, since $f(y_n) = (y_n)$, $\lim\limits_{n \to \infty} f(y_n) = 0$.
        Thus, there exists $N \in \mathbb N$ such that for all $n \geq N$, $|f(y_n) - 0| < \epsilon$.
        Let $N'$ be the index of $f(y_N)$ in $(x_n)$.
        Let $n \geq N'$.
        Suppose $x_n$ is rational.
            Then, $x_n = y_m$ for some $m \geq N$.
            Thus, $|f(x_n) - 0| < \epsilon$.
        Now, suppose $x_n$ is irrational.
            Then, $f(x_n) = 0$, so $|f(x_n) - 0| = 0 < \epsilon$.
        Thus, $|f(x_n) - 0| < \epsilon$ for all $n \in \mathbb N$.
        Thus, $\lim\limits_{x \to 0} f(x) = 0$.

    Now, suppose $(y_n)$ is a finite sequence.
        Then, there exists $N \in \mathbb N$ such that for all $n \geq N$, $x_n$ is irrational.
        Thus, $f(x_n) = 0$ for all $n \geq N$.
        Thus, $|f(x_n) - 0| < \epsilon$ for all $n \geq N$.
        Therefore, $\lim\limits_{x \to 0} f(x) = 0$.

    Thus, $\lim\limits_{x \to 0} f(x) = 0$.
\end{proof}

\newpage
\noindent{\bf 7.} Claim: If $f: \mathbb R \to \mathbb R$ is continuous on $\mathbb R$, then $K = \{x~|~f(x)=1\}$ is a closed set.
\begin{proof}
    Let $\epsilon > 0$ be given.
    Let $a$ be a limit point of $K$.
    Then, since $a \in \mathbb R$, $f$ is continuous at $a$.
    Thus, there exists $\delta \in \mathbb R$ such that for all $x \in \mathbb R$ satisfying $|x - a| < \delta$, $|f(x) - f(a)| < \epsilon$.
    Since $a$ is a limit point of $K$, there exists $k \in K$ satisfying $|k - a| < \delta$.
    Thus, $|1 - f(a)| < \epsilon$.
    Thus, $|1 - f(a)| < \frac1n$ for all $n \in \mathbb N$.
    Thus, by the Archimedean Property, $|1 - f(a)| \leq 0$.
    Therefore, $f(a) = 1$.
    Thus, $a \in K$.
\end{proof}

\newpage
\noindent{\bf 8.} Claim: If $f: \mathbb R \to \mathbb R$ is continuous on $\mathbb R$, then $S = \{x~|~1<f(x)<2\}$ is an open set.
\begin{proof}
    Let $s \in S$.
    Then, since $s \in \mathbb R$, $f$ is continuous at $s$.
    Thus, there exists $\delta \in \mathbb R$ such that for all $x \in \mathbb R$ satisfying $|x - s| < \delta$, $|f(x) - f(s)| < \min\{2 - f(s), f(s) - 1\}$.
    Let $y \in (s - \delta, s + \delta)$.
    Then, $|f(y) - f(s)|$ is less than both $2 - f(s)$ and $f(s) - 1$.
    Therefore, $-(2 - f(s)) < f(y) - f(s) < 2 - f(s)$ and $-(f(s) - 1) < f(y) - f(s) < f(s) - 1$.
    Then, $-(f(s) - 1) < f(y) - f(s) < 2 - f(s)$.
    Therefore, $1 < f(y) < 2$, so $y \in S$.
    Thus, $(s - \delta, s + \delta) \subseteq S$, so $S$ is an open set.
\end{proof}
\end{document}