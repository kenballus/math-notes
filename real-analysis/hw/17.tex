\documentclass[12pt]{article}
\usepackage[english]{babel}
\usepackage[utf8]{inputenc}
\usepackage{amsmath, amssymb,amsthm}
\usepackage{graphicx}
\usepackage{hyperref}
\usepackage{geometry}
\usepackage{xcolor}

\setlength{\topmargin}{0pt}
\setlength{\headsep}{0pt}
\textheight = 600pt

\title{Real Analysis \\ Homework 17}
\author{Ben Kallus, Noah Barton}
\date{Due Thursday, October 29}

\begin{document}
\pagecolor{black}
\color{white}
\maketitle

\hrule
\bigskip

\noindent {\bf Acknowledgements:} None.

\bigskip
\hrule
\bigskip

\noindent{\bf 5.}

\medskip
\noindent{\bf a. Claim:} $\mathbb Q$ is not open and not closed.
\begin{proof}
    Observe that $\sqrt2$ has an infinite decimal representation: $1.41421...$. Let $(d_i)$ be defined by $d_i =$ the $i^\text{th}$ digit after the decimal point in the decimal representation of $\sqrt2$. Let $(a_n)$ be defined by $$a_n = 1 + \sum_{k=1}^n \frac{d_k}{10^k}.$$
    Then, $(a_n)$ converges to $\sqrt2$. Therefore, for all $\epsilon > 0$, $|\sqrt2 - \epsilon| < a_n$ for some $n \in \mathbb N$. Thus, for all $\epsilon > 0$, there exists $n \in \mathbb N$ such that $a_n \in (\sqrt2 - \epsilon, \sqrt2 + \epsilon)$. Since $a_n \in \mathbb Q$ for all $n \in \mathbb N$, $\sqrt2$ is a limit point of $\mathbb Q$. Thus, $\mathbb Q$ is not closed.
    
    $\mathbb Q$ is not open, since the irrationals are dense in $\mathbb R$, so any $\epsilon$-neighborhood of a rational number must contain an irrational number.
\end{proof}

\medskip
\noindent{\bf b. Claim:} $\mathbb N$ is closed and not open.
\begin{proof}
    $\mathbb N$ is closed because it has no limit points.

    Let $\epsilon > 0$ be given. Then, the $\epsilon$-neighborhood of 1 contains a number less than $1$, which must not be a natural number. Thus, $\mathbb N$ is not open.
\end{proof}

\medskip
\noindent{\bf c. Claim:} $S = \{x \in \mathbb R~|~x \neq 0\}$ is open and not closed.
\begin{proof}
    Observe that each term of the sequence $(a_n)$ defined by $a_n=\frac1n$ is an element of $S$. Thus, since $(a_n)$ converges to 0, 0 is a limit point of $S$. This can be shown using an argument similar to the one made in part (a). Thus, $S$ is not closed.

    $S$ is open, since for all $x \in S$, $(x - \left|\frac x2 \right|, x + \left|\frac x2 \right|) \subseteq S$.
\end{proof}

\medskip
\noindent{\bf d. Claim:} $S = \{1 + \frac14 + \frac19 + \hdots + \frac1{n^2}~|~n\in \mathbb N\}$ is not open and not closed.
\begin{proof}
    Observe that the sequence $(a_n)$ defined by $a_n = \sum\limits_{i=0}^n\frac1{i^2}$ converges to $\frac{\pi^2}6$, and that $a_n \in S$ for all $n \in \mathbb N$. Thus, $S$ is not closed, since $\frac{\pi^2}6$ is a limit point of $S$. This can be shown using a method similar to the one used in part (a).

    Since $S$'s minimum is 1, and any $\epsilon$-neighborhood of 1 must contain a number less than 1, $S$ is not open.
\end{proof}

\medskip
\noindent{\bf e. Claim:} $S = \{1 + \frac12 + \frac13 + \hdots + \frac1{n}~|~n\in \mathbb N\}$ is closed and not open.
\begin{proof}
    $S$ is closed, since it has no limit points.

    $S$ is not open, since $S$'s minimum is 1, and any $\epsilon$-neighborhood of 1 must contain a number less than 1.
\end{proof}

\newpage
\noindent{\bf 6.}

\medskip
\noindent{\bf a.} Example: $A = \mathbb N$

    

\medskip
\noindent{\bf b.} Example: $A = \mathbb Q$

    

\medskip
\noindent{\bf c.} Example: $A = \{1\}$



\medskip
\noindent{\bf d.} Example:
\begin{align*}
    A_1 &= \bigcup_{i=0}^\infty \left(\frac1{2n+1}, \frac1{2n+2}\right) \\
    A_2 &= \bigcup_{i=0}^\infty \left(\frac1{2n+2}, \frac1{2n+3}\right)
\end{align*}


\medskip
\noindent{\bf e.} Impossible.

\end{document}