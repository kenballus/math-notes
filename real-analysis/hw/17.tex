\documentclass[12pt]{article}
\usepackage[english]{babel}
\usepackage[utf8]{inputenc}
\usepackage{amsmath, amssymb,amsthm}
\usepackage{graphicx}
\usepackage{hyperref}
\usepackage{geometry}
\usepackage{xcolor}

\setlength{\topmargin}{0pt}
\setlength{\headsep}{0pt}
\textheight = 600pt

\title{Real Analysis \\ Homework 17}
\author{Ben Kallus, Noah Barton}
\date{Due Thursday, October 29}

\begin{document}
\pagecolor{black}
\color{white}
\maketitle

\hrule
\bigskip

\noindent {\bf Acknowledgements:} None.

\bigskip
\hrule
\bigskip

\noindent{\bf 5.}

\medskip
\noindent{\bf a. Claim:} $\mathbb Q$ is not open and not closed.
\begin{proof}
    $\mathbb Q$ is not closed, since $\sqrt2$ is a limit point of $\mathbb Q$, and $\sqrt2 \notin \mathbb Q$. $\mathbb Q$ is not open, since the irrationals are dense in $\mathbb R$, so any $\epsilon$-neighborhood of a rational number must contain an irrational number.
\end{proof}

\medskip
\noindent{\bf b. Claim:} $\mathbb N$ is not open and closed.
\begin{proof}
    $\mathbb N$ is closed, since it has no limit points. Let $\epsilon > 0$ be given. Then, the $\epsilon$-neighborhood of 1 contains a number less than $1$, which must not be a natural number. Thus, $\mathbb N$ is not open.
\end{proof}

\medskip
\noindent{\bf c. Claim:} $S = \{x \in \mathbb R~|~x \neq 0\}$ is open and not closed.
\begin{proof}
    $S$ is not closed, since 0 is a limit point of $S$. $S$ is open, since for all $x \in S$, $(x - \left|\frac x2 \right|, x + \left|\frac x2 \right|) \subseteq S$.
\end{proof}

\medskip
\noindent{\bf d. Claim:} $S = \{1 + \frac14 + \frac19 + \hdots + \frac1{n^2}~|~n\in \mathbb N\}$ is not open and not closed.
\begin{proof}
    $S$ is not closed, since $\frac\pi2$ is a limit point of $S$. $S$ is not open, since $S$ is increasing, $S$'s minimum is 1, and any $\epsilon$-neighborhood of 1 must contain a number less than 1.
\end{proof}

\medskip
\noindent{\bf e. Claim:} $S = \{1 + \frac12 + \frac13 + \hdots + \frac1{n}~|~n\in \mathbb N\}$ is 
\begin{proof}
    $S$ is closed, since it has no limit points. $S$ is not open, since $S$ is increasing, $S$'s minimum is 1, and any $\epsilon$-neighborhood of 1 must contain a number less than 1.
\end{proof}

\newpage
\noindent{\bf 6.}

\medskip
\noindent{\bf a.} Example: $A = \mathbb N$

    

\medskip
\noindent{\bf b.} Example: $A = \mathbb Q$

    

\medskip
\noindent{\bf c.} Example: $A = \{1\}$



\medskip
\noindent{\bf d.} Example: 



\medskip
\noindent{\bf e.} Impossible.

\end{document}