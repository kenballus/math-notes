\documentclass[12pt]{article}
\usepackage[english]{babel}
\usepackage[utf8]{inputenc}
\usepackage{amsmath, amssymb, amsthm}
\usepackage{graphicx}
\usepackage{hyperref}
\usepackage{geometry}
\usepackage{xcolor}

\setlength{\topmargin}{0pt}
\setlength{\headsep}{0pt}
\textheight = 600pt

\title{Real Analysis \\ Homework 23}
\author{Ben Kallus and Jonathan Sills}
\date{Due Monday, November 23, 2020}

\begin{document}
\pagecolor{black}
\color{white}
\maketitle

\hrule
\bigskip

\noindent {\bf Acknowledgements:} None.

\bigskip
\hrule
\bigskip

\noindent{\bf 14.} Example: The function $f: A \to \mathbb Q$ defined by $$f(x) = \begin{cases} -1 & x < \frac\pi2, \\ 1 & \text{otherwise,} \end{cases}$$ has the desired properties. Observe that $f(0) = -2$, and $f(2) = 1$, but there is no $x \in A$ for which $f(x) = 0$.

\newpage
\noindent{\bf 2.}

\medskip
\noindent{\bf a.} $P = \{0,1,2\}$.
\begin{align*}
    L(f, P) &= 0 \\
    U(f, P) &= 2
\end{align*}

\medskip
\noindent{\bf b.} $P = \{0, 1 - \frac\epsilon3, 1 + \frac\epsilon3, 2 - \frac\epsilon3, 2\}$ with a fixed $\epsilon$ such that $0 < \epsilon < 1$.
\begin{align*}
    L(f, P) &= 2 - \epsilon \\
    U(f, P) &= 2
\end{align*}

\medskip
\noindent{\bf c.}
\begin{proof}
    Let $P_1$ be a partition of $[0,2]$.
    Since the irrational numbers are dense in $\mathbb R$, every interval of the real numbers contains some irrational number.
    Thus, every interval with endpoints in $P$ contains some real number $x$ such that $f(x) = 1$.
    Since 1 is the maximum possible value for $f(x)$, $U(f, P_1)$ is equal to the sum of the lengths of the intervals in $P_1$.
    Thus, $U(f,P_1) = 2$.
    Thus, $U(f) = 2$.

    Let $(\mu_n)$ be a sequence of real numbers such that $0 < \mu_i < 1$ for all $i \in \mathbb N$, and $\lim\limits_{n \to \infty} \mu_n = 0$.
    Then, for the partition $P_n = \left\{0, 1 - \frac{\mu_n}3, 1 + \frac{\mu_n}3, 2 - \frac{\mu_n}3, 2 \right\}$, $L(f, P_n) = 2 - \mu_n$.
    Thus, $\lim\limits_{n\to\infty}L(f, P_n) = 2$.
    Thus, $L(f) \geq 2$.
    By Theorem 7.2.6, $L(f) \leq U(f) = 2$.
    Thus, $L(f) = 2$.

    Thus, $f$ is Riemann integrable, and $$\int_0^2 f(x) \,dx = 2.$$
\end{proof}

\end{document}