\documentclass[12pt]{article}
\usepackage[english]{babel}
\usepackage[utf8]{inputenc}
\usepackage{amsmath, amssymb,amsthm}
\usepackage{graphicx}
\usepackage{hyperref}
\usepackage{geometry}
\usepackage{xcolor}

\setlength{\topmargin}{0pt}
\setlength{\headsep}{0pt}
\textheight = 600pt

\title{Real Analysis \\ Homework 21}
\author{Ben Kallus and Jonathan Sills}
\date{Due Monday, November 16, 2020}

\begin{document}
\pagecolor{black}
\color{white}
\maketitle

\hrule
\bigskip

\noindent {\bf Acknowledgements:} None.

\bigskip
\hrule
\bigskip

\noindent{\bf 9.}

\medskip
\noindent{\bf a.} Claim: If $f: [a,b] \to \mathbb R$ is a function such that, for every $x \in [a,b]$, there is a number $\delta_x>0$ for which $f$ is bounded on $V_{\delta_x}(x)$, then $f$ is bounded on $[a,b]$.
\begin{proof}
    Suppose $f$ is not bounded on $[a,b]$.
    Then, for each natural number $n$, there exists $x_n \in [a,b]$ such that $|f(x_n)| > n$.
    This defines a bounded sequence $(x_n)$, and an unbounded sequence $f(x_n)$.
    Note that every subsequence of $f(x_n)$ is unbounded.
    By the Bolzano-Weierstrass Theorem, there exists a convergent subsequence of $(x_n)$.
    Let $(y_n)$ be one such sequence.
    Let $y = \lim\limits_{n \to \infty}(y_n)$.
    Thus, $y$ is a limit point of $[a,b]$.
    Since $[a,b]$ is a closed set, $y \in [a,b]$.
    By the definition of $f$, there exists $\delta_y > 0$ such that f is bounded on $V_{\delta_y}(y)$.
    Since $\lim\limits_{n \to \infty} y_n = y$, there exists $N \in \mathbb N$ such that for all $n \geq N$, $y_n \in V_{\delta_y}(y)$.
    Let $(z_n)$ be the sequence obtained by removing the first $N-1$ terms of $(y_n)$.
    Then, $f(z_n)$ is unbounded, since every subsequence of $f(x_n)$ is unbounded.
    Let $M$ be the bound of $f$ on $V_{\delta_y}(y)$.
    Then, since $f(z_n)$ is unbounded, there exists $m \in \mathbb N$ such that $|f(z_m)| > M$.
    However, since $z_m \in V_{\delta_y}(y)$, $|f(z_m)| \leq M$.
    We have now shown a contradiction, so $f$ must be bounded on $[a,b]$.
\end{proof}

\medskip
\noindent{\bf b.} Claim: The result of part (a) does not hold if the interval is $(a,b)$.
\begin{proof}
    Let $(a, b) = (0, 1)$, and let $f: (0, 1) \to \mathbb R$ by \[f(x) = \frac1x.\]
    Observe that $f$ contradicts the claim from part (a), since $f$ is bounded on any interval with endpoints in $(0,1)$, but $f$ is not bounded on $(0,1)$.
\end{proof}

\newpage
\noindent{\bf 10.}

\medskip
\noindent{\bf a.} Impossible.

\medskip
\noindent{\bf b.} $f(x) = \sin\left(\pi x\right) + 1$.

\medskip
\noindent{\bf c.} $f(x) = \frac1{2^x}\sin(\frac1x)$.

\medskip
\noindent{\bf d.} $f(x) = \begin{cases} x & x < \pi, \\ 0 & \text{otherwise.} \end{cases}$
\end{document}