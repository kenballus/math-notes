\documentclass[12pt]{article}
\usepackage[english]{babel}
\usepackage[utf8]{inputenc}
\usepackage{amsmath, amssymb, amsthm}
\usepackage{graphicx}
\usepackage{hyperref}
\usepackage[margin=.75in]{geometry}
\usepackage{xcolor}
\usepackage{tikz}

\newcommand{\id}{\text{id}}
\newcommand{\od}{\text{od}}

\setlength{\topmargin}{0pt}
\setlength{\headsep}{0pt}
\textheight = 600pt

\title{Graph Theory \\ Homework 11}
\author{Ben Kallus and Ryan Friedman}
\date{Due Friday, April 2}

\begin{document}
\maketitle

\noindent\textbf{7.8} Proposition: If every vertex of some tournament of order $n$ has the same outdegree $x$, then $x = \frac{n-1}2$.
\begin{proof}
	Let $T$ be a tournament of order $n$ such that each vertex in $V(T)$ has outdegree $x$.
	Then, since a tournament has $\frac{n(n-1)}2$ arcs, and each arc contributes 1 to the total outdegree of the graph, $$nx = \frac{n(n-1)}2.$$
	Thus, $$x = \frac{n-1}2.$$
\end{proof}

\newpage\noindent\textbf{7.10} Proposition: If $u$ and $v$ are vertices of a tournament such that $\vec{d}(u,v) = k$, then $\id(u) \geq k - 1$.
\begin{proof}
	Let $T$ be a tournament, and let $u,v \in V(T)$ such that $\vec d(u,v) = k$.
	Then, the shortest directed $u-v$ path $P = (u = p_0, \hdots, p_k = v)$ has length $k$.
	Since $P$ is the shortest $u-v$ path, it must be that $p_1u, p_2u, \hdots, p_ku \in A(T)$.
	Thus, $\id(u) \geq k-1$.
\end{proof}
	

\newpage\noindent\textbf{7.12}

\textbf{(a)}

	

\textbf{(b)}



\textbf{(c)}
	

\newpage\noindent\textbf{7.14}

	

\newpage\noindent\textbf{8.1}

	

\newpage\noindent\textbf{8.1}

	

\newpage\noindent\textbf{8.2}

	

\newpage\noindent\textbf{8.4}

	

\end{document}
