\documentclass[12pt]{article}
\usepackage[english]{babel}
\usepackage[utf8]{inputenc}
\usepackage{amsmath, amssymb, amsthm}
\usepackage{graphicx}
\usepackage{hyperref}
\usepackage[margin=.75in]{geometry}
\usepackage{xcolor}
\usepackage{tikz}

\newcommand{\id}{\text{id}}
\newcommand{\od}{\text{od}}

\setlength{\topmargin}{0pt}
\setlength{\headsep}{0pt}
\textheight = 600pt

\title{Graph Theory \\ Homework 11}
\author{Ben Kallus and Ryan Friedman}
\date{Due Friday, April 2}

\begin{document}
\maketitle

\noindent\textbf{7.8} Proposition: If every vertex of some tournament of order $n$ has the same outdegree $x$, then what is $x$?
\begin{proof}
	Let $T$ be a tournament of order $n$.
\end{proof}

\newpage\noindent\textbf{7.10} Proposition: If $u$ and $v$ are vertices of a tournament such that $\vec{d}(u,v) = k$, then $\id(u) \geq k - 1$.

	

\newpage\noindent\textbf{7.12}

\textbf{(a)} Proposition: If every vertex in a tournament $T$ belongs to a cycle in $T$, then $T$ is not necessarily strong.
\begin{proof}
\end{proof}

\textbf{(b)} Proposition: For every pair $u,v$ of vertices in a strong tournament $T$, there exists either a Hamiltonian $u-v$ path or a Hamiltonian $v-u$ path.
\begin{proof}
\end{proof}

\textbf{(c)}
	

\newpage\noindent\textbf{7.14}

\textbf{(a)} Proposition: If an odd number of teams play in a round robin tournament, it is possible for all teams to tie for first place.
	
\textbf{(b)} Proposition: If an even number of teams play in a round robin tournament, then it is not possible for all teams to tie for first place.
	
\newpage\noindent\textbf{8.1}

	

\newpage\noindent\textbf{8.2}

	

\newpage\noindent\textbf{8.4}

	

\newpage\noindent\textbf{ALSO}

	

\end{document}
