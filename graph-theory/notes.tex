\documentclass{article}
\usepackage[utf8]{inputenc}
\usepackage{amsmath}
\usepackage{amsfonts}
\usepackage{amssymb}
\usepackage{graphicx}
\usepackage{geometry}
\usepackage{xcolor}

\newcommand{\inv}{^{-1}}   
\newcommand{\Z}{\mathbb Z}
\newcommand{\R}{\mathbb R}
\newcommand{\Q}{\mathbb Q}
\newcommand{\C}{\mathbb C}
\newcommand{\N}{\mathbb N}

\begin{document}
\pagecolor{black}
\color{white}

\noindent{\bf Simple Graph}

    A simple graph $G$ consists of a finite nonempty set $V(G)$ of vertices, and a set $E(E)$ of edges, 2-element subsets of $V$.

\medskip\noindent{\bf Trivial Graph}

    A graph with only one vertex.

\medskip\noindent{\bf Size of a Graph}

    The number of edges in a graph.

\medskip\noindent{\bf Order of a Graph}

    The number of a vertices in a graph.

\medskip\noindent{\bf Word Graph}

    The word graph for a set of words has one vertex for each word in the set, and an edge between two vertices $a$ and $b$ indicate that $a$'s word can be transformed into $b$'s word by either exchanging two letters, or replacing a letter.

    A graph is a word graph if it is the word graph for some set of words.

\medskip\noindent{\bf Incident Edges and Vertices}

    A vertex $v$ and an edge $e$ are said to be incident if $v \in e$.

\medskip\noindent{\bf Adjacent Edges}

    Two edges are adjacent if they are both incident to some vertex $v$.

\medskip\noindent{\bf Adjacent Vertices}

    Two vertices are adjacent if they are both incident to some edge $e$.

\medskip\noindent{\bf Subgraph}

    A graph $H$ is a subgraph of a graph $G$ ($H \subseteq G$) if $V(H) \subseteq V(G)$ and $E(H) \subseteq E(G)$.

    If $H \subseteq G$, $V(H) \subset V(G)$, and $E(H) \subset E(G)$, then $H$ is a proper subgraph of $G$ ($H \subset G$).

\medskip\noindent{\bf Spanning Subgraph}

    If $H \subseteq G$ and $V(H) = V(G)$, then $H$ is a spanning subgraph of $G$.

\medskip\noindent{\bf Induced Subgraph}

    If $H \subseteq G$, and $(u,v \in V(H) \wedge uv \in E(G)) \implies uv \in E(H)$, then $H$ is an induced subgraph of $G$.

    If $S$ is a nonempty set of vertices from $G$, then the subgraph of $G$ induced by $S$ ($G[S]$ or $\langle S \rangle G$) is the induced subgraph $J$ with $V(J) = S$.

    If $X$ is a nonempty set of vertices from $G$, then the subgraph of $G$ induced by $X$ ($G[X]$ or $\langle X \rangle G$) is the induced subgraph $K$ with $E(J) = X$ and $v \in V(J) \iff v \in e$ for some $e \in E(J)$. This subgraph is called an edge-induced subgraph.

\medskip\noindent{\bf Walk}

    A $u-v$ walk $W$ in $G$ is a sequence of vertices in $G$, beginning with $u$ and ending with $v$ such that consecutive vertices in the sequence are adjacent. A walk of length 0 is a trivial walk.

    The length of a walk $W$ is one less than its length as a sequence, since a walk's length is defined in terms of edges.

\medskip\noindent{\bf Closed Walk}

    A closed walk is nontrivial, and has equal first and last nodes.

\medskip\noindent{\bf Open Walk}

    An open walk is nontrivial, and is not closed.

\medskip\noindent{\bf Trail}

    A $u-v$ trail is a $u-v$ walk in which no edge is traversed more than once.

\medskip\noindent{\bf Path}

    A $u-v$ walk in which no vertices are repeated is a $u-v$ path.

\medskip\noindent{\bf Theorem 1.6}

    If a graph $G$ contains a $u-v$ walk of length $l$, then $G$ contains a $u-v$ path of length at most $l$.

\medskip\noindent{\bf Circuit}

    A circuit in a graph $G$ is a closed trail of length 3 or more.

\medskip\noindent{\bf Cycle}

    A circuit in which no vertices are repeated is a cycle. A cycle is referred to as an odd cycle or an even cycle depending on the parity of its length.

\medskip\noindent{\bf Connected Graph}

    A graph $G$ is connected if for all $u,v \in G$, there exists a $u-v$ walk in $G$.

\medskip\noindent{\bf Connected Component}

    A connected subgraph of a graph $G$ that is not a proper subgraph of any other connected subgraph of $G$ is a connected component (or just component) of $G$.

\medskip\noindent{\bf Equivalence Relation}

    A binary relation $R$ is an equivalence relation on a set $S$ if and only if for all $s,u,v \in S$,
    \begin{enumerate}
        \item $u R u$ ($R$ is reflexive)
        \item $u R v$ and $v R s$ $\implies$ $u R s$ ($R$ is transitive)
        \item $u R v \implies v R u$ ($R$ is symmetric)
    \end{enumerate}

\medskip\noindent{\bf Theorem 1.7}

    Let $R$ be the relation defined on the vertex set of a graph $G$ by $u R v$, where $u,v \in V(G)$, if $u$ is connected to $v$. $R$ is an equivalence relation.

\medskip\noindent{\bf Theorem 1.8}

    Note: Subsumed by 1.10.
    Let $G$ be a graph of order 3 or more. If $G$ contains two distinct vertices $u$ and $v$ such that $G - u$ and $G - v$ are connected, then $G$ itself is connected.

\medskip\noindent{\bf Distance}

    The distance between two vertices $u$ and $v$ in a graph $G$ is the length of the shortest path between $u$ and $v$.

\medskip\noindent{\bf Geodesic}

    A path between $u$ and $v$ of length $d(u,v)$ is a $u-v$ geodesic. Equivalently, a $u-v$ geodesic is a shortest path between $u$ and $v$.

\medskip\noindent{\bf Diameter}

    The diameter of a graph $G$ is the greatest distance between any two vertices in $G$.

\medskip\noindent{\bf Theorem 1.9}

    Note: Subsumed by 1.10.
    If $G$ is a connected graph of order 3 or more, then $G$ contains two distinct vertices $u$ and $v$ such that $G - u$ and $G - v$ are connected.

\medskip\noindent{\bf Theorem 1.10}

    Let $G$ be a graph of order 3 or more. Then $G$ is connected if and only if $G$ contains two distinct vertices $u$ and $v$ such that $G - u$ and $G - v$ are connected.

\medskip\noindent{\bf Complete Graph}

    A complete graph on $n$ vertices ($K_n$) has an edge between every pair of distinct vertices.
    Thus, it has size ${n \choose 2}$.

\medskip\noindent{\bf Path Graph}

    A path graph on $n$ vertices ($P_n$) is a connected graph with all vertices of degree less than or equal to 2, and at least one vertex of degree less than or equal to 1. Equivalently, it can be labeled $v_1, v_2, \hdots, v_n$ so that its edges are $v_1v_2, v_2v_3, \hdots, v_{n-1}v_n$.

\medskip\noindent{\bf Cycle Graph}

    A path graph on $n$ vertices ($P_n$) is a connected graph with only degree 2 vertices. Equivalently, it can be labeled $v_1, v_2, \hdots, v_n$ so that its edges are $v_1v_2, v_2v_3, \hdots, v_{n-1}v_n, v_nv_1$.

\medskip\noindent{\bf Complement}

    The complement $\overline G$ of a graph $G$ is a graph with the same vertex set as $G$, but it has edges only where $G$ has none.

\medskip\noindent{\bf Empty/Null Graph}

    The empty graph, or null graph, of order $n$ has $n$ vertices and 0 edges, and is denoted $N_n$.

\medskip\noindent{\bf Theorem 1.11}

    If $G$ is a disconnected graph, then $\overline G$ is connected.

\medskip\noindent{\bf Bipartite Graph}

    A bipartite graph can be partitioned into two sets called partite sets, such that no edge in the graph connects two edges in the same partite set.

\medskip\noindent{\bf Theorem 1.12}

    A nontrivial graph $G$ is a bipartite graph if and only if $G$ contains no odd cycles.

\medskip\noindent{\bf Complete Bipartite Graph}

    A complete bipartite graph is a bipartite graph with all possible edges present. This is denoted $K_{n,m}$ for a bipartite graph with $n$ vertices in one partite set and $m$ in the other.

\medskip\noindent{\bf $k$-partite Graph}

    A $k$-partite graph can be partitioned into $k$ partite sets.

    A complete $k$-partite graph is defined as you would expect.

\medskip\noindent{\bf Union of Graphs}

    For graphs $G, H$, $G \cup H$ is defined by $V(G \cup H) = V(G) \cup V(H)$, and $E(G \cup H) = E(G) \cup E(H)$.

\medskip\noindent{\bf Join}

    For graphs $G, H$, the join $G + H$ consists of $G \cup H$ with all possible edges added between vertices in $G$ and vertices in $H$.

\medskip\noindent{\bf Cartesian Product}

    For two graphs $G$ and $H$, their Cartesian product $G \times H$ (or $G \square H$) is defined by $V(G \times H) = V(G) \times V(H)$, and vertex $(u,v)$ is adjacent to $(x,y)$ if either $u=x$ and $vy \in E(H)$, or $v=y$ and $ux \in E(G)$.

\medskip\noindent{\bf $n$-cube (Hypercube)}

    Define $Q_1$ to be $K_2$, and define $Q_n$ for $n \geq 2$ to be $Q_{n-1} \times K_2$.
    These graphs are known as the $n$-cubes or hypercubes.
    The $n$ cube may also be constructed by adding a vertex for each $n$-bit string, then connecting each vertex $v$ to each other vertex with Hamming distance 1 from $v$.

\medskip\noindent{\bf Multiset}

    A multiset is a set in which multiple copies of the same element are permitted.

\medskip\noindent{\bf Multigraph}

    A multigraph is a graph with a multiset for its edge set. In other words, multiple copies of the same edge are permitted. Edges that connect the same two vertices are known as parallel edges. (In many definitions of a multigraph, loops are allowed. In this one, they are not.)

    In a multigraph, the degree of a vertex is the number of edges incident to that vertex, counted with multiplicity. Thus, the Handshaking Lemma holds, since each edge contributes one to the degree of each of its endvertices.

\medskip\noindent{\bf Pseudograph}

    A pseudograph is a multigraph in which edges are 2-element multisets of the vertex set. In other words, it's a multigraph in which loops are permitted.

    In a pseudograph, the degree of a vertex is the number of edges incident to that vertex, plus double the number of loops at that vertex. Thus, the Handshaking Lemma still applies.

\medskip\noindent{\bf Digraph}

    A graph in which edges are not 2-element subsets of the vertex set, but ordered pairs with entries from the vertex set, are known as directed graphs, or digraphs. Note that an edge $(a,b)$ and an edge $(b,a)$ are permitted in the same directed graph, without requiring that that graph be a multigraph, too. Directed edges are called arcs, so the edge set of a digraph is usually labeled $A$.

    Note that loops are fine here, since we're dealing with ordered pairs, not sets.
    Also note that (unintuitively) $(a,b)$ is not parallel to $(b,a)$.

    In a digraph, we split the notion of degree into indegree and outdegree. The indegree of a veretx is the number of arcs entering that vertex, and the outdegree is the number of arcs leaving that vertex. Thus, the Handshaking Lemma applies to the sum of either degree, and the sum is $m$, instead of $2m$.

\medskip\noindent{\bf Oriented Graph}

    An oriented graph is a digraph with no directed loops and if $uv \in A$, then $vu \notin A$.

\medskip\noindent{\bf Underlying Undirected Graph of a Directed Graph}

    The underlying undirected graph $G$ of a directed graph $D$ is constructed by making $D$'s edges undirected.

\medskip\noindent{\bf Degree of a Vertex}

    The degree of a vertex in a graph $G$ is the number of vertices adjacent to $v$ in $G$.

\medskip\noindent{\bf Isolated Vertex}

    A vertex of degree 0.

\medskip\noindent{\bf Endvertex/Leaf/Pendant Vertex}

    A vertex of degree 1.

\medskip\noindent{\bf Minimum and Maximum Degree Vertices}

    The minimum degree of a vertex in a graph $G$ is denoted $\delta(G)$.

    The maximum degree of a vertex in a graph $G$ is denoted $\Delta(G)$.

\medskip\noindent{\bf Theorem 2.1 (The Handshaking Lemma)}

    If $G$ is a graph of size $m$, then $$\sum_{v \in V(G)} \deg(v) = 2m.$$

\medskip\noindent{\bf Corollary 2.3}

    Every graph has an even number of vertices of odd degree.

\medskip\noindent{\bf Theorem 2.4}

    Let $G$ be a graph of order $n$. If $\deg(u) + \deg(v) \geq n-1$ for all nonadjacent $u,v \in V(G)$, then $G$ is connected, and the diameter of $G$ is not more than 2.

\medskip\noindent{\bf }

\medskip\noindent{\bf Regular Graph}

    If all vertices in a graph are degree $r$, then it is a $r$-regular graph

\medskip\noindent{\bf Cubic Graph}

    A 3-regular graph is called a cubic graph.

\medskip\noindent{\bf Theorem 2.6}

    Let $r, n \in \mathbb Z$, with $0 \leq r \leq n-1$. Then, there exists an $r$-regular graph of order $n$ if and only if either $r$ or $n$ is even.

\medskip\noindent{\bf Theorem 2.7}

    Let $G$ be a graph and let $r \in \mathbb Z$ such that $r \geq \Delta(G)$. Then, $G$ is an induced subgraph of an $r$-regular graph $H$.

\medskip\noindent{\bf Degree Sequence}

    The degree sequence of a graph $G$ with vertices $\{v_1, v_2, \hdots, v_n\}$ is the non-increasing rearrangement of $(\deg(v_1), \deg(v_2), \hdots, \deg(v_n))$.

    Note: Degree sequence does not uniquely identify a graph.

\medskip\noindent{\bf Theorem 2.10}

    A non-increasing sequence of $n \geq 2$ non-negative integers $(d_1, d_2, \hdots, d_n)$, is graphical (a degree sequence for some graph) if and only if $(d_2-1, d_3-1, \hdots, d_{d_1+1}-1, d_{d_1+2}, \hdots, d_n)$ graphical.

\medskip\noindent{\bf Adjacency Matrix}

    An adjacency matrix for a graph $G$ is an $n \times n$ square matrix $A = [a_{ij}]$ with entries from $\{0, 1\}$.
    $a_{ij} = 1$ if and only if $v_i$ is adjacent to $v_j$ in $G$.

\medskip\noindent{\bf Incidence Matrix}

    The incidence matrix of $G$ is an $n \times m$ matrix $B = [b_{ij}]$ with entries from $\{0,1\}$.
    $b_{ij} = 1$ if and only if $v_i$ is incident to $e_j$.

\medskip\noindent{\bf Theorem 2.13}

    Let $A - [a_{ij}]$ be the adjacency matrix of a graph $G$. Then for all $k \in \N$, the $ij^{\text{th}}$ entry of $A^k$ (denoted $a_{ij}^{(k)}$, to be distinct from $a_{ij}$ raised to the power $k$) is equal to the number of $v_1 - v_j$ walks in $G$ of length $k$.

\medskip\noindent{\bf Graph Isomorphism}

    Graphs $G$ and $H$ are isomorphic if there exists a bijection $\phi: V(G) \to V(H)$ such that $$\forall u,v \in V(G), uv \in E(G) \iff phi(u)\phi(v) \in E(H).$$
    $phi$ is called a graph isomorphism.

\medskip\noindent{\bf Theorem 3.1}

    For all graphs $G$, $H$, $\phi: G \to H$ is an isomorphism if and only if $\phi: \overline G \to \overline H$ is an isomorphism.

\medskip\noindent{\bf Self-Complementary Graph}

    A graph is self-complementary if and only if $G \cong \overline G$.

    Note that this means $G$ has either $0~(\text{mod } 4)$ or $1~(\text{mod } 4)$ vertices.

    Examples : $P_4, C_5$.

\medskip\noindent{\bf Theorem 3.2}

    If $G \cong H$, then $G$ and $H$ have the same degree sequence.

\medskip\noindent{\bf Theorem 3.5}

    If $G \cong H$, then
    \begin{enumerate}
        \item $G$ is bipartite if and only if $H$ is bipartite.
        \item $G$ is connected if and only if $H$ is connected.
    \end{enumerate}

\medskip\noindent{\bf Theorem 3.6}

    Graph isomorphism is an equivalence relation.

\medskip\noindent{\bf Bridge}

    Let $G$ be a graph. An edge $e \in E(G)$ is a bridge if and only if the number of components in $G - e$ is greater than the number of components in $G$.

\end{document}
