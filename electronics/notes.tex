\documentclass{article}
\usepackage[utf8]{inputenc}
\usepackage{amsmath}
\usepackage{amsfonts}
\usepackage{amssymb}
\usepackage{graphicx}
\usepackage{geometry}
\usepackage{xcolor}

\newcommand{\inv}{^{-1}}   
\newcommand{\Z}{\mathbb Z}
\newcommand{\R}{\mathbb R}
\newcommand{\Q}{\mathbb Q}
\newcommand{\C}{\mathbb C}
\newcommand{\N}{\mathbb N}

\begin{document}
\pagecolor{black}
\color{white}

\noindent{\bf Charge}

    Charge is measured in Coulombs. You can think of charge as an invisible fluid with the following properties:
    \begin{enumerate}
        \item You cannot make or destroy charge in any electronic apparatus; all you can do is move it around.
        \item Charge can move freely through conductors, but it has difficulty moving through insulators.
        \item When charge moves it does work.
        \item When charge moves, it creates a magnetic field. When a magnetic field moves near a wire, it causes charge to move in the wire.
        \item Charge comes in two kinds, positive and negative. If you mix the two kinds then they cancel each other out.
    \end{enumerate}

\medskip
\noindent{\bf Current}

    Moving charges form an electric current. We measure current by the amount of charge that passes a pint in 1 second. The symbol for current is $I$ and the unit of current is the Ampere. When a current of 1 Ampere is flowing, a charge of 1 Coulomb passed each point in 1 second.

    Thus, $I=\frac Qt$ and $Q = It$, where $Q$ is the charge passing a point in time $t$.

\medskip
\noindent{\bf Circuits}

    Because of Property 1 of charge, electrical current can flow only in complete circles made of conducting materials, called circuits.

\medskip
\noindent{\bf Voltage}

    In order to make current flow in a circuit, we need to exert an electrical pressure on it. We call that electrical pressure electric potential, and we measure it in Volts. Electrical potential is a relative quantity; we measure the potential difference between two points in a circuit. Still, we often speak of voltage at a particular point in a circuit. When we do this, we are really measuring the potential relative to ground.

    We talk about voltage {\bf across} a component, and current {\bf through} a component.

\medskip
\noindent{\bf Power}

    When current flows it does work, either usefully or by producing waste heat. Power is the rate at which work is done. The power dissipated in a device depends on both the amount of current that flows through it and on the voltage driving it according to the following formula: $$P = IV,$$ where $I$ is the current through the component and $V$ the voltage drop across it.

    We measure energy in Joules and power in Watts. $$1J = 1W \times 1s,$$ $$1W=1\frac Js$$

    Thus, a current of $1A$ flowing through a potential difference of $1V$ generates a power of $1W$ and does work at the rate of $1\frac Js$.

\medskip
\noindent{\bf Resistance}

    A potential difference causes a current flow. For most materials under normal conditions, the amount of current that flows is proportional to the voltage difference that makes it flow. If we double the voltage, then the current will double. Materials that behave this way are called linear components. For linear components, $$V=kI,$$ where $k$ is a constant associated with the component. This constant is known as the component's resistance, and is denoted $R$. Thus, $$V=IR.$$ This relationship is known as Ohm's Law.

\medskip
\noindent{\bf Capacitance}

    A capacitor consists of two conductors separated by an insulator. If one of the conductors is charged, then an equal and opposite charge is induced in the other conductor. The magnitude of the charges, $Q$, is proportional to the voltage, $V$, across the capacitor. $$Q = C \times V$$, where $C$ is a quantity called capacitance. The bigger the capacitance, the more charge it takes to get a certain voltage drop across the component. The unit of $C$ is the Farad. If you have 1 Coulomb of charge on each plate, and 1 Volt across the capacitor, then it's a 1 Farad capacitor.

    Capacitors in parallel add like resistors in series.

    Capacitors in series add like resistors in parallel.

\medskip
\noindent{\bf Angular Frequency}

    The angular frequency of an AC voltage is $2\pi \cdot f$, and is denoted $\omega$.

\medskip
\noindent{\bf Phase Difference Formula}

    The phase difference $\phi$ of two AC voltages with time lag $t_p$ and period $\tau$ is $$\phi = \frac{t_p}{\tau} \cdot 2\pi.$$

\medskip
\noindent{\bf RMS Measurements}

    AC voltages are commonly measured in terms of RMS amplitude, because RMS makes it easy to calculate the power delivered by the source: $$P_{\text{average}} = \frac{V^2_{\text{RMS}}}{R} = \frac{\text{Amplitude}^2}{R}$$

    For a sine wave, RMS voltage is given by $$V_{\text{RMS}} = \frac{\text{Amplitude}}{\sqrt2}.$$

\medskip
\noindent{\bf Impedance of a Capacitor}

    In a circuit driven by a sine wave of frequency $f$, the capacitor acts like a frequency-dependent resistor of value $$Z = \frac{1}{\omega C},$$
    for $\omega = 2\pi f$. This value is known as the capacitor's imedance.

\medskip
\noindent{\bf Time Constant of a Capacitor}

    When a steady voltage $V_f$ is applied to a cpacitor through a resistors $R$, the voltage across the capacitor changes from its initial value $V_0$ to its final value $V_jF$ according to the following equation: $$V(t) = V_0 + (V_f-V_0) \cdot (1 - e^{-t / \tau}),$$ where $\tau$ is the time constant given by $$\tau = RC.$$

\end{document}
