\documentclass{article}
\usepackage[utf8]{inputenc}
\usepackage{amsmath}
\usepackage{amsfonts}
\usepackage{amssymb}
\usepackage{graphicx}
\usepackage{geometry}
\usepackage{xcolor}
\usepackage{gensymb}
\usepackage{hyperref}
\usepackage{gensymb}

\newcommand{\inv}{^{-1}}   
\newcommand{\Z}{\mathbb Z}
\newcommand{\R}{\mathbb R}
\newcommand{\Q}{\mathbb Q}
\newcommand{\C}{\mathbb C}
\newcommand{\N}{\mathbb N}

\begin{document}
\pagecolor{black}
\color{white}

\noindent{\bf 1.}
Let current flow to the right through the horizontal resistors, and down through the vertical resistors.
Then, $$I_1 = I_2 + I_3, ~~~ I_3 = I_4 + I_5,$$
and $$I_{1} = \frac{12-V_{a}}{1000}, ~~~ I_{2} = \frac{V_{a}-4}{2000}, ~~~ I_{3} = \frac{V_{a}-V_{b}}{10000}, ~~~ I_{4} = \frac{V_{b}}{1000}, ~~~ I_{5} = \frac{V_{b}}{3000}.$$
Then,
\begin{align*}
    I_{1}&=I_{2}+I_{3} \\
    \implies \frac{12-V_{a}}{1000}&=\frac{V_{a}-4}{2000}+\frac{V_{a}-V_{b}}{10000}, \\
    \implies V_a &= \frac{35}4 + \frac{1}{16}V_b.
\end{align*}
Thus,
\begin{align*}
    I_3 &= I_4 + I_5, \\
    \implies \frac{V_{a}-V_{b}}{10000}&=\frac{V_{b}}{1000}+\frac{V_{b}}{3000}, \\
    \implies V_{a}&=\frac{43}{3}V_{b}.
\end{align*}
Therefore,
\begin{align*}
    \frac{35}4 + \frac{1}{16}V_b &= \frac{43}{3}V_{b} \\
    \implies V_b &= \frac{29455}{144} \\
    &\approx 613.1\text{mV}.
\end{align*}

Now, we can solve for $V_a$:
\begin{align*}
    V_a &= \frac{43}3V_b \\
        &= \frac{1266565}{432} \\
        &\approx 8.788V
\end{align*}

Thus, $$I_{1} \approx 3.21\text{mA}, ~~~ I_{2} \approx 2.39\text{mA}, ~~~ I_{3} \approx 0.818\text{mA}, ~~~ I_{4} \approx 0.613\text{mA}, ~~~ I_{5} \approx .204\text{mA}.$$

\newpage\noindent{\bf 2.}

\begin{align*}
    \frac{V_{out}}{V_{in}} &= \frac{1}{\sqrt{1+(\omega RC)^2}}, \\
    \implies \frac{V_{out}}{\sin\left(1000t\right)} &= \frac{1}{\sqrt{1+\left(2\pi\cdot1000\cdot10000\cdot.2\cdot10^{-6}\right)^{2}}}, \\
    \implies V_{out} &= \frac{\sin\left(1000t\right)}{\sqrt{1+\left(2\pi\cdot1000\cdot10000\cdot.2\cdot10^{-6}\right)^{2}}} \\
            &= \frac{\sin\left(1000t\right)}{\sqrt{1+16\pi^{2}}}.
\end{align*}
Thus, the amplitude of the output wave is $\frac{1}{\sqrt{1+16\pi^{2}}} = 79.3$mV.

\begin{align*}
    \tan\phi &= \omega RC \\
             &= 2\pi\cdot1000\cdot10000\cdot.2\cdot10^{-6}, \\
    \implies \phi &= \arctan(4\pi) \\
                  &\approx 1.49.
\end{align*}
Thus, the phase shift of the output wave is 85.37\degree.

\newpage\noindent{\bf 3.}

The cut-off frequency is given by $$\omega_0 = \frac RL = \frac{50}{100\cdot10^{-6}} = \frac{50}{100\cdot10^{-6}} = 500000 \frac{\text{rad}}{\text{s}}$$
Thus, $$f = \frac{500000}{2\pi} \approx 79577\text{Hz}.$$

\newpage\noindent{\bf 4.}

% At .4V, we'll just see Vin, since the voltmeter has a high resistance and the diodes are not activated.
% At .7V, it's the same as before, but flat when it hits .6 and -.6
% At 1V it's the same logic as .7

\newpage\noindent{\bf 5.}

% V=IR -> I = .06A. That intersects the graph at ~.57V, so that's the expected voltage at point a.

\newpage\noindent{\bf 6.}

    $$V_{\text{RMS}} = \sqrt2 \cdot 15 \approx 21.2V.$$

\newpage\noindent{\bf 7.}

    {\bf (a)} You need smaller filter capacitors for a bridge rectifier than you do for a half-wave rectifier because a full bridge rectifier's output frequency is twice its input frequency, so the capacitor needs to store less charge to provide current between peaks.

    {\bf (b)} They are the same thing, so they need similarly-sized capacitors.

\newpage\noindent{\bf 8.}

\newpage\noindent{\bf 9.}

See \texttt{04.py}.

\newpage\noindent{\bf 10.}

{\bf (a)}
Since no current will flow through the 2k resistor, the open circuit is equivalent to the following circuit:
\begin{center}
\includegraphics[scale=.5]{schematic04.png}
\end{center}
Thus, the open circuit voltage is 7.5V.

\medskip{\bf (b)}
\begin{center}
\includegraphics[scale=0.25]{schematic042.png}
\end{center}
Thus, the short circuit current is $$\frac{2.73 \cdot 10^{-6}}{.001} = .00273\text{A}.$$

\medskip{\bf (c)}
Thus, the Thévenin equivalent of the circuit consists of a 7.5V voltage source connected to a $\frac{7.5}{.00273} \approx 2747\Omega$ resistor.

\end{document}
