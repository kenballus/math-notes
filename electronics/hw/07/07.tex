\documentclass{article}
\usepackage[utf8]{inputenc}
\usepackage{amsmath}
\usepackage{amsfonts}
\usepackage{amssymb}
\usepackage{graphicx}
\usepackage{geometry}
\usepackage{xcolor}
\usepackage{gensymb}
\usepackage{hyperref}
\usepackage{gensymb}
\usepackage{listings}

\newcommand{\inv}{^{-1}}   
\newcommand{\Z}{\mathbb Z}
\newcommand{\R}{\mathbb R}
\newcommand{\Q}{\mathbb Q}
\newcommand{\C}{\mathbb C}
\newcommand{\N}{\mathbb N}

\begin{document}
\pagecolor{black}
\color{white}

\noindent\textbf{1.}

	\textbf{(a)} The voltage drop across the 5M$\Omega$ resistor is 5V, because $9 - \frac{4}{4+5} \cdot 9 = 5$.

	\textbf{(b)} The DMM and 5M$\Omega$ resistor in parallel have resistance equivalent to a single $3.33$M$\Omega$ resistor. Thus, the voltage drop across the 4M$\Omega$ resistor is $\frac4{4+3.33}\cdot9 = 4.909091$, so the voltage drop across the 5M$\Omega$ resistor is $9 - 4.909091 = 4.090909$V. Thus, the meter will display 4.09V.

\noindent\textbf{2.}

	(PICTURES ON PHONE)

\noindent\textbf{3.}

	(Smush the two on the right together because they're in series, and then that's in parallel with the one next to it. repeat until 1 resistor. Not sure what the = sign is.)

\noindent\textbf{4.}

	When both switches are open, Vout is 5V. When one or both switches are closed, Vout is 4.4V, because there needs to be a voltage drop of .6V across any diode participating in the circuit.

\noindent\textbf{5.}

	Based on the circuit diagram of NOR given on page 12-16, and the circuit diagram of NOT given on page 12-1, this circuit computes
	\begin{align*}
		!A \text{NOR} !B &= !(!A \text{OR} !B) \\
				 &= !(!(A \text{AND} B)) \\
				 &= A \text{AND} B.
	\end{align*}

\noindent\textbf{6.}

	(PICTURE ON PHONE)

\noindent\textbf{7.}

	Figure 16.1 cannot be represented with logical equations because the output of the 1-1 state is impossible to determine without knowing the prior state of the circuit.

\noindent\textbf{8.}

	(DO ME LATER)

\noindent\textbf{9.}

	(DO ME LATER)

\noindent\textbf{10.}

	\begin{verbatim}def pulse(pin, period, on_time):
	    while True:
	        pin.value = 1
	        time.sleep(on_time / 1000)
	        pin.value = 0
	        time.sleep((period - on_time) / 1000)
	\end{verbatim}

\end{document}
