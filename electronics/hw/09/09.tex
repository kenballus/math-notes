\documentclass{article}
\usepackage[utf8]{inputenc}
\usepackage{amsmath}
\usepackage{amsfonts}
\usepackage{amssymb}
\usepackage{graphicx}
\usepackage{geometry}
\usepackage{xcolor}
\usepackage{gensymb}
\usepackage{hyperref}
\usepackage{gensymb}
\usepackage{listings}

\newcommand{\inv}{^{-1}}   
\newcommand{\Z}{\mathbb Z}
\newcommand{\R}{\mathbb R}
\newcommand{\Q}{\mathbb Q}
\newcommand{\C}{\mathbb C}
\newcommand{\N}{\mathbb N}

\begin{document}
\pagecolor{black}
\color{white}

\medskip\noindent\textbf{1.} 

    Bridge rectifier

\medskip\noindent\textbf{2.} 

    Output wil be sine wave. Figure out gain using formula from book.

\medskip\noindent\textbf{3.}

    Obvious answer is -10Vin.

    Real answer comes from thinking about currents.

\newpage\noindent\textbf{4.}

\medskip\noindent\textbf{5.} Because it doesn't pull any current from the input 

\medskip\noindent\textbf{6.}

\newpage\noindent\textbf{7.} 
    
\medskip\noindent\textbf{8.}

\newpage\noindent\textbf{9.}

\medskip\noindent\textbf{10.}

\end{document}
