\documentclass{article}
\usepackage[utf8]{inputenc}
\usepackage{amsmath}
\usepackage{amsfonts}
\usepackage{amssymb}
\usepackage{graphicx}
\usepackage{geometry}
\usepackage{xcolor}

\newcommand{\inv}{^{-1}}   
\newcommand{\Z}{\mathbb Z}
\newcommand{\R}{\mathbb R}
\newcommand{\Q}{\mathbb Q}
\newcommand{\C}{\mathbb C}
\newcommand{\N}{\mathbb N}

\begin{document}
\pagecolor{black}
\color{white}

\noindent{\bf Continuity}

    A function $f: \R \to \R$ is continuous if for all $c \in \R, \epsilon > 0$, there exists $\delta>0$ such that $$f(B_\delta(c)) \subseteq B_\epsilon(f(c)),$$
    where $B_\delta(c)=\{x \in \R~|~|x-c|<\delta\}$, called the open interval (or ball) of radius $\delta$ centered at $x$.

    Alternatively:

    Let $f: \R \to \R$ be a real function. Then $f$ is continuous if for all $c \in \R$ and for all $\epsilon > 0$ there exists $\delta>0$ such that $$|f(x)-f(c)<\epsilon|$$ whenever $x \in \R$ and $|x-c|<\delta$.

\medskip
\noindent{\bf Metric}

	Let $X$ be a set. A metric on $X$ is a function $d: X \times X \to \R_{\geq 0}$ satisfying
	\begin{enumerate}
		\item $d(x,x')=0 \iff x=x'$ ($d$ separates points)
		\item $d(x,x') = d(x',x)$ ($d$ is symmetric)
		\item $d(x,x'') \leq d(x,x') + d(x', x'')$ (Triangle inequality)
	\end{enumerate}

\medskip
\noindent{\bf Metric Space}
	
	A metric space is a set equipped with a metric.

\medskip
\noindent{\bf Euclidean Space}

	An $n$-dimensional Euclidean space is the metric space $\R^n \times \R^n, d)$, with $d: \R^n \to \R$ by $$d(x, y) = \sqrt{\sum_{i=1}^n (x_i-y_i)^2}.$$

\medskip
\noindent{\bf Open Set}

	Let $U \subseteq \R$. $U$ is open if for all $u \in U$, there exists $\epsilon > 0$ such that $$B_\epsilon(x) \subseteq U.$$

\medskip
\noindent{\bf Continuity (round 2)}

	A function $f: \R \to \R$ is continuous if the pre-image of open sets is open. In other words, $U \subseteq \R$ is open $\implies f\inv(U) = \{x \in \R~|~f(x) \in U\} \subseteq \R$.

\medskip
\noindent{\bf Topology}

	A topology on a set $X$ is a collection $\tau$ of subsets of $X$ having the following properties:

    \begin{enumerate}
        \item $\emptyset$ and $X$ are in $\tau$.
        \item The union of the elements of any subcollection of $\tau$ is in $\tau$.
        \item The intersection of the elements of any finite subcollection of $\tau$ is in $\tau$.
    \end{enumerate}

\medskip
\noindent{\bf Topological Space}

    A set $X$ for which a topology $\tau$ has been specified is called a topological space.

\medskip
\noindent{\bf Open Set of a Topological Space}

    If $(X, \tau)$ is a topological space, we say that a subset $U$ of $X$ is an open set of $X$ is $U$ belongs to the collection $\tau$.

\medskip
\noindent{\bf Discrete Topology}

    If $X$ is any set, the collection of all subsets of $X$ is a topology on $X$ called the discrete topology.

\medskip
\noindent{\bf Indiscrete/Trivial Topology}

    If $X$ is any set, the trivial (or indiscrete) topology on $X$ is $(X, \{\emptyset, X\})$.

\end{document}
