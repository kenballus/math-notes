\documentclass{article}
\usepackage[utf8]{inputenc}
\usepackage{amsmath}
\usepackage{amsfonts}
\usepackage{amssymb}
\usepackage{graphicx}
\usepackage{geometry}
\usepackage{xcolor}

\newcommand{\inv}{^{-1}}   
\newcommand{\Z}{\mathbb Z}
\newcommand{\R}{\mathbb R}
\newcommand{\Q}{\mathbb Q}
\newcommand{\C}{\mathbb C}
\newcommand{\N}{\mathbb N}

\begin{document}
\pagecolor{black}
\color{white}

\noindent{\bf Continuity}

    A function $f: \R \to \R$ is continuous if for all $c \in \R, \epsilon > 0$, there exists $\delta>0$ such that $$f(B_\delta(c)) \subseteq B_\epsilon(f(c)),$$
    where $B_\delta(c)=\{x \in \R~|~|x-c|<\delta\}$, called the open interval (or ball) of radius $\delta$ centered at $x$.

    Alternatively:

    Let $f: \R \to \R$ be a real function. Then $f$ is continuous if for all $c \in \R$ and for all $\epsilon > 0$ there exists $\delta>0$ such that $$|f(x)-f(c)<\epsilon|$$ whenever $x \in \R$ and $|x-c|<\delta$.

\end{document}
