\documentclass{article}
\usepackage[utf8]{inputenc}
\usepackage{amsmath}
\usepackage{amsfonts}
\usepackage{amssymb}
\usepackage{graphicx}
\usepackage{geometry}
\usepackage{xcolor}

\newcommand{\inv}{^{-1}}   
\newcommand{\Z}{\mathbb Z}
\newcommand{\R}{\mathbb R}
\newcommand{\Q}{\mathbb Q}
\newcommand{\C}{\mathbb C}
\newcommand{\N}{\mathbb N}
\newcommand{\B}{\mathcal B}

\begin{document}
\pagecolor{black}
\color{white}

\noindent{\bf Continuity}

    A function $f: \R \to \R$ is continuous if for all $c \in \R, \epsilon > 0$, there exists $\delta>0$ such that $$f(B_\delta(c)) \subseteq B_\epsilon(f(c)),$$
    where $B_\delta(c)=\{x \in \R~|~|x-c|<\delta\}$, called the open interval (or ball) of radius $\delta$ centered at $x$.

    Alternatively:

    Let $f: \R \to \R$ be a real function. Then $f$ is continuous if for all $c \in \R$ and for all $\epsilon > 0$ there exists $\delta>0$ such that $$|f(x)-f(c)<\epsilon|$$ whenever $x \in \R$ and $|x-c|<\delta$.

\medskip\noindent{\bf Metric}

	Let $X$ be a set. A metric on $X$ is a function $d: X \times X \to \R_{\geq 0}$ satisfying
	\begin{enumerate}
		\item $d(x,x')=0 \iff x=x'$ ($d$ separates points)
		\item $d(x,x') = d(x',x)$ ($d$ is symmetric)
		\item $d(x,x'') \leq d(x,x') + d(x', x'')$ (Triangle inequality)
	\end{enumerate}

\medskip\noindent{\bf Metric Space}
	
	A metric space is a set equipped with a metric.

\medskip\noindent{\bf Euclidean Space}

	An $n$-dimensional Euclidean space is the metric space $\R^n \times \R^n, d)$, with $d: \R^n \to \R$ by $$d(x, y) = \sqrt{\sum_{i=1}^n (x_i-y_i)^2}.$$

\medskip\noindent{\bf Open Set}

	Let $U \subseteq \R$. $U$ is open if for all $u \in U$, there exists $\epsilon > 0$ such that $$B_\epsilon(x) \subseteq U.$$

\medskip\noindent{\bf Continuity (round 2)}

	A function $f: \R \to \R$ is continuous if the pre-image of open sets is open. In other words, $U \subseteq \R$ is open $\implies f\inv(U) = \{x \in \R~|~f(x) \in U\} \subseteq \R$.

\medskip\noindent{\bf Topology}

	A topology on a set $X$ is a collection $\tau$ of subsets of $X$ having the following properties:

    \begin{enumerate}
        \item $\emptyset$ and $X$ are in $\tau$.
        \item The union of the elements of any subcollection of $\tau$ is in $\tau$.
        \item The intersection of the elements of any finite subcollection of $\tau$ is in $\tau$.
    \end{enumerate}

\medskip\noindent{\bf Point/Open Set}

	Elements of a topology are called points, or open sets.

\medskip\noindent{\bf Topological Space}

    A set $X$ for which a topology $\tau$ has been specified is called a topological space.

\medskip\noindent{\bf Discrete Topology}

    If $X$ is any set, the collection of all subsets of $X$ is a topology on $X$ called the discrete topology.

\medskip\noindent{\bf Indiscrete/Trivial Topology}

    If $X$ is any set, the trivial (or indiscrete) topology on $X$ is $(X, \{\emptyset, X\})$.

\medskip\noindent{\bf Closed Set}

	A subset $Z \subseteq X$ is closed if $X \setminus Z$ is open.

\medskip\noindent{\bf Lemma 0.01}

	\begin{enumerate}
		\item $\emptyset$ and $X$ are closed.
		\item Arbitrary intersections of closed sets are closed.
		\item Finite unions of closed sets are closed.
	\end{enumerate}

\medskip\noindent{\bf Lemma 0.02}

	(Use this when you are proving that a collection $\tau$ is a topology.
	
	Let $X$ be a set and $\tau$ a collection of subsets of $X$ such that for all $A,B \in \tau$, $A \cap B \in \tau$, then Axiom 3 of the topology definition is satisfied.

\medskip\noindent{\bf Fine/Coarse}

	A topology $\tau'$ is finer than a topology $\tau$ if $\tau \subseteq \tau'$. We also say that $\tau$ is coarser than $\tau'$. The topologies $\tau$ and $\tau'$ are comparable if $\tau \subseteq \tau'$ or $\tau' \subseteq \tau$.

\medskip\noindent{\bf Complement}

    Let $S \subseteq X$. Then, the complement of $S$ in $X$ is defined by $X \setminus S = X - S = \{x \in X~|~x \notin S\}$. This is what you'd expect.

\medskip\noindent{\bf Cofinite Topology}

    Let $X$ be any infinite set and $\tau_f = \{S \subseteq X~|~X \setminus S \text{ is finite or all of } X\}$. $\tau_f$ is known as the cofinite topology or the finite complement topology.

\medskip\noindent{\bf Basis for a Topology}

    The basis for a topology $\tau$ on a set $X$ is a collection $\mathcal B$ of subsets of $X$ such that
    \begin{enumerate}
        \item For each $x \in X$, there is at least one basis element $B$ containing $x$. ($\mathcal B$ covers $X$)
        \item If $x$ belongs to the intersection of two basic sets $B_1$ and $B_2$, then there exists a basic set $B_3$ that contains $x$ such that $B_3 \subseteq (B_1 \cap B_2)$.
    \end{enumerate}

\medskip\noindent{\bf Topology Generated by a Basis}

    The topology generated by a basis $\mathcal B$ is defined by the following:

    $U \subseteq X$ is in $\tau$ if for each $x \in U$, there exists $B \in \mathcal B$ such that $x \in B$ and $B \subseteq U$. This is analogous to the definition of an open set from real analysis.

\medskip\noindent{\bf Lemma 0.03}

    Let $\mathcal B$ be a basis for a topology on $X$. Define $\tau = \{U \subseteq X ~|~ x \in U \implies x \in B \subseteq U \text{ for some } B \in \mathcal B\}.$ $\tau$ is a topology on $X$.

\medskip\noindent{\bf Standard Topology on $\R$}

    Let $\mathcal B$ be the collection of all open intervals in $\R$. The topology generated by $\mathcal B$ is called the standard topology on $\R$.

\medskip\noindent{\bf Lemma 13.1}

    Let $X$ be a set, and let $\mathcal B$ be a basis for a topology $\tau$ on the set $X$. Then, $\tau$ is the collection of all unions of the elements of $\mathcal B$.

\medskip\noindent{\bf Lemma 13.2}

    Let $(X,\tau)$ be a topological space. If $\mathcal C$ is a collection of elements of $\tau$ such that for each open subset $U \subseteq X$, and each $x \in U$, there is an element $C \in \mathcal C$ such that $x \in C \subseteq U$. Then $\mathcal C$ is a basis for the topology $\tau$ on $X$.

\medskip\noindent{\bf Lemma 13.3}

    Let $\mathcal B$ and $\mathcal B'$ be bases for topologies $\tau$ and $\tau'$, respectively, on $X$. Then, the following are equivalent:

    \begin{enumerate}
        \item $\tau'$ is finer than $\tau$.
        \item For each $x \in X$ and each basis element $B \in \mathcal B$ containing $x$, there is a basis element $B' \in \mathcal B'$ such that $x \in B' \subseteq B$.
    \end{enumerate}

\medskip\noindent{\bf $K$-Topology on $\mathbb R$}

    Let $K = \{\frac1n~|~n\in\mathbb N\}$.

    Let $\mathcal B'' = \{(a,b)~|~a,b\in \mathbb R,a<b\} \cup \{(a,b)\setminus K~|~a,b\in\mathbb R,a<b\}.$

    The topology generated by $\mathcal B''$ on $\mathbb R$ is the $K$-topology on $\mathbb R$, denoted $\R_K$.

\medskip\noindent{\bf Lemma 13.4}

    The lower limit topology and the $K$-topology on $\mathbb R$ are finer than the standard topology on $\mathbb R$, but the lower limit topology is not comparable to the $K$-topology.

\medskip\noindent{\bf Subbasis}

A subbasis $\mathcal S$ for a topology on $X$ is a collection of subsets of $X$ whose union is equal to $X$. The topology generated by the subbasis $\mathcal S$ is defined to be the collection $\tau_{\mathcal S}$ of all unions of finite intersections of elements of $\mathcal S$.

\medskip\noindent{\bf Order Relation}

    A simple order, or order relation, is a relation on a set satisfying the following:
    \begin{enumerate}
        \item Comparability: For every $x,y \in A$ for which $x \neq y$, either $x < y$ or $y < x$.
        \item Nonreflexivity: For no $x \in A$ does $x < x$ hold.
        \item Transitivity: If $x<y$ and $y<z$, then $x<z$.
    \end{enumerate}

\medskip\noindent{\bf Intervals}

    Let $X$ be a set with a simple order relation $<$. The following sets are intervals in $X$:
    \begin{enumerate}
        \item $(a,b) = \{x \in X ~|~ a < x < b\}$ (Open interval)
        \item $[a,b) = \{x \in X ~|~ a < x \leq b\}$ (Half-open interval)
        \item $(a,b] = \{x \in X ~|~ a \leq x < b\}$ (Half-open interval)
        \item $[a,b] = \{x \in X ~|~ a \leq x \leq b\}$ (Closed interval)
    \end{enumerate}

\medskip\noindent{\bf Order Topology}

    Let $X$ be a set with a simple order relation and assume $X$ has more than one element. Ley $\B$ be the collection of all sets of the following types:
    \begin{enumerate}
        \item All open intervals $(a,b)$ in $X$.
        \item All intervals of the form $[a_0,b)$ where $a_0$ is the least element of $X$, if it exists.
        \item All intervals of the form $(a,b_0]$ where $b_0$ is the greatest element of $X$, if it exists.
    \end{enumerate}

    The collection $\B$ is the basis for a topology on $X$ called the order topology.

\medskip\noindent{\bf Rays}

    If $X$ is a set with the simple order relation $<$, and $a \in X$, then there are four subsets of $X$ that are called rays determined by $a$. They are the following:
    \begin{enumerate}
        \item $(a, \infty) = \{x ~|~ a < x\}$
        \item $(-\infty, a) = \{x ~|~ x < a\}$
        \item $[a, \infty = \{x ~|~ a < x \text{ or } a = x\}$
        \item $(-\infty, a] = \{x ~|~ x < a \text{ or } a = x\}$
    \end{enumerate}
    The rays of forms 1 and 2 are called open rays, and the rays of forms 3 and 4 are called closed rays.

\medskip\noindent{\bf Theorem 14.05}

    Let $X$ be a set with a simple order relation $<$. Then, the open rays form a subbasis for the order topology on $X$.

\medskip\noindent{\bf Product Topology}

    Let $X$ and $Y$ be topological spaces. The product topology on $X \times Y$ is the topology with basis $\B$ of all sets of the form $U \times V$, where $U$ is open in $X$ and $V$ is open in $Y$.

\medskip\noindent{\bf Theorem 15.1}

    If $\B$ is a basis for the topology on $X$ and $\mathcal C$ is a basis for the topology on $Y$, then the collection $$\mathcal D = \{B \times C \mid B \in \B, c \in \mathcal C\}$$ is a basis for the topology of $X \times Y$.

\medskip\noindent{\bf Projection and Preimage}

    Let $\pi_1: X \times Y \to X$ be defined by $\pi_1(x,y) = x$.
    Let $\pi_2: X \times Y \to Y$ be defined by $\pi_2(x,y) = y$.
    The maps $\pi_1$ and $\pi_2$ are projections of $X \times Y$ onto its first and second factor, respectively.

    For any $U \in X$, we can consider the preimage of $U$ under $\pi_1$:
    \begin{align*}
        \pi_1\inv(U) &= \{(x,y) \in X \times Y \mid \pi_1(x,y) \in U\} \\
                     &= \{(x,y) \in X \times Y \mid x \in U\} \\
                     &= U \times Y.
    \end{align*}
    We define the preimage of $V$ under $\pi_2$ similarly.

\medskip\noindent{\bf Neighborhood}

    Let $(X, \tau)$ be a topological space.
    The set $N \subseteq X$ is a neighborhood of a point $x \in X$ if $x \in N$ and there exists an open set $U \in \tau$ such that $U \subseteq N$.

\medskip\noindent{\bf Hausdorff Space}

    Let $X$ be a topological space. $X$ is a Hausdorff space if for any two distinct points there exist neighborhoods of each which are disjoint from each other.
\end{document}
