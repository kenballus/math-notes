\documentclass{article}
\usepackage[utf8]{inputenc}
\usepackage{amsmath}
\usepackage{amsfonts}
\usepackage{amssymb}
\usepackage{graphicx}
\usepackage{geometry}
\usepackage{xcolor}

\newcommand{\inv}{^{-1}}   
\newcommand{\Z}{\mathbb Z}
\newcommand{\R}{\mathbb R}
\newcommand{\Q}{\mathbb Q}
\newcommand{\C}{\mathbb C}
\newcommand{\N}{\mathbb N}
\newcommand{\B}{\mathcal B}

\begin{document}
\pagecolor{black}
\color{white}

\noindent\textbf{Continuity}

    A function $f: \R \to \R$ is continuous if for all $c \in \R, \epsilon > 0$, there exists $\delta>0$ such that $$f(B_\delta(c)) \subseteq B_\epsilon(f(c)),$$
    where $B_\delta(c)=\{x \in \R~|~|x-c|<\delta\}$, called the open interval (or ball) of radius $\delta$ centered at $x$.

    Alternatively:

    Let $f: \R \to \R$ be a real function. Then $f$ is continuous if for all $c \in \R$ and for all $\epsilon > 0$ there exists $\delta>0$ such that $$|f(x)-f(c)<\epsilon|$$ whenever $x \in \R$ and $|x-c|<\delta$.

\medskip\noindent\textbf{Metric}

	Let $X$ be a set. A metric on $X$ is a function $d: X \times X \to \R_{\geq 0}$ satisfying
	\begin{enumerate}
		\item $d(x,x')=0 \iff x=x'$ ($d$ separates points)
		\item $d(x,x') = d(x',x)$ ($d$ is symmetric)
		\item $d(x,x'') \leq d(x,x') + d(x', x'')$ (Triangle inequality)
	\end{enumerate}

\medskip\noindent\textbf{Metric Space}
	
	A metric space is a set equipped with a metric.

\medskip\noindent\textbf{Euclidean Space}

	An $n$-dimensional Euclidean space is the metric space $\R^n \times \R^n, d)$, with $d: \R^n \to \R$ by $$d(x, y) = \sqrt{\sum_{i=1}^n (x_i-y_i)^2}.$$

\medskip\noindent\textbf{Open Set}

	Let $U \subseteq \R$. $U$ is open if for all $u \in U$, there exists $\epsilon > 0$ such that $$B_\epsilon(x) \subseteq U.$$

\medskip\noindent\textbf{Continuity (round 2)}

	A function $f: \R \to \R$ is continuous if the pre-image of open sets is open. In other words, $U \subseteq \R$ is open $\implies f\inv(U) = \{x \in \R~|~f(x) \in U\} \subseteq \R$.

\medskip\noindent\textbf{Topology}

	A topology on a set $X$ is a collection $\tau$ of subsets of $X$ having the following properties:

    \begin{enumerate}
        \item $\emptyset$ and $X$ are in $\tau$.
        \item The union of the elements of any subcollection of $\tau$ is in $\tau$.
        \item The intersection of the elements of any finite subcollection of $\tau$ is in $\tau$.
    \end{enumerate}

\medskip\noindent\textbf{Point/Open Set}

	Elements of a topology are called points, or open sets.

\medskip\noindent\textbf{Topological Space}

    A set $X$ for which a topology $\tau$ has been specified is called a topological space.

\medskip\noindent\textbf{Discrete Topology}

    If $X$ is any set, the collection of all subsets of $X$ is a topology on $X$ called the discrete topology.

\medskip\noindent\textbf{Indiscrete/Trivial Topology}

    If $X$ is any set, the trivial (or indiscrete) topology on $X$ is $(X, \{\emptyset, X\})$.

\medskip\noindent\textbf{Closed Set}

	A subset $Z \subseteq X$ is closed if $X \setminus Z$ is open.

\medskip\noindent\textbf{Lemma 0.01}

	\begin{enumerate}
		\item $\emptyset$ and $X$ are closed.
		\item Arbitrary intersections of closed sets are closed.
		\item Finite unions of closed sets are closed.
	\end{enumerate}

\medskip\noindent\textbf{Lemma 0.02}

	(Use this when you are proving that a collection $\tau$ is a topology.
	
	Let $X$ be a set and $\tau$ a collection of subsets of $X$ such that for all $A,B \in \tau$, $A \cap B \in \tau$, then Axiom 3 of the topology definition is satisfied.

\medskip\noindent\textbf{Fine/Coarse}

	A topology $\tau'$ is finer than a topology $\tau$ if $\tau \subseteq \tau'$. We also say that $\tau$ is coarser than $\tau'$. The topologies $\tau$ and $\tau'$ are comparable if $\tau \subseteq \tau'$ or $\tau' \subseteq \tau$.

\medskip\noindent\textbf{Complement}

    Let $S \subseteq X$. Then, the complement of $S$ in $X$ is defined by $X \setminus S = X - S = \{x \in X~|~x \notin S\}$. This is what you'd expect.

\medskip\noindent\textbf{Cofinite Topology}

    Let $X$ be any infinite set and $\tau_f = \{S \subseteq X~|~X \setminus S \text{ is finite or all of } X\}$. $\tau_f$ is known as the cofinite topology or the finite complement topology.

\medskip\noindent\textbf{Basis for a Topology}

    The basis for a topology $\tau$ on a set $X$ is a collection $\mathcal B$ of subsets of $X$ such that
    \begin{enumerate}
        \item For each $x \in X$, there is at least one basis element $B$ containing $x$. ($\mathcal B$ covers $X$)
        \item If $x$ belongs to the intersection of two basic sets $B_1$ and $B_2$, then there exists a basic set $B_3$ that contains $x$ such that $B_3 \subseteq (B_1 \cap B_2)$.
    \end{enumerate}

\medskip\noindent\textbf{Topology Generated by a Basis}

    The topology generated by a basis $\mathcal B$ is defined by the following:

    $U \subseteq X$ is in $\tau$ if for each $x \in U$, there exists $B \in \mathcal B$ such that $x \in B$ and $B \subseteq U$. This is analogous to the definition of an open set from real analysis.

\medskip\noindent\textbf{Lemma 0.03}

    Let $\mathcal B$ be a basis for a topology on $X$. Define $\tau = \{U \subseteq X ~|~ x \in U \implies x \in B \subseteq U \text{ for some } B \in \mathcal B\}.$ $\tau$ is a topology on $X$.

\medskip\noindent\textbf{Standard Topology on $\R$}

    Let $\mathcal B$ be the collection of all open intervals in $\R$. The topology generated by $\mathcal B$ is called the standard topology on $\R$.

\medskip\noindent\textbf{Lemma 13.1}

    Let $X$ be a set, and let $\mathcal B$ be a basis for a topology $\tau$ on the set $X$. Then, $\tau$ is the collection of all unions of the elements of $\mathcal B$.

\medskip\noindent\textbf{Lemma 13.2}

    Let $(X,\tau)$ be a topological space. If $\mathcal C$ is a collection of elements of $\tau$ such that for each open subset $U \subseteq X$, and each $x \in U$, there is an element $C \in \mathcal C$ such that $x \in C \subseteq U$. Then $\mathcal C$ is a basis for the topology $\tau$ on $X$.

\medskip\noindent\textbf{Lemma 13.3}

    Let $\mathcal B$ and $\mathcal B'$ be bases for topologies $\tau$ and $\tau'$, respectively, on $X$. Then, the following are equivalent:

    \begin{enumerate}
        \item $\tau'$ is finer than $\tau$.
        \item For each $x \in X$ and each basis element $B \in \mathcal B$ containing $x$, there is a basis element $B' \in \mathcal B'$ such that $x \in B' \subseteq B$.
    \end{enumerate}

\medskip\noindent\textbf{$K$-Topology on $\mathbb R$}

    Let $K = \{\frac1n~|~n\in\mathbb N\}$.

    Let $\mathcal B'' = \{(a,b)~|~a,b\in \mathbb R,a<b\} \cup \{(a,b)\setminus K~|~a,b\in\mathbb R,a<b\}.$

    The topology generated by $\mathcal B''$ on $\mathbb R$ is the $K$-topology on $\mathbb R$, denoted $\R_K$.

\medskip\noindent\textbf{Lemma 13.4}

    The lower limit topology and the $K$-topology on $\mathbb R$ are finer than the standard topology on $\mathbb R$, but the lower limit topology is not comparable to the $K$-topology.

\medskip\noindent\textbf{Subbasis}

A subbasis $\mathcal S$ for a topology on $X$ is a collection of subsets of $X$ whose union is equal to $X$. The topology generated by the subbasis $\mathcal S$ is defined to be the collection $\tau_{\mathcal S}$ of all unions of finite intersections of elements of $\mathcal S$.

\medskip\noindent\textbf{Order Relation}

    A simple order, or order relation, is a relation on a set satisfying the following:
    \begin{enumerate}
        \item Comparability: For every $x,y \in A$ for which $x \neq y$, either $x < y$ or $y < x$.
        \item Nonreflexivity: For no $x \in A$ does $x < x$ hold.
        \item Transitivity: If $x<y$ and $y<z$, then $x<z$.
    \end{enumerate}

\medskip\noindent\textbf{Intervals}

    Let $X$ be a set with a simple order relation $<$. The following sets are intervals in $X$:
    \begin{enumerate}
        \item $(a,b) = \{x \in X ~|~ a < x < b\}$ (Open interval)
        \item $[a,b) = \{x \in X ~|~ a < x \leq b\}$ (Half-open interval)
        \item $(a,b] = \{x \in X ~|~ a \leq x < b\}$ (Half-open interval)
        \item $[a,b] = \{x \in X ~|~ a \leq x \leq b\}$ (Closed interval)
    \end{enumerate}

\medskip\noindent\textbf{Order Topology}

    Let $X$ be a set with a simple order relation and assume $X$ has more than one element. Let $\B$ be the collection of all sets of the following types:
    \begin{enumerate}
        \item All open intervals $(a,b)$ in $X$.
        \item All intervals of the form $[a_0,b)$ where $a_0$ is the least element of $X$, if it exists.
        \item All intervals of the form $(a,b_0]$ where $b_0$ is the greatest element of $X$, if it exists.
    \end{enumerate}

    The collection $\B$ is the basis for a topology on $X$ called the order topology.

\medskip\noindent\textbf{Rays}

    If $X$ is a set with the simple order relation $<$, and $a \in X$, then there are four subsets of $X$ that are called rays determined by $a$. They are the following:
    \begin{enumerate}
        \item $(a, \infty) = \{x ~|~ a < x\}$
        \item $(-\infty, a) = \{x ~|~ x < a\}$
        \item $[a, \infty) = \{x ~|~ a < x \text{ or } a = x\}$
        \item $(-\infty, a] = \{x ~|~ x < a \text{ or } a = x\}$
    \end{enumerate}
    The rays of forms 1 and 2 are called open rays, and the rays of forms 3 and 4 are called closed rays.

\medskip\noindent\textbf{Theorem 14.05}

    Let $X$ be a set with a simple order relation $<$. Then, the open rays form a subbasis for the order topology on $X$.

\medskip\noindent\textbf{Product Topology}

    Let $X$ and $Y$ be topological spaces. The product topology on $X \times Y$ is the topology with basis $\B$ of all sets of the form $U \times V$, where $U$ is open in $X$ and $V$ is open in $Y$.

\medskip\noindent\textbf{Theorem 15.1}

    If $\B$ is a basis for the topology on $X$ and $\mathcal C$ is a basis for the topology on $Y$, then the collection $$\mathcal D = \{B \times C \mid B \in \B, c \in \mathcal C\}$$ is a basis for the topology of $X \times Y$.

\medskip\noindent\textbf{Projection and Preimage}

    Let $\pi_1: X \times Y \to X$ be defined by $\pi_1(x,y) = x$.
    Let $\pi_2: X \times Y \to Y$ be defined by $\pi_2(x,y) = y$.
    The maps $\pi_1$ and $\pi_2$ are projections of $X \times Y$ onto its first and second factor, respectively.

    For any $U \in X$, we can consider the preimage of $U$ under $\pi_1$:
    \begin{align*}
        \pi_1\inv(U) &= \{(x,y) \in X \times Y \mid \pi_1(x,y) \in U\} \\
                     &= \{(x,y) \in X \times Y \mid x \in U\} \\
                     &= U \times Y.
    \end{align*}
    We define the preimage of $V$ under $\pi_2$ similarly.

\medskip\noindent\textbf{Theorem 15.2}

    The set $$\mathcal S = \{\pi_1\inv(U) \mid U \text{ is open in } X\} \cup \{\pi_2\inv(V) \mid V \text{ is open in } Y\}$$ is a subbasis for the product topology on $X \times Y$.

\medskip\noindent\textbf{Subspace Topology}

    Let $(X,\tau)$ be a topological space. If $Y$ is a subset of $X$, then the set $\tau_Y = \{Y \cap U \mid U \in \tau\}$ is a topology on $Y$ called the subspace topology. With this topology, $Y$ is called a subspace of $X$.

\medskip\noindent\textbf{Lemma 16.1}

    If $\B$ is a basis for the topology on $X$, then the set $\B_Y = \{B \cap Y \mid B \in \B\}$ is a basis for the subspace topology on $Y$.

\medskip\noindent\textbf{Lemma 16.2}

    Let $Y$ be a subspace of $X$. If $U$ is open in $Y$, and $Y$ is open in $X$, then $U$ is open in $X$.

\medskip\noindent\textbf{Lemma 16.3}

    If $A$ is a subspace of $X$ and $B$ is a subspace of $Y$, then the product topology on $A \times B$ is the same as the topology as the topology $A \times B$ inherits as a subspace of $X \times Y$.

\medskip\noindent\textbf{Convex}

    Given an order set $X$, a subset $Y \subseteq X$ is convex in $X$ if for each pair of points $a,b \in Y$ with $a<b$, the entire interval lies in $Y$.

\medskip\noindent\textbf{Theorem 16.4}

    Let $X$ be an ordered set with the order topology.
    Let $Y$ be a subset of $X$ that is convex in $X$.
    Then, the order topology on $Y$ is the same as the subspace topology on $Y$.

\medskip\noindent\textbf{Theorem 17.1}

    Let $X$ be a topological space. Then, the following conditions hold:
    \begin{enumerate}
        \item $\emptyset$ and $X$ are closed.
        \item Arbitrary intersections of closed sets are closed.
        \item Finite unions of closed sets are closed.
    \end{enumerate}

\medskip\noindent\textbf{Theorem 17.2}

    Let $Y$ be a subspace of $X$. Then, a set $A$ is closed in $Y$ if and only if it equals the intersection of a closed set of $X$ with $Y$.

\medskip\noindent\textbf{Theorem 17.3}

    Let $Y$ be a subspace of $X$. If $A$ is closed in $Y$ and $Y$ is closed in $X$, then $A$ is closed in $X$.

\medskip\noindent\textbf{Interior and Closure}

    Given a subset $A$ of a topological space $X$, then interior of $A$, denoted $\text{int}(A)$ or $\mathring A$, is the union of all open sets contained in $A$.
    The closure of $A$, denoted $\overline A$ or $\text{cl}(A)$, is the intersection of all closed sets containing $A$.

\medskip\noindent\textbf{Lemma 17.06}

    Let $A$ be a subset of a topological space $X$. Then, $A$ is open if and only if $A = \text{Int(A)}$, and $A$ is closed if and only if $A=\overline A$.

\medskip\noindent\textbf{Theorem 17.4}

    Let $Y$ be a subspace of $X$. Let $A \subseteq Y$ and denote the closure of $A$ in $X$ as $\overline A$. Then, the closure of $A$ in $Y$ equals $\overline A \cap Y$.

\medskip\noindent\textbf{Intersect}

    A set $A$ intersects a set $B$ if $A \cap B \neq \emptyset$.

\medskip\noindent\textbf{Neighborhood}

    An open set $U \subseteq X$ is a neighborhood of $x \in X$ if $x \in U$.

\medskip\noindent\textbf{Theorem 17.5}

    Let $A$ be a subset of a topological space $X$.
    Then, $x \in \overline A$ if and only if every neighborhood of $x$ intersects $A$.

    Supposing the topology of $X$ is given by a basis, then $x \in \overline A$ if and only if every basis element $B$ containing $x$ intersects $A$.

\medskip\noindent\textbf{Limit Point}

    If $A$ is a subset of a topological space $X$ and if $x \in X$, then $x$ is a limit point (or cluster point or point of accumulation) of $A$ if every neighborhood of $x$ intersects $A$ in some point other than $x$ itself.

\medskip\noindent\textbf{Convergent Sequences}

    Let $x_1, x_2, \hdots$ be a sequence of points of a topological space $X$.
    The sequence converges to $x \in X$ if for every neighborhood $U$ of $x$ there exists $N \in \mathbb N$ such that for all $n \geq N$, we have $x_n \in U$. 
    In this case, $x$ is called a limit point of the sequence.

\medskip\noindent\textbf{Hausforff Space}

    A topological space $X$ is a Hausdorff space if for each pair of distinct points $x_1, x_2 \in X$, there exist neighborhoods $U_1$ of $x_1$ and $U_2$ of $x_2$ such that $U_1 \cap U_2 = \emptyset$.

\medskip\noindent\textbf{Theorem 17.8}

    Every finite point set in a Hausdorff space $X$ is closed. In particular, singletons are closed sets in a Hausdorff space.

\medskip\noindent\textbf{$T_1$ Axiom}

    The condition that finite point sets be closed is called the $T_1$ axiom. Chapter 4 contains more of these conditions, known as the separation axioms.

\medskip\noindent\textbf{Theorem 17.9}

    Let $X$ be a space satisfying the $T_1$ axiom; let $A$ be a subset of $X$. Then, the point $x$ is a limit point of $A$ if and only if every neighborhood of $x$ contains infinitely many points of $A$.

\medskip\noindent\textbf{Theorem 17.10}

    If $X$ is a Hausdorff space, then a sequence of points of $X$ converges to at most one point of $X$.

\medskip\noindent\textbf{Theorem 17.11}
    \begin{enumerate}
    \item Every simply ordered set is a Hausdorff space in the order topology.
    \item The product topology of two Hausdorff spaces is a Hausdorff space.
    \item A subspace of a Hausdorff space is a Hausdorff space.
    \end{enumerate}

\medskip\noindent\textbf{Continuity}

    Let $X$ and $Y$ be topological spaces. A function $f: X \to Y$ is continuous if for each open set $V$ of $Y$, the preimage of $V$ under $f$ is open in $X$.

\medskip\noindent\textbf{Theorem 18.07}

    A function $f: X \to Y$ is continuous if and only if for each basis element $B \in \B$ for the topology on $Y$, the preimage of $B$ is open in $X$.

\medskip\noindent\textbf{Theorem 18.1}

    Let $X$ and $Y$ be topological spaces. Let $f: X \to Y$. Then the following are equivalent:
    \begin{enumerate}
        \item $f$ is continuous.
        \item For every subset $A$ of $X$, the image of the closure of $A$ is a subset of the closure of the image of $A$.
        \item For every closed subset $B$ of $Y$, the preimage of $B$ under $f$ is closed in $X$.
        \item For each $x \in X$ and each neighborhood $V$ of $f(x)$, there is a neighborhood $U$ of $x$ such that $f(u) \subseteq V$.
    \end{enumerate}

\medskip\noindent\textbf{Theorem 17.6}

    Let $A$ be a subset of a topological space $X$.
    Let $A'$ be the set of all limit points of $A$.
    Then $\overline A = A \cup A'$.

\medskip\noindent\textbf{Corollary 17.7}

    A subset of a topological space is closed if and only if it contains all of its limit points.

\medskip\noindent\textbf{Homeomorphism}

    Let $X,Y$ be topological spaces. Let $f: X \to Y$ be a bijection. If both $f$ and $f\inv$ are continuous (we say $f$ is bicontinuous) then $f$ is a homeomorphism.

\medskip\noindent\textbf{Topological Property}

    Any property in a topological space $X$ that is expressed entirely in terms of the topology on $X$ yields through the homeomorphism the corresponding property in $Y$. Such properties are called topological properties.

    Examples: open/closed, limit points of sets, limits of sequences, a basis and subbasis, connectedness, compactness.

\medskip\noindent\textbf{Theorem 18.2}

    \begin{enumerate}
        \item (constant function) If $f: X \to Y$ maps all of $X$ to a point $y_0 \in Y$, then $f$ is continuous.
        \item (inclusion) If $A$ is a subspace of $X$, then inclusion function $j: A \to X$ is continuous.
        \item (composites) If $f: X \to Y$ and $g: Y \to Z$ are continuous, then the map $g \circ f: X \to Z$ is continuous.
        \item (Restriction of domain) If $f: X \to Y$ is continuous and $A$ is a subspace of $X$, then $f|_A: A \to Y$ is continuous.
        \item (Restriction or expansion of codomain) Let $f: X \to Y$ be continuous. If $Z$ is a subspace of $Y$ containing the image set $f(X)$, then the function $g: X \to Z$ obtained from restricting the range of $f$ is continuous.
        If $Z$ is a space having $Y$ as a subspace, then the function $g: X \to Z$ obtained by expanding the range of $f$ is continuous, too.
        \item (local formulation of continuity) The map $f: X \to Y$ is continuous if $X$ can be written as the union of open sets $U_i$ such that $f|_{U_i}$ is containuous for each $i$.
    \end{enumerate}

\medskip\noindent\textbf{Theorem 18.3 (The pasting Lemma)}

    Let $X = A \cup B$, where $A$ and $B$ are closed sets in a topological space $X$.
    Let $f: A \to Y$ and $g: B \to Y$ be continuous.
    If $f(x) = g(x)$ for all $x \in A \cap B$, then $f$ and $g$ combine to give a continuous function $h: X \to Y$ defined by
    $$h(x) = \begin{cases} f(x) & x \in A \\ g(x) & x \in B \end{cases}.$$

\medskip\noindent\textbf{Theorem 18.4}

    Let $f: A \to X \times Y$ given by $f(t) = (f_1(t), f_2(t))$ where $f_1: A \to X$ and $f_2: A \to Y$. Then $f$ is continuous if and only if $f_1$ and $f_2$ are continuous.

\medskip\noindent\textbf{Ball}

    Let $x \in M$, a metric space, and let $\epsilon > 0$. Then, the $\epsilon$-ball centered at $x$ is the set of all points $y$ in $M$ with $d(x,y) \leq \epsilon$.

\medskip\noindent\textbf{Metric Topology}

    If $d$ is a metric on $X$, then the collection of $\epsilon$-balls for all $x \in X$ and all $\epsilon > 0$ is a basis for a topology on $X$, called the metric topology induced by $d$.

\medskip\noindent\textbf{Lemma 20.08}

    Let $B_d(x, \epsilon)$ be an $\epsilon$-ball in a topological space with the metric topology and metric $d$. Let $y \in B_d(x, \epsilon)$. Then, there is $\delta > 0$ such that $B_d(y, \delta) \subseteq B_d(x,\epsilon)$.

\medskip\noindent\textbf{Lemma 20.09}

    A set $U$ is open in the metric topology induced by $d$ if and only if for each $y \in U$, there exists $\delta > 0$ such that $B_d(y, \delta) \subseteq U$.

\medskip\noindent\textbf{Theorem 20.1}

	Let $f: X \to Y$. Let $X$ and $Y$ be metric spaces with metrics $d_X$ and $d_Y$.
	Then, continuity of $f$ is equivalent to the requirement that given $x \in X$ and given $\epsilon > 0$, there exists $\delta > 0$ such that $$d_X(x,y) < \delta \implies d_Y(f(x),f(y))< \epsilon.$$

\medskip\noindent\textbf{Lemma 20.2}

	Let $d, d'$ be two metrics on $X$. Let $\tau$ and $\tau'$ be the topologies induced by $d$ and $d'$, respectively. Then, $\tau'$ is finer than $\tau$ if and only if for each $x \in X$ and each $\epsilon > 0$, there exists $\delta > 0$ such that $$B_{d'}(x,\delta) \subseteq B_d(x,\epsilon).$$

\medskip\noindent\textbf{Square Metric}

	The square metric on $\R^2$ is defined by $\rho(\mathbf x,\mathbf y) = \max\{|x_1-y_1|, |x_2-y_2|\}$.

\medskip\noindent\textbf{Theorem 20.3}

	The topologies on $\R^2$ induced by the Euclidean metric and the square metric are the same as the product topology on $\R^2$.

\medskip\noindent\textbf{Quotient Map}

	Let $X$ and $Y$ be topological spaces.
	Let $p: X \to Y$ be a surjective (onto) map.
	The map $p$ is a quotient map provided a subset $W$ of $Y$ is open in $Y$ if and only if $p\inv(W)$ is open in $X$.
	Equivalently, $p$ is a quotient map provided a subset $A$ of $Y$ is closed in $Y$ if and only if $p\inv(A)$ is closed in $X$.

\medskip\noindent\textbf{Theorem 22.010}

	If $p: X \to Y$ is continuous, surjective, and open or closed, then $p$ is a quotient map.

\medskip\noindent\textbf{Quotient Topology}

	If $X$ is a space, $A$ is a set, and $p:X \to A$ is surjective, then there exists exactly one topology $\tau$ on $A$ relative to which $p$ is a quotient map. This topology is called the quotient topology induced by $p$. Note that this topology is the collection of all subsets of $A$ where $p\inv(B)$ is open in $X$.

\medskip\noindent\textbf{Partition}

	Let $X$ be a topological space.
	A partition of $X$ is a collection $\mathcal P=\{P_\alpha\}$ of disjoint nonempty subsets of $X$ such that $\bigcup\limits_{\alpha\in A}P_\alpha = X$.

\medskip\noindent\textbf{Quotient Space}

	Let $X$ be a topological space and $X^*$ be a partition of $X$ into disjoint subsets whose union is $X$. Let $p: X \to X^*$ be a surhective map that carries each point of $X$ to the element of $X^*$ containing it. In the quotient topology induced by $p$, the space $X^*$ is called the quotient space of $X$.

\medskip\noindent\textbf{Separation}

	Let $X$ be a topological space.
	A separation of $X$ is a pair $U,V$ of disjoint nonempty open subsets of $X$ whose unions is $X$.

\medskip\noindent\textbf{Connected}
	A topological space $X$ is connected if there is no separation of $X$.

\medskip\noindent\textbf{Theorem 23.011}

	A topological space $X$ is connected if and only if the only subsets of $X$ that are clopen in $X$ are $\emptyset$ and $X$.

\medskip\noindent\textbf{Theorem 23.012}

	Let $f: X \to Y$ be a surjective continuous map.
	If $X$ is connected, then $Y$ is connected.

\medskip\noindent\textbf{Corollary 23.013}

	If $f: X \to Y$ is a homeomorphism, then $X$ is connected if and only if $Y$ is connected.

\medskip\noindent\textbf{Lemma 23.2}

	If sets $C, D$ form a separation of $X$, and $Y$ is a connected subspace of $X$, then $Y$ lies entirely in either $C$ or $D$.

\medskip\noindent\textbf{Theorem 23.5}

	The image of a connected space under a continuous map is connected.

\medskip\noindent\textbf{Theorem 24.1}

	The real line is a connected topological space.

\medskip\noindent\textbf{Theorem 23.4}

	Let $A$ be a connected subspace of $X$. If $A \subseteq B \subseteq \overline A$, then $B$ is connected.

\medskip\noindent\textbf{Theorem 23.3}

	The union of a collection of connected subspaces of $X$ that have a point in common is connected.

\medskip\noindent\textbf{Theorem 23.6}

	The product of two connected spaces is connected.



\end{document}
