\documentclass{article}
\usepackage[utf8]{inputenc}
\usepackage{amsmath, amsfonts, amssymb, amsthm}
\usepackage{graphicx}
\usepackage{geometry}
\usepackage{xcolor}

\newcommand{\inv}{^{-1}}   
\newcommand{\Z}{\mathbb Z}
\newcommand{\R}{\mathbb R}
\newcommand{\Q}{\mathbb Q}
\newcommand{\C}{\mathbb C}
\newcommand{\N}{\mathbb N}

\date{Due Friday, February 12, 2021}
\author{Ben Kallus}
\title{Topology \\ Homework 1}

\begin{document}
\pagecolor{black}
\color{white}
\maketitle

\noindent{\bf 1.}
\begin{proof}
    Let $\tau = \{A \subset \R~|~\R \setminus A \text{ is a finite set or } \R\}$.
    We claim that $(\R, \tau)$ is a topological space.

    Since $\R \setminus \R = \emptyset$, which is a finite set, $\R \in \tau$.
    Since $\R \setminus \emptyset = \R$, $\emptyset \in \tau$.

    Let $A,B \in \tau$.
    By DeMorgan's Law, $\R \setminus (A \cap B) = (\R \setminus A) \cup (\R \setminus B)$.
    Since $\R \setminus A$ and $\R \setminus B$ are either finite or $\R$, their union is therefore either finite or $\R$.
    Thus, $A \cap B \in \tau$.
    Therefore, by Lemma 0.02, $\tau$ contains all finite intersections of its elements.
    
    Let $C \subseteq \tau$, and let $x \in C$.
    Let $U$ be the union of the elements of $C$.
    Suppose $U = \emptyset$.
    In this case, $U \in \tau$, as shown above.
    Now, suppose $U \neq \emptyset$.
    Then, there exists a nonempty $x \in C$.
    Therefore, $\R \setminus x \neq \R$, and is thus finite.
    Since $x \subseteq U$, $\R \setminus U \subset \R \setminus x$.
    Thus, since $\R \setminus x$ is finite, $\R \setminus U$ is also finite.
    Therefore, $U \in \tau$ in this case, as well.

    Thus, $\tau$ is a topology on $\R$.
\end{proof}

\bigskip
\noindent{\bf 2.}
\begin{proof}
    Let $(X, \tau)$ be a topological space, and let $A$ be a subset of $X$ such that for each $x \in A$, there exists an open set $U_x \subseteq A$ containing $x$.
    We claim that $A$ is an open set.

    Observe that $\bigcup\limits_{x \in A} U_x \subseteq A$, since $U_x \subseteq A$ for all $x \in A$.
    Next, observe that $A \subseteq \bigcup\limits_{x \in A} U_x$, since $x \in U_x$ for all $x \in A$.
    Thus, $A = \bigcup\limits_{x \in A} U_x$.
    Therefore, by the definition of a topology, $A$ is an open set.
\end{proof}

\bigskip
\noindent{\bf 3.} No, since $X \notin \tau$.

\newpage
\noindent{\bf 4.}

{\bf a.}
\begin{proof}
    Let $\mathcal B = \{[a,b) ~|~ a,b \in \R\}$. We claim that $\mathcal B$ is a basis for a topology on $\R$.

    Let $a \in \R$.
    Observe that $a \in [a, a+1) \in \mathcal B$.
    Thus, $\mathcal B$ covers $\R$.
    Let $[s, t), [v, w) \in \mathcal B$, and suppose, without loss of generality, that $s \leq v$.
    Let $B_0 = [s, t) \cap [v, w)$.
    Then, $B_0$ is one of $\emptyset, [v,w), [v,t)$.
    Thus, $B_0 \in \mathcal B$.
    Thus, $\mathcal B$ is a basis for a topology on $\R$.
\end{proof}

\medskip
{\bf b.} The half-open topology can be generated from $\mathcal B$ by taking all unions of elements of $\mathcal B$.

\bigskip
\noindent{\bf 5.}

{\bf a.} No.

\medskip
{\bf b.} No.

\medskip
{\bf c.} No.

\medskip
{\bf d.} Yes.

\medskip
{\bf e.} No.


\end{document}