\documentclass[12pt]{article}
\usepackage[english]{babel}
\usepackage[utf8]{inputenc}
\usepackage{amsmath, amssymb,amsthm}
\usepackage{graphicx}
\usepackage{hyperref}
\usepackage{geometry}
\usepackage{xcolor}

\setlength{\topmargin}{0pt}
\setlength{\headsep}{0pt}
\textheight = 600pt

\title{Miscellaneous Notes}
\author{Ben Kallus}

\begin{document}
\pagecolor{black}
\color{white}
\maketitle

\section*{Warning:} These are notes on topics that I have not (yet) studied in class. It is likely that some of them are wrong. This file is for accumulating definitions that I would otherwise forget, but may be useful to me in the future.

\bigskip
\noindent{\bf Metric}

    A metric $f$ is a function that defines a concept of distance between any two members of a set $S$. A metric satisfies the following properties for all $a, b \in S$:
    \begin{align*}
        f(a, b) &= 0 \implies a = b, \\
        f(a, b) &= f(b, a), \\
        |f(a, b)| &= f(a, b), \\
        f(a, b) &\leq f(a, c) + f(c, b).
    \end{align*}

\medskip
\noindent{\bf Metric Space}

    A metric space is a set $S$ together with a metric on $S$.

\medskip
\noindent{\bf Compactness}

    A space is considered compact if every infinite subsequence of points sampled from the space has an infinite subsequence that converges to some point of the space. Bolzano-Weierstrass tells us that $\mathbb R$ has this property, so $\mathbb R$ is compact. There are other notions of compactness, but I think this is the one that I will care about for now.

\medskip
\noindent{\bf Neighborhood (Topology)}

    Let $p \in S$, a set. A neighborhood $N$ of $p$ is a subset of $S$ containing an open subset of $S$ containing $p$. For example, $[1,5]$ is a neighborhood of $3 \in \mathbb R$.

\newpage
\noindent{\bf Topological Space}

    Let $S$ be (potentially empty) set. Let $\mathbf N$ be a function mapping each $p \in S$ to a set of subsets of $S$, which we'll call neighborhoods. $S$ is a topological space if the following all hold:
    \begin{itemize}
        \item $p \in N$ for all $N \in \mathbf N(p)$.
        \item If $M \subseteq S$ and $N \subseteq M$ for some $N \in \mathbf N(p)$, then $M$ is a neighborhood of $p$.
        \item For all $N_1, N_2 \in \mathbf N(p)$, $N_1 \cap N_2 \in \mathbf N(p)$.
        \item For all $N \in \mathbf N(p)$, there exists $M \in \mathbf N(p)$ such that $M \subseteq N$ and $N \in \mathbf N(m)$ for all $m \in M$.
    \end{itemize}

\medskip
\noindent{\bf Discrete Space}

    A discrete space is a topological space in which all subsets are open.

\medskip
\noindent{\bf Category}

    A category $C$ consists of a class ob$(C)$ of objects and a class of arrows hom$(C)$ between the objects such that arrows can be composed associatively, and there exists an arrow $\text{id}_X$ from $X$ to $X$ for all $X \in \text{ob}(C)$.

    Examples:
    \begin{itemize}
        \item The category of sets, in which the objects are sets and the arrows are unary functions.
        \item The category of rings, in which the objects are rings and the arrows are ring homomorphisms.
    \end{itemize}

\medskip
\noindent{\bf Functor}

    Let $C, D$ be categories. A functor $F$ from $C$ to $D$ is a mapping such that
    \begin{itemize}
        \item For each object $X \in \text{ob}(C)$, $F(X) \in \text{ob}(D)$.
        \item For each morphism $f: X \to Y$ for $X,Y \in \text{ob}(C)$, $F(f) : F(X) \to F(Y)$.
        \item $F(\text{id}_X) = \text{id}_{F(X)}$ for all $X \in \text{ob}(C)$.
        \item $F(g \circ f) = F(g) \circ F(f)$ for all $f : X \to Y, g : Y \to Z \in \text{ob}(C)$.
    \end{itemize}

    In other words, a functor is a mapping from one category to another that preserves composition of arrows and plays nice with identity arrows.

\medskip
\noindent{\bf Endofunctor}

    An endofunctor is a functor from a category $C$ to itself.

\end{document}