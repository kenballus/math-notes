\documentclass[11pt]{article}       % The percent symbol in your code starts a comment.  The comment ends at the next linebreak.
\usepackage[english]{babel}         % Packages add functionality and style conventions to your documents. Don't edit this section!
\usepackage{fullpage}               % Eliminates wasted space
\usepackage[utf8]{inputenc}         % Necessary for character encoding
\usepackage{amsmath, amssymb,amsthm}% Required math packages
\usepackage{graphicx}               % For handling graphics
\usepackage[colorinlistoftodos]{todonotes}  % For the fancy "todo" stuff
\usepackage{hyperref}               % For clickable links in the final PDF
\usepackage{fourier}
\usepackage{tikz}
\theoremstyle{definition}
\newtheorem{theorem}{Theorem}
\newtheorem{lemma}[theorem]{Lemma}
\newtheorem{prop}[theorem]{Proposition}
\newtheorem{claim}[theorem]{Claim}

\title{Complex Analysis -- Homework \#14}

\author{Feldman, Kallus, Komissar}

\date{March 12 2021}

\begin{document}

\maketitle

\noindent{\bf 1. }  The unit circle in the plane centered at the origin is pictured below.  Also, $\overline{ON} \parallel \overline{RP}$, and $a$ and $b$ are points on the $x$-axis.  Prove that the product of the numbers $a$ and $b$ is $1$.

\centerline{
\begin{tikzpicture}[xscale=2, yscale=2]
\draw [<->] (0,-1.25) -- (0,1.5);
\draw [<->] (-2.25,0) -- (2.5,0);
\node [ right] at (2.5,0)  { {\small $x$}};
\node [ above right] at (0,1.25)  { {\small $y$}};
\node [ below] at (.8, -3/5)  { {\small $R$}};
\node [ above] at (.735, -.425)  { {\tiny $2$}};
\node [ right] at (-0.02,0.715)  { {\tiny $3$}};
\node [ below left ] at (0,0)  { {\small $O$}};
\node [ above left] at (0.05,0.95) { {\small $N$}};
\node [ above] at (.9, 3/5)  { {\small $P$}};
\node [ below left ] at (-0.9,0.05)  { {\small $W$}};
\node [ below right ] at (0.9,0.05)  { {\small $E$}};
\draw (0,0) circle (1cm);
\draw[blue,dashed,-] (0.8,-3/5) -- (0,1);
\draw[red,dashed,-] (0,1) -- (2,0);
\draw[red,thick,-] (-1,0) -- (2,0);
\draw[blue, dashed,-] (4/5, 3/5) -- (4/5, -3/5);
\node [ below] at (2,0) { {\small  $a$}};
\node [ above ] at (1.7,-0.04) { {\tiny  $1$}};
\node [ below] at (0.45,0) { {\small $b$ }};
\draw[red,thick] (0,1) arc (90:180:1cm);
\draw[red,thick] (1,0) arc (0:35:1cm);]
\draw[blue,thick] (0,1) arc (90:35:1cm);
\end{tikzpicture}
}

\begin{proof}
To begin, the measure of the exterior angle $\angle 1$ to the circle satisfies the following:
\begin{align}
\angle 1 &= \frac{1}{2} \left( \widearc{NW} - \widearc{PE} \right) \\
&= \frac{1}{2} \left( \frac{\pi}{2} - \widearc{PE} \right) \\
&= \frac{1}{2} \left( \widearc{NP} \right) \\
&= \angle 2
\end{align}
Next, because $\overline{ON} \parallel \overline{RP}$, the alternate interior angles are equal, so $\angle 2 = \angle 3$. Then by the defintion of tangent, $\tan \angle 3 = b$. Then, by the previous equalities, $\tan \angle 1 = b$. Further, by the definition of tangent, $\tan \angle 1 = \frac1a$. Then we have $\frac1a = b$. Thus, $ab = a\frac1a = 1$.
\end{proof}

\vskip.1in
\hrule
\vskip.1in

\noindent{\bf 2. } Find the stereographic projection of $2+i$ onto the Riemann sphere.  Show work.

\vskip.1in

\noindent{\sc Solution. } We begin by parametrizing the line segment $\mathcal L$ in $\mathbb R^3$ that joins the north pole $(0,0,1)$ of the Riemann sphere $\mathcal S: x^2+y^2+z^2=1$ with the given point $(2,1,0)$ in the complex plane: 
\begin{align}
x &= 0 +  \underbar{2} t\\    
y &= 0 +  \underbar{1} t\\     
z &= 1 +  \underbar{-1} t
\end{align}
for $t \in [0,1]$.  Next, to find the point of intersection $\mathcal L \cap \mathcal S$, we plug these values into the equation for the sphere:

\begin{align*}
(2t)^2 + t^2 + (1-t)^2 &= 1\\
4t^2 + t^2 + 1 - 2t + t^2 &= 1\\
6t^2 - 2t + 1 &= 1\\
6t^2 - 2t &= 0\\
t &= 0, \frac13
\end{align*}

When $t = 0$, we have the intersect at the north pole, so we are concerned only with $t = \frac13$. When $t = \frac13$, we have:

\begin{align*}
x &= \frac23\\    
y &= \frac13\\     
z &= \frac23
\end{align*}

Thus, the point of intersection is $\left(\frac23, \frac13, \frac23\right)$.
\vskip.1in
\hrule

\end{document}