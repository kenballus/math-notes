\documentclass[11pt]{article}       % The percent symbol in your code starts a comment.  The comment ends at the next linebreak.
\usepackage[english]{babel}         % Packages add functionality and style conventions to your documents. Don't edit this section!
\usepackage{fullpage}               % Eliminates wasted space
\usepackage[utf8]{inputenc}         % Necessary for character encoding
\usepackage{amsmath, amssymb,amsthm}% Required math packages
\usepackage{graphicx}               % For handling graphics
\usepackage[colorinlistoftodos]{todonotes}  % For the fancy "todo" stuff
\usepackage{hyperref}               % For clickable links in the final PDF
\usepackage{tikz}
\theoremstyle{definition}
\newtheorem{theorem}{Theorem}
\newtheorem{lemma}[theorem]{Lemma}
\newtheorem{prop}[theorem]{Proposition}
\newtheorem{claim}[theorem]{Claim}

\title{Complex Analysis -- Homework \#1}

\author{ Josef Komissar and Ben Kallus }

\date{ Due Friday, February 5 }

\begin{document}
\pagecolor{black}
\color{white}

\maketitle

\noindent{\bf 1. }  Use the definition for the {\bf \emph{sum}} $z_1 + z_2$ and {\bf \emph{product}} $z_1z_2$ of the complex numbers $z_1=(x_1, y_1)$ and $z_2=(x_2, y_2)$ given in equations (3) and (4) on page 2 to compute each of the following:
\vskip.1in
\noindent{\bf a.} $(1,2)\cdot(-4,1)= (-4-2, -8+1) = (-6,-8)$
\vskip.1in
\noindent{\bf b.} $(0,1)\cdot(2,7)+(1,0)=(-7+1, 2) = (-6,2)$
\vskip.1in
\noindent{\bf c.} $(1,1)^2 -(2,0)\cdot(1,1) + (2,0)= (1 - 1 + 2 -2, 2-2) = (0,0)$

\bigskip
\noindent{\bf 2. }  The change of variable $x \mapsto t - \frac{b}{3}$ transforms the general cubic equation $x^3+bx^2+cx+d=0$ into a new cubic equation without a squared term.  The process, dubbed \emph{depressing the cubic equation,} is attributable to Nicol\`{o} Fontana Tartaglia (1500-1557). 
\vskip.1in
\noindent{\bf a. }  Exhibit the general depressed cubic that results from Tartaglia's transformation.
\begin{align*}
    \left(t-\frac b3\right)^3 + b\left(t-\frac b3\right) + c\left(t-\frac b3\right) + d
    &= \left(t^3 - bt^2 + \frac{b^2t}3 - \frac{b^3}{27}\right) + b\left(t^2 - \frac{2bt}3 + \frac{b^2}9\right) + c\left(t-\frac b3\right) + d \\
    &= t^3 - bt^2 + \frac{b^2t}3 - \frac{b^3}{27} + bt^2 - \frac{2b^2t}3 + \frac{b^3}9 + ct - \frac{bc}3 + d \\
    &= t^3 - bt^2 + bt^2 + ct + \frac{b^2t}3 - \frac{2b^2t}3 - \frac{b^3}{27} + \frac{b^3}9 - \frac{bc}3 + d \\
    &= t^3 - t^2\left(b - b\right) + t\left(c + \frac{b^2}3 - \frac{2b^2}3\right) - \frac{b^3}{27} + \frac{b^3}9 - \frac{bc}3 + d \\
    &= t^3 + t\left(c - \frac{b^2}3\right) + \frac{2b^3}{27} - \frac{bc}3 + d
\end{align*}

\medskip
\noindent{\bf b. }  Calculate the depressed cubic equation for $x^3-6x^2+13x-10=0$.
\begin{align*}
    t^3 + t\left(13 - \frac{(-6)^2}3\right) + \frac{2(-6)^3}{27} - \frac{(-6)(13)}3 - 10
    &= t^3 + t\left(13 - 12\right) + \frac{2(-216)}{27} + 26 - 10 \\
    &= t^3 + t
\end{align*}

\newpage
\noindent{\bf c. }  Use the result of part  {\bf b}  to solve the cubic equation $x^3-6x^2+13x-10=0$ for $x$.
\begin{align*}
    t &= \sqrt[3]{\frac 02 + \sqrt{\left(\frac 02\right)^2 - \left(\frac{-1}3\right)^3}} + \sqrt[3]{\frac 02 - \sqrt{\left(\frac 02\right)^2 - \left(\frac{-1}3\right)^3}} \\
    t &= \sqrt[3]{\sqrt{\frac{1}{27}}} + \sqrt[3]{-\sqrt{\frac{1}{27}}} \\
    t &= 0
\end{align*} Thus, $$x = 0 - \frac{-6}{3} = 2.$$


\end{document}