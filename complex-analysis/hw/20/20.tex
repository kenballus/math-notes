\documentclass[11pt]{article}		% The percent symbol in your code starts a comment.  The comment ends at the next linebreak.
\usepackage[english]{babel} 		% Packages add functionality and style conventions to your documents. Don't edit this section!
\usepackage{fullpage}				% Eliminates wasted space
\usepackage[utf8]{inputenc}			% Necessary for character encoding
\usepackage{amsmath, amssymb,amsthm}% Required math packages
\usepackage{graphicx}				% For handling graphics
\usepackage[colorinlistoftodos]{todonotes}	% For the fancy "todo" stuff
\usepackage{hyperref}				% For clickable links in the final PDF
\usepackage{tikz}
\theoremstyle{definition}
\newtheorem{theorem}{Theorem}
\newtheorem{lemma}[theorem]{Lemma}
\newtheorem{prop}[theorem]{Proposition}
\newtheorem{claim}[theorem]{Claim}

\title{Complex Analysis -- Homework \#20}

\author{ Ben, Connor, and Josef Kamisman }

\date{Due Friday, March 26}

\begin{document}

\maketitle

\noindent{\bf 1. a.} Apply the $\varepsilon$-$\delta$ definition of differentiability to prove that the function $f : \mathbb R^2 \longrightarrow \mathbb R$ given by $f(x,y)=x^2y$ satisfies
$f^\prime(1,2)=A=
\begin{bmatrix}
4 & 1
\end{bmatrix}
$.

\vskip.15in
\noindent
\dotfill
\vskip.05in

\noindent{\bf b. }  Is the function $f : \mathbb C \longrightarrow \mathbb C$ given by $f(z)=(\text{Re}(z))^2 \cdot \text{Im}(z)$ differentiable at the point $z_0=1+2i$?


\vskip.15in
\hrule

\end{document}
