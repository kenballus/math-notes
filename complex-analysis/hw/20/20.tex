\documentclass[11pt]{article}		% The percent symbol in your code starts a comment.  The comment ends at the next linebreak.
\usepackage[english]{babel} 		% Packages add functionality and style conventions to your documents. Don't edit this section!
\usepackage{fullpage}				% Eliminates wasted space
\usepackage[utf8]{inputenc}			% Necessary for character encoding
\usepackage{amsmath, amssymb,amsthm}% Required math packages
\usepackage{graphicx}				% For handling graphics
\usepackage[colorinlistoftodos]{todonotes}	% For the fancy "todo" stuff
\usepackage{hyperref}				% For clickable links in the final PDF
\usepackage{tikz}
\theoremstyle{definition}
\newtheorem{theorem}{Theorem}
\newtheorem{lemma}[theorem]{Lemma}
\newtheorem{prop}[theorem]{Proposition}
\newtheorem{claim}[theorem]{Claim}

\title{Complex Analysis -- Homework \#20}

\author{Commissar, Callous, Field-Man}

\date{March 2021}

\begin{document}
\pagecolor{black}
\color{white}
\maketitle

\noindent{\bf 1. a.} Apply the $\varepsilon$-$\delta$ definition of differentiability to prove that the function $f : \mathbb R^2 \longrightarrow \mathbb R$ given by $f(x,y)=x^2y$ satisfies
$f^\prime(1,2)=A=
\begin{bmatrix}
4 & 1
\end{bmatrix}
$.

\begin{proof}
Note that when $||\vec x - a||_2 < 1$, it follows that $||\vec x - a||_2^2 < ||\vec x - a||_2$.

Let $\varepsilon > 0$ be given and let $\delta = \min(1, \frac\varepsilon5)$. Note that $|x - 1| \leq ||\vec x - (1, 2)||_2$ and $|y - 2| \leq ||\vec x - (1, 2)||_2$. Then whenever $\vec x\in \mathbb R^2$ satisfies $||\vec x - (1, 2)||_2 < \delta$, it follows that

\begin{align*}
    |f(\vec x) - p(\vec x)| &= |f(\vec x) - f(1, 2) - f'(1, 2)(\vec x - (1, 2))|\\
                            &= \left|x^2y - 1^2\cdot 2 -
                                                         \begin{bmatrix}
                                                             4 & 1
                                                         \end{bmatrix}
                                                         \begin{bmatrix}
                                                             x - 1 \\ y - 2
                                                         \end{bmatrix} \right| \\
    &= |x^2y - 2 - (4x-4 + y - 2)|\\
    &= |x^2y + 4 - 4x - y|\\
    &= |y(x-1)(x+1) - 4(x-1)|\\
    &= |x-1||y(x+1)-4|\\
    &= |x-1||(y-2+2)(x-1+2) - 4|\\
    &= |x-1||(y-2)(x-1) + 2(y-2) + 2(x-1)|\\
    &\leq |x-1|(|y-2||x-1| + 2|y-2| + 2|x-1|)\\
    &\leq ||\vec x - (1, 2)||_2(||\vec x - (1, 2)||_2^2 + 4||\vec x - (1, 2)||_2)\\
    &\leq 5||\vec x - (1, 2)||_2||\vec x - (1, 2)||_2\\
	&\leq 5\frac\varepsilon5 ||\vec x - (1,2)||_2\\
	&= \varepsilon||\vec x - (1,2)||_2 \\
	&\leq \varepsilon \cdot 1 \\
	&= \varepsilon.
\end{align*}

	Thus, 

\end{proof}

\vskip.15in
\noindent
\dotfill
\vskip.05in

\noindent{\bf b. }  Is the function $f : \mathbb C \longrightarrow \mathbb C$ given by $f(z)=(\text{Re}(z))^2 \cdot \text{Im}(z)$ differentiable at the point $z_0=1+2i$?


\vskip.15in
\hrule

\end{document}
