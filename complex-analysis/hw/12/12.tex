\documentclass[11pt]{article}       % The percent symbol in your code starts a comment.  The comment ends at the next linebreak.
\usepackage[english]{babel}         % Packages add functionality and style conventions to your documents. Don't edit this section!
\usepackage{fullpage}               % Eliminates wasted space
\usepackage[utf8]{inputenc}         % Necessary for character encoding
\usepackage{amsmath, amssymb,amsthm}% Required math packages
\usepackage{graphicx}               % For handling graphics
\usepackage[colorinlistoftodos]{todonotes}  % For the fancy "todo" stuff
\usepackage{hyperref}               % For clickable links in the final PDF
\usepackage{tikz}
\theoremstyle{definition}
\newtheorem{theorem}{Theorem}
\newtheorem{lemma}[theorem]{Lemma}
\newtheorem{prop}[theorem]{Proposition}
\newtheorem{claim}[theorem]{Claim}

\title{Complex Analysis -- Homework \#12}

\author{ Kallus, Koomissar, Felderman }

\date{ Due Monday, March 8 }

\begin{document}

\maketitle


\noindent{\bf 1. }  Prove that if $A$ is an $n \times n$ Hermitian matrix, then $\mathbf z^H A \mathbf z$ is real for all
$\mathbf z
=
\begin{bmatrix}
z_1 \\
\vdots \\
z_n
\end{bmatrix}
\in \mathbb C^n$
\begin{proof}
Let $A$ be an $n \times n$ Hermitian matrix, and let $\mathbf z \in \mathbb C^n$.  It suffices to show that $\overline{\mathbf z^H A \mathbf z} = \mathbf z^H A \mathbf z$.  To this end,
\begin{align*}
    \overline{\mathbf z^HA\mathbf z} &= \overline{\mathbf z^H} \overline A \overline{\mathbf z} \\
                                     &= \mathbf z^T \overline{A^H} (\mathbf z^H)^T \\
                                     &= \mathbf z^T A^T (\mathbf z^H)^T \\
                                     &= (\mathbf z^HA\mathbf z)^T.
\end{align*}
Since $\mathbf z^HA\mathbf z$ is a $1 \times 1$ matrix, $(\mathbf z^HA\mathbf z)^T = \mathbf z^HA\mathbf z$.
Thus, $\overline{\mathbf z^H A \mathbf z} = \mathbf z^H A \mathbf z$.
Thuse $\mathbf z^H A \mathbf z$ is real.
\end{proof}

\vskip.1in
\hrule
\vskip.1in

\noindent{\bf 2. }   Let $\mathcal L$ denote the line with equation $x+3y=10$.
\vskip.1in
\noindent {\bf a. }   The point $a \in \mathbb C$ for which $\mathcal L = \{ z \in \mathbb C : |z|=|z-a|\}$ is $a=2+6i$.

\vskip.05in
\noindent \dotfill

\noindent {\bf b. }   Find the image of the line $\mathcal L$ under the inversion mapping $w=f(z)=1/z$.  Show steps.
\begin{align*}
w \in f(\mathcal L) &\iff \frac{1}{w} \in \mathcal L \\
                    &\iff \left|\frac1w\right| = \left|\frac1w - (2+6i)\right| \\
                    &\iff \frac1{|w|} = \left|\frac1w - (2+6i)\right| \\
                    &\iff 1 = |1 - w(2+6i)|
\end{align*}
Thus, the image of $\mathcal L$ under the inversion mapping is the circle of radius 1 centered at 


\vskip.15in
\hrule

\end{document}