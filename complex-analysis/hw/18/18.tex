\documentclass[11pt]{article}		% The percent symbol in your code starts a comment.  The comment ends at the next linebreak.
\usepackage[english]{babel} 		% Packages add functionality and style conventions to your documents. Don't edit this section!
\usepackage{fullpage}				% Eliminates wasted space
\usepackage[utf8]{inputenc}			% Necessary for character encoding
\usepackage{amsmath, amssymb,amsthm}% Required math packages
\usepackage{graphicx}				% For handling graphics
\usepackage[colorinlistoftodos]{todonotes}	% For the fancy "todo" stuff
\usepackage{hyperref}				% For clickable links in the final PDF
\usepackage{tikz}
\theoremstyle{definition}
\newtheorem{theorem}{Theorem}
\newtheorem{lemma}[theorem]{Lemma}
\newtheorem{prop}[theorem]{Proposition}
\newtheorem{claim}[theorem]{Claim}

\title{Complex Analysis -- Homework \#18}

\author{Kosef Jomissar, Fennor Coldman, Kan Bellus}

\date{Due Monday, March 22}

\begin{document}
\color{white}
\pagecolor{black}
\maketitle

\noindent{\bf 1. Exercise 18.10(a) }
\vskip.15in
\hrule
\vskip.15in
\noindent{\bf 2. Exercise 18.10(c) }

\vskip.15in
\hrule
\vskip.15in

\noindent{\bf 3. } Can the function
$
f(z) =  \dfrac{\text{Re}(z)}{z}
$
be defined at $z=0$ so that it is continuous there?  (Justify accordingly.)
\begin{proof}
    Observe that for all $z \in \mathbb C$ satisfying $\text{Im}(z) = 0$, $$f(z) = \frac{\text{Re}(z)}{z} = \frac zz = 1.$$
    Next, observe that for all $z \in \mathbb C$ satisfying $\text{Re}(z) = 0$, $$f(z) = \frac{\text{Re}(z)}{z} = \frac0z = 0.$$

    Let $\epsilon > 0$ be given.
    Note that $\frac{\epsilon}2$ is complex number with imaginary component 0.
    Thus, $f(\frac{\epsilon}2) = 0$.
\end{proof}

\vskip.15in
\hrule

\end{document}
