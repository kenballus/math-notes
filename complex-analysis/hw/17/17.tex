\documentclass[11pt]{article}       % The percent symbol in your code starts a comment.  The comment ends at the next linebreak.
\usepackage[english]{babel}         % Packages add functionality and style conventions to your documents. Don't edit this section!
\usepackage{fullpage}               % Eliminates wasted space
\usepackage[utf8]{inputenc}         % Necessary for character encoding
\usepackage{amsmath, amssymb,amsthm}% Required math packages
\usepackage{graphicx}               % For handling graphics
\usepackage[colorinlistoftodos]{todonotes}  % For the fancy "todo" stuff
\usepackage{hyperref}               % For clickable links in the final PDF
\usepackage{tikz}
\theoremstyle{definition}
\newtheorem{theorem}{Theorem}
\newtheorem{lemma}[theorem]{Lemma}
\newtheorem{prop}[theorem]{Proposition}
\newtheorem{claim}[theorem]{Claim}

\title{Complex Analysis -- Homework \#17}

\author{Connor Kallus, Josef Feldman, and Ben Komissar}

\date{Due Friday, March 19, 2021}

\begin{document}
\color{white}
\pagecolor{black}
\maketitle

\noindent {\bf 1. }  Use the $\varepsilon$-$\delta$ definition of limit to prove $\displaystyle \lim_{z \to i} \, \dfrac{z-i}{iz-1}= 0$.
\begin{proof}
    Let $\epsilon > 0$ be given.
    Note that for all $z$ satisfying $|z-i|<1$,
    \begin{align*}
        |z+i| &= |z+i+i-i| \\
              &= |z-i + 2i| \\
              &\geq \left||z-i| - |2i|\right| \\
              &= ||z-i|-2| \\
              &> |1-2| \\
              &= 1.
    \end{align*}
    Let $\delta = \min(\epsilon, 1)$.
    Thus, for all $z$ satisfying $|z-i| < \delta$,
    \begin{align*}
        \left|\frac{z-i}{iz-1} - 0\right| &= \left|\frac{z-i}{iz-1}\right| \\
                                          &= |z-i|\left| \frac{1}{iz-1} \right| \\
                                          &= |z-i|\left| \frac{1}{i(z+i)} \right| \\
                                          &= |z-i|\left| \frac{1}{z+i} \right| \\
                                          &< |z-i| \\
                                          &< \epsilon.
    \end{align*}
    Thus, $$\lim_{z \to i} \frac{z-i}{iz-1} = 0.$$
\end{proof}

\newpage
\noindent {\bf 2. Exercise 18.1(b) }
\begin{proof}
    Let $\epsilon > 0$ be given.
    Let $z_0 \in \mathbb C$.
    Then, for all $z$ satisfying $|z-z_0| < \epsilon$,
    \begin{align*}
        |\overline z - \overline {z_0}| &= |\overline{z-z_0}| \\
                                        &= |z-z_0| \\
                                        &< \epsilon.
    \end{align*}
    Thus, $$\lim_{z \to z_0} \overline z = \overline {z_0}.$$
\end{proof}

\end{document}