\documentclass[11pt]{article}		% The percent symbol in your code starts a comment.  The comment ends at the next linebreak.
\usepackage[english]{babel} 		% Packages add functionality and style conventions to your documents. Don't edit this section!
\usepackage{fullpage}				% Eliminates wasted space
\usepackage[utf8]{inputenc}			% Necessary for character encoding
\usepackage{amsmath, amssymb,amsthm}% Required math packages
\usepackage{graphicx}				% For handling graphics
\usepackage[colorinlistoftodos]{todonotes}	% For the fancy "todo" stuff
\usepackage{hyperref}				% For clickable links in the final PDF
\usepackage{tikz}
\theoremstyle{definition}
\newtheorem{theorem}{Theorem}
\newtheorem{lemma}[theorem]{Lemma}
\newtheorem{prop}[theorem]{Proposition}
\newtheorem{claim}[theorem]{Claim}

\title{Complex Analysis -- Homework \#24}

\author{``Dirty" Josef Komissar, Benjamin Kallus, Connor ``Drippy" Feldman}

\date{Due April 9, 2021}

\begin{document}

\maketitle

\medskip\noindent\textbf{1.}

\medskip\textbf{a.}
\begin{align*}
    \exp\left(-\frac{\pi i}3\right) &= \exp\left(i \cdot -\frac{\pi}3\right)
                                    &=  e^0\left(\cos\left(-\frac{\pi}3\right) + i\sin\left(-\frac{\pi}3\right)\right)
                                    &= \cos\left(\frac{5\pi}3\right) + i\sin\left(\frac{5\pi}3\right)
\end{align*}
\medskip\textbf{b.}
\begin{align*}
    \exp\left(\frac12 - \frac{\pi i}4\right) &= e^{\frac12}\left( \cos\left(-\frac{\pi}4\right) + i\sin\left( -\frac{\pi}4 \right) \right) \\
                                             &= \sqrt{e}\left( \cos\left( \frac{7\pi}4 \right) + i\sin\left( \frac{7\pi}4 \right) \right)
\end{align*}
\medskip\textbf{c.}
\begin{align*}
    \log(4i) &= \{z \mid \exp(z) = 4i\} \\
             &= \{x+iy \mid e^x( \cos(y) + i\sin(y) ) = 4i\} \\
             &= \{x+iy \mid \cos(y) = 0 \text{ and } e^x( i\sin(y) ) = 4i\} \\
             &= \left\{x + iy \mid y \in \left\{\frac{(2n+1)\pi}2 \mid n \in \mathbb Z^{\geq0}\right\} \text{ and } e^x( \sin(y) ) = 4 \right\}
             &= \left\{x + iy \mid y \in \left\{\frac{(2n+1)\pi}2 \mid n \in \mathbb Z^{\geq0}\right\} \text{ and } e^x( \sin(y) ) = 4 \right\}
\end{align*}

\end{document}
