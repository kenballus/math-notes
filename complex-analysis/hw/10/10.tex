\documentclass[11pt]{article}       % The percent symbol in your code starts a comment.  The comment ends at the next linebreak.
\usepackage[english]{babel}         % Packages add functionality and style conventions to your documents. Don't edit this section!
\usepackage{fullpage}               % Eliminates wasted space
\usepackage[utf8]{inputenc}         % Necessary for character encoding
\usepackage{amsmath, amssymb,amsthm}% Required math packages
\usepackage{graphicx}               % For handling graphics
\usepackage[colorinlistoftodos]{todonotes}  % For the fancy "todo" stuff
\usepackage{hyperref}               % For clickable links in the final PDF
\usepackage{tikz}
\theoremstyle{definition}
\newtheorem{theorem}{Theorem}
\newtheorem{lemma}[theorem]{Lemma}
\newtheorem{prop}[theorem]{Proposition}
\newtheorem{claim}[theorem]{Claim}

\title{Complex Analysis -- Homework \#10}

\author{Komissar, Feldman, Kallus}

\date{ -- insert date here -- }

\begin{document}

\maketitle
\color{white}
\pagecolor{black}

\noindent{\bf 1. }  Let $\mathbf y =\left[ \begin{smallmatrix} 1 \\ 0 \\ -2 \\ 0 \end{smallmatrix} \right]$.  Find the complex polynomial $p(z) = c_0 + c_1z + c_2z^2 +c_3z^3$  for which
\[
p(w^j)=p( {\rm e}^{\pi i j/2} )= p( i^j )=y_j \quad \text{ for } j = 0, 1, 2, 3.
\]

\begin{align*}
    Fc &= y \\
    \begin{bmatrix}
        1 & 1 & 1 & 1 \\
        1 & w & w^2 & w^3 \\
        1 & w^2 & w^4 & w^6 \\
        1 & w^3 & w^6 & w^9
    \end{bmatrix}
    \begin{bmatrix}
        c_0 \\ c_1 \\ c_2 \\ c_3
    \end{bmatrix}
    &=
    \begin{bmatrix}
        1 \\ 0 \\ -2 \\ 0
    \end{bmatrix} \\
    \implies \begin{bmatrix}
        1 & 1 & 1 & 1 \\
        1 & (e^{i\frac{\pi}2})^1 & (e^{i\frac{\pi}2})^2 & (e^{i\frac{\pi}2})^3 \\
        1 & (e^{i\frac{\pi}2})^2 & (e^{i\frac{\pi}2})^4 & (e^{i\frac{\pi}2})^6 \\
        1 & (e^{i\frac{\pi}2})^3 & (e^{i\frac{\pi}2})^6 & (e^{i\frac{\pi}2})^9
    \end{bmatrix}
    \begin{bmatrix}
        c_0 \\ c_1 \\ c_2 \\ c_3
    \end{bmatrix}
    &=
    \begin{bmatrix}
        1 \\ 0 \\ -2 \\ 0
    \end{bmatrix} \\
    \implies \begin{bmatrix}
        1 & 1 & 1 & 1 \\
        1 & i & -1 & -i \\
        1 & -1 & 1 & -1 \\
        1 & -i & -1 & i
    \end{bmatrix}
    \begin{bmatrix}
        c_0 \\ c_1 \\ c_2 \\ c_3
    \end{bmatrix}
    &=
    \begin{bmatrix}
        1 \\ 0 \\ -2 \\ 0
    \end{bmatrix} \\
    \begin{bmatrix}
        c_0 \\ c_1 \\ c_2 \\ c_3
    \end{bmatrix}
    &= \begin{bmatrix}
        1 & 1 & 1 & 1 \\
        1 & i & -1 & -i \\
        1 & -1 & 1 & -1 \\
        1 & -i & -1 & i
    \end{bmatrix}^{-1}
    \begin{bmatrix}
        1 \\ 0 \\ -2 \\ 0
    \end{bmatrix} \\
    \begin{bmatrix}
        c_0 \\ c_1 \\ c_2 \\ c_3
    \end{bmatrix}
    &= \frac14\begin{bmatrix}
        1 & 1 & 1 & 1 \\
        1 & i & -1 & -i \\
        1 & -1 & 1 & -1 \\
        1 & -i & -1 & i
    \end{bmatrix}^{H}
    \begin{bmatrix}
        1 \\ 0 \\ -2 \\ 0
    \end{bmatrix} \\
    \begin{bmatrix}
        c_0 \\ c_1 \\ c_2 \\ c_3
    \end{bmatrix}
    &= \frac14\begin{bmatrix}
        1 & 1 & 1 & 1 \\
        1 & -i & -1 & i \\
        1 & -1 & 1 & -1 \\
        1 & i & -1 & -i
    \end{bmatrix}
    \begin{bmatrix}
        1 \\ 0 \\ -2 \\ 0
    \end{bmatrix} \\
    \begin{bmatrix}
        c_0 \\ c_1 \\ c_2 \\ c_3
    \end{bmatrix}
    &= \frac14
    \begin{bmatrix}
        -1 \\ 3 \\ -1 \\ 3
    \end{bmatrix}
\end{align*}

\vskip.1in
\hrule
\vskip.1in

\noindent{\bf 2. }  Put the vector $\mathbf c = \begin{bmatrix} 1 \\ 0 \\ 1 \\ 0 \end{bmatrix}$ through the three steps of the Fast Fourier Transform  to compute $\mathbf y = F \mathbf c$.

\[
\mathbf y =
\begin{bmatrix}
* & * & * & * \\
* & * & * & * \\
* & * & * & * \\
* & * & * & *
\end{bmatrix}
\cdot
\begin{bmatrix}
* & * & * & * \\
* & * & * & * \\
* & * & * & * \\
* & * & * & *
\end{bmatrix}
\cdot
\begin{bmatrix}
* & * & * & * \\
* & * & * & * \\
* & * & * & * \\
* & * & * & *
\end{bmatrix}
\cdot
\begin{bmatrix} 1 \\ 0 \\ 1 \\ 0 \end{bmatrix}
=
\]


\vskip.15in
\hrule

\end{document}