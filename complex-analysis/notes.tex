\documentclass{article}
\usepackage[utf8]{inputenc}
\usepackage{amsmath}
\usepackage{amsfonts}
\usepackage{amssymb}
\usepackage{graphicx}
\usepackage{geometry}
\usepackage{xcolor}

\newcommand{\inv}{^{-1}}   
\newcommand{\Z}{\mathbb Z}
\newcommand{\R}{\mathbb R}
\newcommand{\Q}{\mathbb Q}
\newcommand{\C}{\mathbb C}
\newcommand{\N}{\mathbb N}
\newcommand{\Arg}{\text{Arg}}

\begin{document}
\pagecolor{black}
\color{white}

\noindent{\bf Cardano's Formula}

    \begin{align*}
        x^3 &= cx + d \\
        x &= \sqrt[3]{\frac d2 + \sqrt{\left(\frac d2\right)^2 - \left(\frac c3\right)^3}} + \sqrt[3]{\frac d2 - \sqrt{\left(\frac d2\right)^2 - \left(\frac c3\right)^3}}
    \end{align*}

\medskip
\noindent{\bf Modulus}

	The modulus of $a+bi$ is $\sqrt{a^2+b^2}$.

\medskip
\noindent{\bf Theorem}

	$\C$ cannot be ordered as a field. This is because if $i > 0$, then $i^2 > 0$, which is a contradiction. On the other hand, if $i < 0$, then $-i^2 < 0$, which is also a contradiction.

\medskip
\noindent{\bf Complex Conjugate}

    If $z = a+bi$, the complex conjugate of $z$, denoted $\overline z$, is $a-bi$.

\medskip
\noindent{\bf Modulus}

    For $z = a + ib$, $$|z| = \sqrt{a^2+b^2}.$$

\medskip
\noindent{\bf Properties of Complex Numbers}
    \begin{enumerate}
        \item $|z| \geq 0$
        \item $|z|=0 \iff z=0$
        \item $x \leq |a| \leq |z|; ~ b \leq |b| \leq |z|$
        \item $|z| = |\overline z|$
        \item $\overline{z_1 + z_2} = \overline{z_1} + \overline{z_2}$
        \item $\overline{z_1z_2} = \overline{z_1}\cdot\overline{z_2}$
        \item $\left( \overline{\frac{z_1}{z_2}} = \frac{\overline{z_1}}{\overline{z_2}} \right), z \neq 0$
        \item $\overline{\overline z} = z$
        \item $\frac{z + \overline z}2 = \text{Re}(z) = a$
        \item $\frac{z - \overline z}2 = \text{Im}(z) = b$
        \item $z \overline z = |z|^2$
        \item $z\inv = \frac{\overline z}{|z|^2}, z \neq 0$
        \item $|z_1 + z_2| \leq |z_1| + |z_2|$
        \item $|z_1z_2|=|z_1||z_2|$
    \end{enumerate}

\medskip
\noindent{\bf Modulus Metric}

    One standard metric for $\C$ is the function $$d(z_1, z_2) = |z_1 - z_2|.$$
    Properties of this metric:

    \begin{enumerate}
        \item $d(z_1, z_2) \geq 0$
        \item $d(z_1, z_2) = 0 \iff z_1 = z_2$
        \item $d(z_1, z_2) = d(z_2, z_1)$
        \item $d(z_1, z_2) \leq d(z_1, z_3) + d(z_3, z_2)$
    \end{enumerate}

\medskip
\noindent{Polar Form}

    The polar form of a complex number $z = a+bi$ is $$r\cos\theta + ir\sin\theta = r(\cos\theta + i\sin\theta),$$ with $$r = |z| = \sqrt{z^2+y^2}, \theta = \arctan{\frac yx},$$ in the appropriate quadrant, except when $z=0$ (since there is no argument of $0$).

    Note that this representation is not unique. For instance,
    \begin{align*}
        1 + i &= \sqrt{2}(\cos{\frac{\pi}4} + i\sin{\frac{\pi}4}) \\
              &= \sqrt{2}(\cos{\frac{9\pi}4} + i\sin{\frac{9\pi}4}) \\
              &= \sqrt{2}(\cos{-\frac{7\pi}4} + i\sin{-\frac{7\pi}4}) \\
              &= \hdots
    \end{align*} Since angles don't have a unique representation, neither do polar forms of complex numbers.

    Thus, $\overline z = r(cos\theta - isin\theta)$, and $ z\inv = \frac1r(\cos\theta-i\sin\theta).$

    Therefore, $|z\inv| = \frac1r = \frac1{|z|} = |z|\inv$.

    $r_1(\cos(\theta_1) + \sin(\theta_1)) \cdot r_2(\cos(\theta_2) + \sin(\theta_2)) = r_1r_2(\cos(\theta_1 + \theta_2) + \sin(\theta_1 + \theta_2))$.

\medskip
\noindent{\bf Argument}

    The argument of a complex number $0 \neq z = r\cos\theta + ir\sin\theta$, denoted $\arg z$, is $\theta$. Note that there are infinitely many equivalent options for $\theta$.

    The Argument (capitalized) of $z$ is the value of $\arg z$ satisfying $-\pi < \arg(z) \leq \pi$.

    If $\theta_1 \in \arg(z_1)$ and $\theta_2 \in \arg(z_2)$, then $\theta_1 + \theta_2 \in \arg(z_1z_2)$

\medskip
\noindent{\bf De Moivre's Formula}

    $$(\cos\theta + i\sin\theta)^n = \cos(n\theta) + i\sin(n\theta)$$

\medskip
\noindent{\bf Euler Notation}

    For $z \neq 0$, $z = r(\cos\theta + i\sin\theta) = re^{i\theta}$, $r > 0$.

\medskip
\noindent{\bf Fourier Matrix}

    A Fourier matrix is an $n \times n$ square matrix $F_n$ with entries given by $F_{jk} = e^{2\pi ijk/n} = \omega^{jk}$, where $\omega$ is the primitive $n^{\text{th}}$ root of unity.

    These can be used for polynomial interpolation.
    Make a coefficient vector $C = \begin{bmatrix} c_0 \\ c_1 \\ c_2 \\ \vdots \\ c_{n-1} \end{bmatrix}$, and solve $F_nC = Y$, where $Y$ is the vector of points to interpolate.

\medskip
\noindent{\bf Fast Fourier Transform}

    (Example with $n=4$)

    Define $$D_n = \begin{bmatrix} \omega^0 & 0 & \hdots \\ 0 & \omega^1 & 0 & \hdots \\ 0 & 0 & \omega^2 & 0 & \hdots \\ \vdots\end{bmatrix} (\text{up to } \omega^{n-1}).$$

    \begin{align*}
        F_4C &= \begin{bmatrix} 1&1&1&1 \\ 1&i&-1&-i \\ 1&-1&1&-1 \\ 1&-i&-1&i \end{bmatrix} \begin{bmatrix} c_o \\ c_1 \\c_2 \\ c_3 \end{bmatrix} \\
        &= \begin{bmatrix} I_2 & D_{2} \\ I_n & D_{2} \end{bmatrix} \begin{bmatrix} F_2 & O_2 \\ O_2 & F_2 \end{bmatrix} P_4 \begin{bmatrix} c_0 \\ c_1 \\ c_2 \\ c_3 \end{bmatrix} \\
        &= \begin{bmatrix} 1&0&1&0\\0&1&0&i\\1&0&-1&0\\0&1&0&-i \end{bmatrix} \begin{bmatrix} 1&1&0&0 \\ 1&-1&0&0 \\ 0&0&1&1 \\ 0&0&1&-1 \end{bmatrix} \begin{bmatrix} 1&0&0&0\\0&0&1&0\\0&1&0&0\\0&0&0&1 \end{bmatrix} \begin{bmatrix} c_0 \\ c_1 \\ c_2 \\ c_3 \end{bmatrix}
    \end{align*}
    This can be applied recursively if your matrices are big. 0-padding works if $n$ is not a power of 2.
    

\end{document}
