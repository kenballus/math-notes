\documentclass{article}
\usepackage[utf8]{inputenc}
\usepackage{amsmath}
\usepackage{amsfonts}
\usepackage{amssymb}
\usepackage{graphicx}
\usepackage{geometry}
\usepackage{xcolor}

\newcommand{\inv}{^{-1}}   
\newcommand{\Z}{\mathbb Z}
\newcommand{\R}{\mathbb R}
\newcommand{\Q}{\mathbb Q}
\newcommand{\C}{\mathbb C}
\newcommand{\N}{\mathbb N}
\newcommand{\Arg}{\text{Arg}}

\begin{document}
\pagecolor{black}
\color{white}

\noindent\textbf{Cardano's Formula}

    \begin{align*}
        x^3 &= cx + d \\
        x &= \sqrt[3]{\frac d2 + \sqrt{\left(\frac d2\right)^2 - \left(\frac c3\right)^3}} + \sqrt[3]{\frac d2 - \sqrt{\left(\frac d2\right)^2 - \left(\frac c3\right)^3}}
    \end{align*}

\medskip\noindent\textbf{Modulus}

	The modulus of $a+bi$ is $\sqrt{a^2+b^2}$.

\medskip\noindent\textbf{Theorem}

	$\C$ cannot be ordered as a field. This is because if $i > 0$, then $i^2 > 0$, which is a contradiction. On the other hand, if $i < 0$, then $-i^2 < 0$, which is also a contradiction.

\medskip\noindent\textbf{Complex Conjugate}

    If $z = a+bi$, the complex conjugate of $z$, denoted $\overline z$, is $a-bi$.

\medskip\noindent\textbf{Modulus}

    For $z = a + ib$, $$|z| = \sqrt{a^2+b^2}.$$

\medskip\noindent\textbf{Properties of Complex Numbers}
    \begin{enumerate}
        \item $|z| \geq 0$
        \item $|z|=0 \iff z=0$
        \item $x \leq |a| \leq |z|; ~ b \leq |b| \leq |z|$
        \item $|z| = |\overline z|$
        \item $\overline{z_1 + z_2} = \overline{z_1} + \overline{z_2}$
        \item $\overline{z_1z_2} = \overline{z_1}\cdot\overline{z_2}$
        \item $\left( \overline{\frac{z_1}{z_2}} = \frac{\overline{z_1}}{\overline{z_2}} \right), z \neq 0$
        \item $\overline{\overline z} = z$
        \item $\frac{z + \overline z}2 = \text{Re}(z) = a$
        \item $\frac{z - \overline z}2 = \text{Im}(z) = b$
        \item $z \overline z = |z|^2$
        \item $z\inv = \frac{\overline z}{|z|^2}, z \neq 0$
        \item $|z_1 + z_2| \leq |z_1| + |z_2|$
        \item $|z_1z_2|=|z_1||z_2|$
    \end{enumerate}

\medskip\noindent\textbf{Modulus Metric}

    One standard metric for $\C$ is the function $$d(z_1, z_2) = |z_1 - z_2|.$$
    Properties of this metric:

    \begin{enumerate}
        \item $d(z_1, z_2) \geq 0$
        \item $d(z_1, z_2) = 0 \iff z_1 = z_2$
        \item $d(z_1, z_2) = d(z_2, z_1)$
        \item $d(z_1, z_2) \leq d(z_1, z_3) + d(z_3, z_2)$
    \end{enumerate}

\medskip\noindent\textbf{Polar Form}

    The polar form of a complex number $z = a+bi$ is $$r\cos\theta + ir\sin\theta = r(\cos\theta + i\sin\theta),$$ with $$r = |z| = \sqrt{z^2+y^2}, \theta = \arctan{\frac yx},$$ in the appropriate quadrant, except when $z=0$ (since there is no argument of $0$).

    Note that this representation is not unique. For instance,
    \begin{align*}
        1 + i &= \sqrt{2}(\cos{\frac{\pi}4} + i\sin{\frac{\pi}4}) \\
              &= \sqrt{2}(\cos{\frac{9\pi}4} + i\sin{\frac{9\pi}4}) \\
              &= \sqrt{2}(\cos{-\frac{7\pi}4} + i\sin{-\frac{7\pi}4}) \\
              &= \hdots
    \end{align*} Since angles don't have a unique representation, neither do polar forms of complex numbers.

    Thus, $\overline z = r(cos\theta - isin\theta)$, and $ z\inv = \frac1r(\cos\theta-i\sin\theta).$

    Therefore, $|z\inv| = \frac1r = \frac1{|z|} = |z|\inv$.

    $r_1(\cos(\theta_1) + \sin(\theta_1)) \cdot r_2(\cos(\theta_2) + \sin(\theta_2)) = r_1r_2(\cos(\theta_1 + \theta_2) + \sin(\theta_1 + \theta_2))$.

\medskip\noindent\textbf{Argument}

    The argument of a complex number $0 \neq z = r\cos\theta + ir\sin\theta$, denoted $\arg z$, is $\theta$. Note that there are infinitely many equivalent options for $\theta$.

    The Argument (capitalized) of $z$ is the value of $\arg z$ satisfying $-\pi < \arg(z) \leq \pi$.

    If $\theta_1 \in \arg(z_1)$ and $\theta_2 \in \arg(z_2)$, then $\theta_1 + \theta_2 \in \arg(z_1z_2)$

\medskip\noindent\textbf{De Moivre's Formula}

    $$(\cos\theta + i\sin\theta)^n = \cos(n\theta) + i\sin(n\theta)$$

\medskip\noindent\textbf{Euler Notation}

    For $z \neq 0$, $z = r(\cos\theta + i\sin\theta) = re^{i\theta}$, $r > 0$.

\medskip\noindent\textbf{Fractional Linear Transformation}

    A fractional linear transformation, or Mobius transformation, is a function of the form $$f(z) = \frac{az+b}{cz+d},$$ in which $a,b,c,d$ are complex numbers satisfying $ad-bc \neq 0$.
    Thus, FLTs are one-to-one and onto, and their inverses are also FLTs.
    FLTs take circles and lines to circles and lines. A circle goes to a line if and only if it contains the origin.
    Thus, FLTs on circles and lines form a group under composition.
    FLTs also preserve angles between lines. This makes FLTs conformal maps.

\medskip\noindent\textbf{FLT Uniqueness Theorem}

    Given distinct $z_1, z_2, z_3 \in \overline{\mathbb C}$, and distinct $w_1, w_2, w_3 \in \overline{\mathbb C}$, ($\mathbb C$ with infinity), there exists a unique FLT $f$ such that $f(z_1) = w_1$, $f(z_2) = w_2$, and $f(z_3) = w_3$.

\medskip\noindent\textbf{Limit of a Complex Function}

	Let $f: \C \to \C$. Then, $f(z)$ has a limit $w_0$ as $z$ approaches $z_0$ if and only if for all $\epsilon \in \R^+$, there exists $\delta \in \R^+$ such that $$|f(z) - w_0| < \epsilon \implies |z - z_0| < \delta.$$

\medskip\noindent\textbf{Limit Uniqueness Theorem}

	When a limit of a function $f(z)$ exists at a point $z_0$, it is unique.

\medskip\noindent\textbf{Derivative of a Complex Function}

	Let $f$ be a function whose domain contains a neighborhood $|z-z_0| < \epsilon$ around a point $z_0$.
	The derivative of $f$ at $z_0$ is the limit $$f'(z_0) = \lim_{z\to z_0}\frac{f(z)-f(z_0)}{z-z_0}.$$

\medskip\noindent\textbf{Cauchy-Riemann Equations}

	Let $f(z) = u(x,y) + iv(x,y).$
	If $f'(z_0)$ exists, then \begin{center}$u_x = v_y$ and $u_y = -v_x.$\end{center}

\medskip\noindent\textbf{Harmonic Function}

	A function $f: \C \to \C$ is harmonic in a domain of the $xy$ plane if and only if throughout that domain it has continuous partial derivatives of the first and second order satisfying Laplace's equation: $$f_{xx}(x,y) + f_{yy}(x,y)=0.$$

\medskip\noindent\textbf{The Complex Exponential Function}

	$$\text{exp}(z) = e^z = e^{x+iy} = e^x\cos(y) + ie^x\sin(y) = e^x(\cos(y)+i\sin(y)).$$

\medskip\noindent\textbf{Properties of the Exponential Function}
\begin{enumerate}
	\item $\exp(z_1 + z_2) = \exp(z_1)\exp(z_2)$
	\item $|\exp(z)| = e^{\text{Re}(z)}$
	\item $\arg(\exp(z)) = \text{Im}(z) + 2\pi in$ for some $n \in \mathbb Z$.
	\item $\exp$ is periodic with period $2\pi i$. Thus, $\exp(z+2\pi i) = \exp(z).$
	\item $\exp$ maps $\C$ many-to-one onto $\C \setminus \{0\}$. Thus, $\log(z)$ is a multi-valued function.
    \item In general, $z^\alpha + z^\beta \neq z^{\alpha + \beta}$
\end{enumerate}

\medskip\noindent\textbf{Analytic Function}

	A function $f: \C \to \C$ is analytic in a neighborhood $D$ of a point $z_0$ if and only if it has a derivative everywhere in $D$.

\medskip\noindent\textbf{Entire Function}

	An entire function is analytic at every point in the entire plane.

\medskip\noindent\textbf{Harmonic Conjugate}

	$v$ is a harmonic conjugate of $u$ if and only if a function $f(z)=u(x,y) + iv(x,y)$ is analytic in a domain $D$.

\medskip\noindent\textbf{Complex Log}

	The $\log$ function is the inverse of $\exp$. The $\text{Log}$ function is the principle value of the $\log$ function.
	Thus, $\text{log}(z) = \ln(|z|) + i\arg(z)$.

\medskip\noindent\textbf{Trig Functions}

    $$\sin(z) = \frac{e^{iz} - e^{-iz}}{2i}$$
    $$\cos(z) = \frac{e^{iz}+e^{-iz}}{2}$$

\medskip\noindent\textbf{Properties of the Trig Functions}
\begin{enumerate}
    \item $\cos(-z) = \cos(z)$, $\sin(-z) = -\sin(z)$.
    \item $\cos(z) = \cos(z + 2\pi)$, $\sin(z) = \sin(z + 2\pi)$.
    \item The usual formulas for addition within trig functions hold.
    \item The usualy derivative formulas hold.
    \item $\cos z = \frac12(e^{y} + e^{-y})\cos x - \frac i2(e^y-e^{-y})\sin x = \cosh y \cos x - i\sinh y \sin x$
    \item $\cos z = \sin x \cosh y + i\cos x \sinh y$, $\sin z = \sin x \cosh y + i \cos x \sinh y$
    \item $\cosh^2 x - \sinh^2 x = 1$
    \item Neither $\sin$ nor $\cos$ is bounded.
\end{enumerate}

\medskip\noindent\textbf{Integrals}
    
    For a contour $z(t) = C$,
    $$\int_{C} f(z)\,dz = \int_a^b f(z(t))z'(t)\,dt$$

\medskip\noindent\textbf{Properties of Integrals}
\begin{enumerate}
    \item $\int_{-C}f(z)\,dz = -\int_Cf(z)\,dz$
    \item $|\int_a^bf(t)\,dt| \leq \int_a^b|f(t)|\,dt$
    \item $|\int_Cf(z)\,dz| \leq \int_C |f(zt))| |z'(t)|\,dt \leq M\int_a^b|z'(t)|\,dt$, where $M$ is the maximum value that $|f(z(t))|$ attains within the bounds for $t$.
\end{enumerate}

\medskip\noindent\textbf{Theorem}

    If $D$ is open and connected, and $f: D \to \mathbb C$ is continuous, then the following are equivalent:
    \begin{enumerate}
        \item $f$ has an antiderivative $F$ in $D$.
        \item $\int_C f(z)\,dz = F(z_2) - F(z_1)$ for all $C$ joining $z_1$ to $z_2$.
        \item $\int_Cf(z)\,dz = 0$ for all closed contours $C$ in $D$.
    \end{enumerate}

\medskip\noindent\textbf{Cauchy-Goursat Theorem}

    If $f$ is analytic at all points on and interior to a simple closed contour $C$, the  $\int_Cf(z)\,dz=0$.

\medskip\noindent\textbf{Cauchy's Integral Formula}

    If $f$ is analytic on and interior to simple closed contour $C$ that's positively oriented, then if $z_0$ is interior to $C$, $$f(z_0) = \frac1{2\pi i}\int_C\frac{f(z)}{z-z_0}\,dz.$$

\medskip\noindent\textbf{Cauchy's Integral Formula for Derivatives}

    If $f$ is analytic on and interior to simple closed contour $C$ that's positively oriented, then if $z_0$ is interior to $C$, $$f^{(n)}(z_0) = \frac{n!}{2\pi i}\int_C\frac{f(z)}{(z-z_0)^{n+1}}\,dz.$$

\medskip\noindent\textbf{Cauchy's Inequality}

    If $f$ is analytic on and interior to a positively oriented (counterclockwise) cycle $C$, then $|z-z_0| = r \implies |f^{(n)}(z_0)| \leq \frac{Mn!}{r^n}$, where $M = \max\{|f(z)| \mid z \in C\}$.

\medskip\noindent\textbf{Liouville's Theorem}

    A bounded entire function is constant.

\medskip\noindent\textbf{Taylor's Theorem}

    If $f$ is analytic on the disk $|z-z_0 < r|$, then $f(z) = \sum_{n=1}^\infty\frac{f^{(n)}}{n!}(z-z_0)^n$.

\medskip\noindent\textbf{Bieberbach Conjecture/ de Branges's Theorem}

    If $f(z) = z + a_2z^2 + a_3z^3 + \hdots$ is analytic, 1-1 on the unit disk, then $|a_n| \leq n$.

\end{document}
