\documentclass{article}
\usepackage[utf8]{inputenc}
\usepackage{amsmath}
\usepackage{amsfonts}
\usepackage{amssymb}
\usepackage{graphicx}

\newcommand\inv{^{-1}}   
\newcommand{\Z}{\mathbb Z}
\newcommand{\R}{\mathbb R}
\newcommand{\Q}{\mathbb Q}
\newcommand{\C}{\mathbb C}
\newcommand{\N}{\mathbb N}
\newcommand{\quat}{\mathbb H}
\newcommand{\gal}{\text{Gal }}

\begin{document}

\section{Theorems}
\noindent{\bf Theorem $1.4$}

    Let $m, n \in \Z$, not both 0.
    
    Then, there exists a unique GCD of $m$ and $n$.
    
    Further, there exist integers $a$ and $b$ such that gcd$(m, n) = am + bn$.
    
\noindent{\bf Proposition $1.8$}

    An integer $p > 1$ is prime if and only if for all nonzero $n \in \Z$, either gcd($p, n$) = 1 or $p|n$.

\noindent{\bf Proposition $1.10$}

    Let $m, n \in \Z$. Let $p$ be prime.
    
    If $p|mn$ then either $p|m$ or $p|n$.
    
\noindent{\bf Proposition $1.12$}

    Suppose there exist $m_1, m_2, \hdots, m_k \in \Z$.
    
    Let $p$ be prime.
    
    If $p|m_1m_2\hdots m_k$, then there exists $i \in {1, \hdots k}$ such that $p|m$.
    
\noindent{\bf Proposition $1.17$ -- The Fundamental Theorem of Arithmetic}

    Any integer greater than 1 can be written uniquely as a product of primes.
    
\noindent{\bf Proposition $1.18$}

    If $p$ is prime and $n \in \Z > 1$, then $\sqrt[n]{p}$ is irrational.
    
\newpage
    
\noindent{\bf Proposition $3.\pi.1$}

    For all $z \in \C$, $z \cdot \overline{z} = |z|^2$.
    
\noindent{\bf Proposition $3.\pi.2$}
    
    Let $z \in \C, r = |z|, \theta = \text{arg}(z)$.
    
\noindent{\bf Proposition $3.5$}

    let $z,w \in \C$.
    
    Then, $|z\cdot w| = |z| \cdot |w|$,
    
    and $\text{arg}(z\cdot w) =\text{arg}(z)+\text{arg}(w)$.
    
\noindent{\bf Proposition $3.1$}

    $\C$ has the following properties:
    
    Additive commutativity and associativity.
    
    Multiplicative Commutativity and associativity.
    
    Various distributivities % this probably needs to be more detailed.
    
    Additive closure
    
    Additive identity: 0+0i
    
    Additive inverses: Written as $-z$ for $z \in \C$.
    
\noindent{\bf Proposition $3.\pi.3$}

    If $z=a+bi \neq 0$,
    
    then $z\inv = \frac{1}{a^2 + b^2} \cdot (a-bi) = \frac{1}{|z|^2} \cdot \overline{z}$.
    
\noindent{\bf Proposition $3.7$ -- De Moivre's Theorem}

    For $n \in\Z^+, (\cos(\theta) + i\sin(\theta))^n = \cos(n\theta) + i\sin(n\theta)$.
    
\noindent{\bf Corollary $3.\pi.4$}

    If $z = r(\cos\theta + i\sin\theta)$,
    
    then $z^n = r^n(\cos(n\theta)+i\sin(n\theta))$ for $n \in \Z^+$.
    
\noindent{\bf Corollary $3.\pi.5$}
    
    For all $k \in\Z^+$,
    
    $\zeta_n^n = 1$, which implies $\zeta_n^k$ is an $n^{\text{th}}$ root of 1, and
    
    $\zeta_n^kn = 1$.
    
\noindent{\bf Theorem $3.11$}

    If $p(x)$ is a non-constant polynomial with coefficients in $\C$, then $p(x)$ has a root in $\C$.
    
\noindent{\bf Theorem $14.1$ -- The Fundamental Theorem of Algebra}

    If $p(x)$ is a polynomial with coefficients in $\C$ of degree $n \geq 1$,
    
    then $p(x)$ has $n$ complex roots.
    
\noindent{\bf Proposition $3.12$}

    If $p(x)$ has real coefficients, we have complex conjugate pairs of roots.
    
    If $p(x)$ is a polynomial with $z \in\C$ as a root, $\overline{z}$ is also a root of $p(x)$.

\noindent {\bf Proposition $3.\pi.7$}

    $\mathcal{U}_n = \{1, \zeta_n, \zeta_n^2, \hdots, \zeta_n^{n-1}\}$ is the complete set of $n^{\text{th}}$ roots of 1 in $\C$.

\noindent {\bf Proposition $3.\pi.8$}

    Let $\alpha \in \C$ be nonzero.
    
    Let $n \in \Z^+$.
    
    Let $r = |\alpha|$.
    
    Let $\theta = \text{arg}(\alpha)$.
    
    Then, $\sqrt[n]{r}(\cos{\frac{\theta}{n}} + i\sin{\frac{\theta}{n}})$ is an $n^{\text{th}}$ root of $\alpha$.
    
\noindent {\bf Proposition $3.8$}
    
    Let $\alpha \in \C$ be nonzero.
    
    Let $n \in \Z^+$.
    
    Let $\beta$ be one $n^{\text{th}}$ root of $\alpha$.
    
    Then the complete set of $n^{\text{th}}$ roots of $\alpha$ is $\{\beta, \beta\zeta_n,\beta\zeta_n^2, \hdots, \beta\zeta_n^{n-1}\}$.
    
\newpage
    
\noindent {\bf Proposition $4.\pi.1$}
    
    Let $i,j,n \in \Z$ such that $n \geq 2$.
    
    Then,
    
    1. Either $[i]_n = [j]_n$ or the sets are disjoint. (but not both, obviously)
    
    2. $[i]_n = [j]_n$ if and only if any of the following are true:
    
    \indent\indent a. $j\in[i]_n$.
    
    \indent\indent b. $j=i+kn$ for $k \in\Z$.
    
    \indent\indent c. $j - i = kn$ for $k \in\Z$.
    
    \indent\indent d. $n|(j - i)$.
    
    3. If $0 \leq i \leq n$, then $[i]_n \neq [j]_n$ for $j \neq i$.
    
    4. For all $a \in\Z$, there exists a unique $i \in \{0, 1, \hdots, n-1,\}$ such that $[a]_n = [i]_n$. This value is known as the least non-negative residue (LNNR) of $a$ mod $n$.
    
\noindent {\bf Corollary $4.\pi.2$}

    There exist precisely $n$ distinct equivalence classes modulo $n$.
    
\noindent {\bf Corollary $4.\pi.3$}

    For all $a, b, k, l \in \Z$,
    
    $[a + kn]_n + [b + ln]_n = [a]_n + [b]_n$, and
    
    $[a+kn]_n \cdot [b + ln]_n = [a]_n \cdot [b]_n$.
    
\noindent  {\bf Proposition $4.\pi.4$}

    The following are properties of $\Z_n$:
    
    Additive closure, commutativity, associativity, left and right distributivity, existence of identity, and eistence of inverses.
    
    Multiplicative closure, commutativity, associativity, left and right distributivity, and existence of identity.

\noindent {\bf Proposition $4.2$}

    If $n$ is a composite integer, then $\Z_n$ has zero divisors.
    
\noindent {\bf Theorem $4.\pi.5$}

    $[a]_n^{-1}$ exists if and only if gcd$(a,n) = 1$.

\noindent {\bf Corollary $4.\pi.6$}

    Suppose $[a]_n \neq 0$. 
    
    Then $[a]_n$ is a zero divisor if and only if $[a]_n^{-1}$ does not exist.
    
\noindent {\bf Proposition $4.3$}

    Every nonzero element of $\Z_n$ has a multiplicative inverse if and only if $n$ is prime.

\noindent {\bf Proposition $4.\pi.7$}

    $\quat$ has the following properties:
    
    Additive closure, associativity, commutativity, identity, and inverses.
    
    Multiplicative closure, associativity, identity, and inverses for nonzero elements.
    
    Left and right distributivity for both.
    
\newpage
    
\noindent {\bf Proposition $5.1$}
    
    For any field $F$,
    
    \indent\indent (1) The identity is unique.
    
    \indent\indent (2) For all $a\in F$, $0a = 0$.
    
    \indent\indent (3) Additive inverses are unique.
    
    \indent\indent (4) For all $a,b \in F$, $(-a)b=-(ab)=a(-b)$.
    
    \indent\indent (5) Multiplicative identity is unique.
    
    \indent\indent (6) Multiplicative inverses are unique.
    
    \indent\indent (7) $1 \neq 0$.

\noindent {\bf Proposition $5.2$}

     If $F$ is a field, and $F$ is a subset of $K$ with the same operations,
     then $F$ is a subfield of $K$ if and only if:
     
     \indent\indent (1) $|F| \geq 2$.
     
     \indent\indent (2) $F$ has additive and multiplicative closure and inverses.
     
     \indent\indent Note: Identities are not required, since they fall out of the inverses.

\noindent {\bf Proposition $5.\pi.1$}

    If $F$ is a subfield of $K$, then
    
    \indent\indent (1) $0_F = 0_K$,
    
    \indent\indent (2) $1_F = 1_K$,

    \indent\indent (3) $-a_F=-a_K$,
    
    \indent\indent (4) $a^{-1}_F=a^{-1}_K$.

\newpage

\noindent {\bf Proposition $7.1$}

    Let $F$ be a field.
    
    Then $F[x]$ is nonempty and has the following properties:
    
    \indent\indent Additive closure,

    \indent\indent Additive associativity,

    \indent\indent Additive commutativity,

    \indent\indent Additive identity (0 polynomial),

    \indent\indent Additive inverses,

    \indent\indent Multiplicative closure,

    \indent\indent Multiplicative associativity,

    \indent\indent Multiplicative commutativiy,

    \indent\indent Multiplicative identity ($p(x)=1_F$).
    
    \indent\indent Left and right distributivity of polynomial multiplication over polynomial addition.
    
\noindent {\bf Proposition $7.2$}

    Let $F$ be a field.
    
    Then $p(x)\in F[x]$ has a multiplicative inverse if and only if $p(x)$ is a nonzero constant polynomial.

\noindent {\bf Proposition $7.6$}

    If $F$ is a field, $F[x]$ is an integral domain.
    
\noindent {\bf Proposition $7.\pi.1$}

    If $R$ is a ring, so is $R[x]$.
    
\noindent {\bf Proposition $7.4$}

    If $R$ is a ring, then the following are true:
    
    \indent\indent1. $0_R$ is unique.
    
    \indent\indent2. For all $a\in R$, $a\cdot 0_R = 0_R \cdot a = 0_R$
    
    \indent\indent3. Additive inverses are unique
    
    \indent\indent4. For all $a,b\in R$, $-ab=-(ab)=a(-b)$
    
    \indent\indent5. If $R$ has $1_R$, it is unique.

\newpage

\noindent {\bf Proposition $8.2$} (not a formal proposition in class)

    Let $F$ be a field, and let $p(x), q(x) \in F[x]$.

    deg$(p(x)q(x))=$deg$(p(x)) + $deg$(q(x))$.

\noindent {\bf Theorem $8.3$} The division algorithm for $F[x]$

Let $F$ be a field.

Let $f(x), g(x)\in F[x]$.

Then, there exist $q(x), r(x) \in F[x]$ such that $f(x)=q(x)g(x)+r(x)$ with either $r(x)=0$ or $\text{deg}(r(x)) < \text{deg}(g(x))$.

\noindent {\bf Proposition $8.5$}

Let $F$ be a field.

Let $p(x)$ be a nonzero polynomial in $F[x]$.

$p(a)=0_F$ if and only if $(x-a)|p(x)$ for $a\in F$.

\noindent {\bf Theorem $8.7$}

Let $F$ be a field.

Let $f(x), g(x)$ be nonzero polynomials in $F[x]$.

There exists a unique GCD of $f(x)$ and $g(x)$.

Further, there exist $\alpha(x),\beta(x)\in F[x]$ such that 

\indent\indent$\alpha(x)f(x)+\beta(x)g(x)=\text{gcd}(f(x),g(x))$.

\noindent{\bf Proposition $8.12$}

Let $F$ be a field.

Let $p(x)$ be a non-constant polynomial in $F[x]$.

If deg$(p(x))=1$, then $p(x)$ is irreducible over $F$.

If deg$(p(x)) \in \{2,3\}$, then $p(x)$ is irreducible over $F$ if $p(x)$ has no roots in $F$.

\noindent {\bf Proposition $8.16$}

Let $F$ be a field.

Any non-constant polynomial in $F[x]$ can be written as a product of irreducible polynomials in $F$.

(Note: no uniqueness, but there is uniqueness up to scalar multiples)

\noindent {\bf Proposition $8.11$}

Let $F$ be a field.

Let $p(x)$ be a polynomial in $F[x]$.

Then, $p(x)$ is irreducible over $F$ if and only if for all nonzero polynomials $f(x)\in F[x]$, either $p(x)|f(x)$ or gcd$(p(x),f(x))=1_F$.

(Note: We can't say the GCD is either $1_F$ or $p(x)$ because $p(x)$ isn't necessarily monic.)

\noindent {\bf Proposition $8.9$}

Let $F$ be a subfield of $K$.

Let $g(x),f(x)$ be nonzero polynomials in $F[x]\subseteq K[x]$.

If gcd$(g(x),f(x))=d(x)$ over $F$, then gcd$(g(x),f(x))=d(x)$ over $K$, as well.

\newpage

\noindent {\bf Theorem $9.\pi.1$}

    Let $R$ be a ring.
    
    Let $S\subseteq R$.
    
    $S$ is a subring of $R$ if and only if all of the following hold:
    
    \indent\indent1. $S$ is not empty.
    
    \indent\indent2. $S$ is closed under addition.
    
    \indent\indent3. $S$ is closed under multiplication.
    
    \indent\indent4. $S$ has additive inverses.
    
\noindent{\bf Proposition $9.\pi.2$}

Let $S$ be a subring of $R$.

The following are true:

\indent\indent1. $0_S=0_R$.

\indent\indent3. If $R$ is a commutative ring, so is $S$.

\indent\indent5. If $R$ has no zero divisors, then $S$ has no zero divisors.

\noindent{\bf Proposition $9.2$}

Let $R$ be a ring.

Let $I\subseteq R$.

$I$ is an ideal if and only if the following are true:

\indent\indent1. $I$ is not empty.

\indent\indent2. $I$ has additive closure.

\indent\indent3. $I$ has additive inverses.

\indent\indent4. For all $r\in R$, $x\in I$, $xr,rx\in I$. (This covers multiplicative closure.)

\noindent{\bf Proposition $9.3$}

In any ring $R$, $R$ and $\{0_R\}$ are ideals of $R$.

\noindent{\bf Proposition $9.4$}

Let $R$ be a ring with unit element.

Let $I$ be an ideal of $R$.

Then $I=R$ if and only if $1_R\in I$.

\noindent{\bf Proposition $9.\pi.3$}

Let $R$ be a commutative ring with unit element.

Let $d\in R$.

Then, $\big{(}d\big{)}$ is the smallest ideal of $R$ containing $d$.

\noindent{\bf Proposition $9.6$}

Let R be a commutative ring with unit element.

Let $d\in R$.

Then, $(d)$ is the smallest ideal of $R$ containing $d$.

Note: If $F$ is a field, $F[x]$ is a commutative ring with unit element.

\noindent{\bf Theorem $9.10$}

Let $F$ be a field.

Then, $F[x]$ is a principal ideal domain.

\noindent {\bf Corollary $9.11$}

If $I$ is a nontrivial ideal of $F[x]$, then any element of $I$ of minimal degree is a generator for $I$.

\newpage

\noindent {\bf Lemma $10.7$}

Let $F$ be a subfield of $U$.

Let $r\in U$ be algebraic over $F$ of degree $n$.

Then, for all $b_0, b_1, \hdots b_m \in F$, there exist $\beta_0, \beta_1, \hdots, \beta_{n-1}\in F$ such that $b_0 + b_1r + \hdots + b_mr^m = \beta_0 + \beta_1r + \hdots + \beta_{n-1}r^{n-1}$.

\noindent {\bf Theorem $10.8$}

If $F$ is a subfield of $U$, and $r\in U$ is algebraic over $F$ of degree $n$, then $F(r) = \{a_0, a_1r + a_2r^2 + \hdots + a_{n-1}r^{n-1} | a_i\in F\}$.

\noindent {\bf Proposition $10.3$}

Let $F$ be a subfield of $U$ and let $r\in U$ be algebraic over $F$.

Let $I = \{g(x)| g(x)\in F[x], g(r)=0$\}.

$I$ is a principal ideal of $F[x]$ and its unique monic generator $m(x)$ is the minimum polynomial for $r$ over $F$.

\noindent{\bf Corollary $10.\pi.1$}
(Uses same variables as previous theorem)

If $r$ is algebraic over $F$, then the minimum polynomial for $r$ over $F$ is unique.

\noindent{\bf Corollary $10.\pi.2$}

If $r$ is algebraic over $F$ with minimum polynomial $m(x)$, then $m(x)|f(x)$ for all $f(x)\in F[x]$ with $r$ as a root. 

\newpage

\noindent {\bf Proposition $11.\pi.1$}

If $p$ is prime, then $x^{p-1} + x^{p-2} + \hdots + 1$ is irreducible over $\Q$.

\noindent {\bf Proposition $11.\pi.2$}

If $p$ is prime, the minimum polynomial for $\zeta_p$ over $\Q$ is  $x^{p-1} + x^{p-2} + \hdots + 1$.

\noindent {\bf Proposition $11.5$ -- Eisenstein's Irreducibility Criterion}

Let $g(x)=g_0+g_1x\hdots+g_nx^n\in\Z[x]$.

If there exists a prime $p$ such that all of the following hold:

\indent\indent1. $p$ does not divide $g_n$, \\
\indent\indent2. $p$ divides $g_0, g_1, \hdots, g_{n-1}$, \\
\indent\indent3. $p^2$ does not divide $g_0$,

then $g(x)$ is irreducible over $\Q$.

\newpage

\noindent{\bf Proposition $12.\pi.1$}

If $K$ is an extension field of $F$, then $K$ is a vector space over the field $F$.

\noindent{\bf Proposition $12.4$}

If $r$ is algebraic over $F$ and $n=[r:F]$ then $\{1,r,r^2,\hdots, r^{n-1}\}$ is a basis for $F(r)$ as a vector space over $F$. Thus $[F(r):F]=[r:F]$.

\noindent{\bf Proposition $12.5$}

Let $F\subseteq L \subseteq K$, fields such that $\{a_1,\hdots,a_m\}$ is a basis of the extension field $K$ over the field $L$ and that $\{b_1,\hdots,b_n\} $ is a basis of the extension field $L$ over the field $F$. Then $\{a_1b_1,\hdots,a_1b_n,\hdots,a_mb_n\}$ is a basis of the extension field $K$ over the field $F$. Additionally, $[K:F]=[K:L][L:F]$.

\noindent{\bf Proposition $12.10$}

If $K$ is a finite-dimensional extension field of a field $F$ and $k\in K$ then $k$ is algebraic over $F$ and $[k:F]$ divides $[K:F]$.

\newpage

\noindent{\bf Proposition $13.8$}

Let $f:R_1\to R_2$ be a ring homomorphism. Then,

\indent\indent1. $f(0_{R_1}) = 0_{R_2}$

\indent\indent2. for all $n\in\Z^+, \forall a\in R_1, f(a^n) = (f(a))^n$

\indent\indent3. If $R_1$ and $R_2$ are integral domains and $f$ is not the zero map, then

\indent\indent\indent(a) $f(1_{R_1}) = 1_{R_2}$

\indent\indent\indent(b) If $x\in R_1$ has a multiplicative inverse $x^{-1}$, then so does $f(x)$. In particular, $f(x\inv)=(f(x))\inv$.

\noindent{\bf Proposition $13.9$}

Suppose $K$ and $L$ are extensions fields of a field $F$ and that $f:K\to L$ is a homomorphism such that $f(a)=a$ for all $a\in F$. Let $r\in K$ be algebraic over $F$ and $m(x)\in F[x]$ such that $r$ is a root of $m(x)$. Then $f(r)$ is also a root of $m(x)$.

\noindent{\bf Proposition $13.11$}

Suppose that $F$ is a subfield of a field $K$, $u,v\in K$, algebraic over $F$ with the same minimum polynomial $m(x)$ of degree $n$. Define $f:F(u)\to F(v)$ by $f(a_0+a_1u+\hdots+a_{n-1}u^{n-1})=a_0+a_1v_1+\hdots+a_{n-1}v^{n-1}$. Then $f$ is an isomorphism that fixes $F$.

\noindent{\bf Corollary $13.\pi.1$}

If $u,v$ are algebraic over $F$ with the same minimum polynomial then $F(u)\cong F(v)$.

\newpage

\noindent{\bf Proposition $14.\pi.1$}

If $f\in\gal F(a_1, \hdots a_n)/F$ then $f$ is uniquely determined by the values of $f(a_1), \hdots, f(a_n)$.

\noindent{\bf Proposition $14.8$}

If $F$ is a subfield of $\C$ and $p(x)\in F[x]$, then the number of elements in $\gal F^{p(x)}/F$ is exactly $[F^{p(x)}:F]$.

\noindent{\bf Proposition $14.1$ -- The Fundamental Theorem of Algebra (Reprise)}

Suppose that $p(x)$ is a polynomial of degree $k$ in $\C[x]$. Then there exists a non-zero element $d\in\C$, $k$ distinct complex numbers $c_1,\hdots,c_k$ so that $p(x)=d(x-c_1)\hdots(x-c_k)$.

That is, every polynomial of positive degree can be factored into linear factors over $\C$.

That is, only degree one polynomials are irreducible over $\C$.

\noindent{\bf Proposition $14.2$}

Let $F$ be a subfield of $\C$ and suppose that $m(x)$ is an irreducible polynomial in $F[x]$ of degree $n$. Then $m(x)$ has exactly $n$ distinct complex roots.

That is, no two roots are equal.

\noindent{\bf Corollary $14.\pi.2$}

If $p(x)\in F[x]$ has two equal roots in $\C$, it is reducible over $F$.

\newpage

\noindent{\bf Proposition $15.1$}

If $K$ is an extension field of a field $F$, then the set $\gal K/F$ is a group under the operation of map composition.

\noindent{\bf Proposition $15.9$}

Let $G$ be a group. Then

\indent\indent i) The identity is unique. Additionally, if $\exists x\in G$ so that $gx=g$ or $xg=g$ for all $g\in G$, then $x=e$. 

This is stronger than just saying the identity is unique. It also shows that you only need to show the identity works on one side.

\indent\indent ii) If $\exists x,y\in G$ such that $xy=e=zx$, then $y=z=x\inv$.

This means that inverses are unique. It also means that if an element acts as an inverse on one side, it also acts as an inverse on the other side.

\indent\indent iii) for all $x\in G$, $(x\inv)\inv=x$.

\indent\indent iv) for all $x,y\in G$, $(xy)\inv = y\inv x\inv$.

\indent\indent v) if $xy=xz$ for some $x,y,z\in G$, then $y=z$.

\indent\indent vi) if $xy=zx$ for $x,y,z\in G$, then $x=z$.

\noindent{\bf Proposition $15.\pi.1$}

For all $x\in G, n\in\Z^+,(x^n)\inv=(x\inv)^n$.

\newpage

\noindent{\bf Proposition $16.1$}

Let $F$ be a subfield of $\C$. Let $p(x)\in F[x]$. Let $a_1,\hdots,a_n$ be the roots of $p(x)$. Then $f\in \text{Gal }F^{p(x)}/F$ is a permutation of $\{a_1, \hdots, a_n\}$.

Note that permutations are not homomorphisms, but these automorphisms are permutations.

\noindent{\bf Proposition $16.2$}

$S_n$ is a group with respect to map composition. If $n > 2$ then $S_n$ is not abelian.

\noindent{\bf Proposition $16.6$}

The group $S_n$ has $n!$ elements.

\noindent{\bf Proposition $16.5$}

Disjoint cycles commute.

\noindent{\bf Proposition $16.\pi.1$}

Every finite permutation can be written as the product of disjoint cycles, unique up to cycle order.

\newpage

\noindent{\bf Proposition $18.12$}

Let $f:G \to H$ be a group homomorphism. Then

\indent\indent1. $f(e_G)=e_H$

\indent\indent2. $f(x\inv)=(f(x))\inv$

\indent\indent3. $f(x^n)=(f(x))^n$ for all $n\in\Z^+$.

\noindent{\bf Proposition $17.5$}

Let $F$ be a subfield of $\C$. Let $p(x)\in F[x]$ and let $a_1,\hdots,a_n$ be the distinct roots of $p(x)$. Define $T:\text{Gal}F^{p(x)}/F\to S_n$ by $T(f)=f|_{a_1,\hdots,a_n}$.

Note: $T$ is not always an onto map.

\noindent{\bf Proposition $18.2$}

Let $S$ be a subset of a group $G$. Then $S$ is a subgroup under the same operation as $G$ if and only if the following conditions hold:

\indent\indent(i) $S\neq \emptyset$.

\indent\indent(ii) if $x,y\in S$, then $xy\in S$.

\indent\indent(iii) if $x\in S$, then $x\inv\in S$.

\noindent{\bf Proposition $18.13$}

Let $G$ and $H$ be groups and $f:G \to H$ a homomorphism. Then ker$(f)$ and $f(G)$ are subgroups of $G$ and $H$ respectively.

\noindent{\bf Corollary $18.15$}

Let $F$ be a subfield of $\C$. Let $p(x)\in F[x]$ have $n$ distinct roots. Then there exists a one-to-one homomorphism of Gal $F^{p(x)}/F$ onto a subgroup of $S_n$.

\noindent{\bf Proposition $18.16$}

Suppose that $F,L,K$ are fields so that $F\subseteq L \subseteq K$. Then

\indent\indent1. Gal $K/L$ is a subgroup of Gal $K/F$.

\indent\indent2. Gal $K/K = \{id\}$, which is a subgroup of Gal $K/F$.

\newpage

\noindent{\bf Proposition $19.\pi.1$}

$S(X)$ is the smallest subgroup of $G$ that contains $X$.

This means that any subgroup of $G$ that contains $X$ also contains $S(X)$.

\noindent{\bf Proposition $19.6$}

Every permutation in $S_n$ can be written as a product of transpositions, (not usually disjoint).

\noindent{\bf Corollary $19.7$}

$S_n$ is generated by its transpositions.

\noindent{\bf Proposition $19.8$}

$S_n$ is generated by any $n$-cycle ($a_1$ $a_2$ $\hdots$ $a_n$) and any transposition of the form ($a_i$ $a_j$) where $[j]_n=[i+1]_n$.

The elements of the transposition are the first two (or any consecutive two) of the cycle. This is necessary for general $n$.

\noindent{\bf Proposition $19.9$}

The group $S_5$ is generated by any 5-cycle and any transposition.

\noindent{\bf Proposition $19.\pi.2$}

If $p$ is a prime number, $S_p$ is generated by any $p$-cycle and any transposition.

If $n$ is not prime, then there is an $n$ cycle and a transposition that do not generate $S_n$.

% \newpage
% \noindent{\bf Unnamed Facts}

% \indent\indent $\cdot$ If $F$ is a field, $F[x]$ is a commutative ring with unit element.

\newpage
\noindent{\bf Proposition $20.4$}

Let $G$ be a group, $S$ a subgroup of $G$, and $x,y\in G$.

Then,

\indent\indent 1. Either $xS=yS$ or $xS\cap yS = \emptyset$,

\indent\indent 2. $xS=yS$ if and only if $y\inv x \in S$,

\indent\indent 3. $G$ is the disjoint union of the distinct elements of $G/S$,

\indent\indent 4. The function $f:S\to xS$ by $f(s)=xs$ is one-to-one and onto. Thus if $S$ is finite $|S|=|xS|$.

\newpage

\noindent{\bf Theorem $21.\pi.1$}

The order of a $k$-cycle in $S_n$ is its length $k$. The order of a permutation written in terms of disjoint cycles is the least common multiple of the orders (lengths) of those disjoint cycles.

\noindent{\bf Theorem $21.\pi.1$}

Let $G,H$ be groups, $x\in G$. Let $f:G\to H$ be a group homomorphism.

\indent\indent1. $o(x)=1 \iff x=e$. 

\indent\indent2. Let $o(x)=n$ and $m\in\Z^+$. Then $x^m=e$ if and only if $n$ divides $m$.

\indent\indent3. Let $p$ be prime, $x$ non-trivial in $G$. If $x^p=e$ then $o(x)=p$.

\indent\indent4. If $p$ is a prime then the $p$-cycles are the only elements of order $p$ in $S_p$.

\indent\indent5. If $G$ is finite, $o(x)$ divides $|G|$.

\indent\indent6. $o(f(x))$ divides $o(x)$.

\indent\indent7. If $f$ is one-to-one, then $o(f(x)) = o(x)$.

\noindent{\bf Theorem $21.\pi.3$}

Let $G$ be a group, $S\leq G, x\in G$.

\indent\indent1. If $o(x)=n$, then $S(x)=\{e,x,x^2,\hdots,x^{n-1}\}$ and $|S(x)|=n$.

\indent\indent {\bf Lagrange's Theorem} If $G$ is finite, then $|S|[G:S]=|G|$ and hence $|S|$ divides $|G|$.

\indent\indent Every cyclic group is abelian.

\indent\indent Every finite group of prime order is cyclic and hence abelian.

\indent\indent If $p$ is a prime and $G$ is a group of order $p$, then $G$ is isomorphic to $(\Z_p,+)$.

\newpage
\noindent{\bf Cauchy's Theorem}

If $p$ is a prime and divides the order of a group $G$ then $G$ contains an element of order $p$.

\newpage
\noindent{\bf Proposition $23.1$}

For a subgroup $H$ of a group $G$, the following statements are equivalent:

\indent\indent For all $x,y\in G$, there exists $z \in G$ such that $(xH)(yH)=zH$.

\indent\indent For all $x,y \in G$, $(xH)(yH) = (xy)H$.

\indent\indent For all $x \in G$, $xHx\inv \subseteq H$.

\indent\indent For all $x \in G$, $xHx\inv = H$.

\indent\indent For all $x \in G$, $xH = Hx$.

\noindent{\bf Proposition $23.\pi.1$}

Let $H \leq G$. There is closure under multiplication of left cosets of $H$ if and only if $H \lhd G$.

\noindent{\bf Proposition $23.4$}

Every subgroup of an abelian group is a normal subgroup.

\noindent{\bf Proposition $23.5$}

If $f:G \to H$ is a homomorphism then $\ker(f)\lhd G$.

\noindent{\bf Proposition $23.6$}

If $G$ is a group and $N \lhd G$ then $G/N$ is a group with respect to multiplication of left cosets. We call $G/N$ the quotient group of $G$ modulo $N$ or $G$ mod $N$. If $G$ is abelian, $G/N$ is abelian. If $G$ is finite then $|G/N|=[G:N]$.

\newpage
\noindent{\bf Lemma $24.4$}

Let $G$ and $H$ be groups and let $f:G\to H$ be a homomorphism. Then $f$ is one-to-one if and only if $\ker(f)=\{e_G\}$.

\noindent{\bf Proposition $24.1$}

Let $G$ be a group. Let $N \lhd G$. Define $p:G\to G/N$ by $p(x)=xN$. Then $p$ is an onto homomorphism with kernel $N$.

\noindent{\bf Theorem $24.6$: The Homomorphism Theorem for Groups}

Let $G$ and $H$ be groups and let $f:G \to H$ be an onto homomorphism with kernel $K$. Then there is an isomorphism

$\varphi: G/K\to H$

given by $\varphi(xK) = f(x)$. Thus $G/K \cong H$.

\noindent{\bf Corollary $24.1\pi.1$}

Let $G$ and $H$ be groups, $f: G \to H$ a group homomorphism. Then $G/\ker(f)\cong f(G)$.

\newpage
\noindent{\bf Proposition $25.\pi.1$}

If $p(x)$ is solvable by radicals over $F$ then $F^{p(x)}$ contains a chain of subfields $F=F_0\subseteq F_1 \subseteq \hdots \subseteq F_m = F^{p(x)}$ so that for each $0 \leq i \leq k-1$, 

\indent\indent Gal $F_m/F_{i+1} \lhd$ Gal $F_m/F_i$

\indent\indent The quotient group (Gal $F_m/F_i) / ($Gal $F_m/F_{i+1})$ is abelian.

\newpage

\noindent{\bf Proposition $26.\pi.1$}

If $G$ is an abelian group then $G$ is solvable.

\noindent{\bf Proposition $26.4$}

Let $G$ be a group and let $N \lhd G$. Then $G/N$ is abelian if and only if $\{x\inv y\inv x y~|~x,y\in G\} \subseteq N$.

\noindent{\bf Proposition $26.5$}

Let $G$ be a group and let $N \lhd G$. Then $G/N$ is abelian if and only if $G'\subseteq N$.

\noindent{\bf Proposition $26.\pi.2$}

Let $G$ be a group. Then $G^{(i)} \lhd G^{(i+1)}$.

In particular, $G' \lhd G \implies G^{(2)} \lhd G' \implies \hdots \implies G^{(k+1)} \lhd G^{(k)}$.

Note: $G^{(i)} \lhd G$.

\noindent{\bf Proposition $26.9$}

A group $G$ is solvable if and only if $G^{(k)} = \{e\}$ for some positive integer $k$.

\noindent{\bf Proposition $26.12$}

For $n \geq 5$, $S_n$ is not solvable.

\end{document}