\documentclass{article}
\usepackage[utf8]{inputenc}
\usepackage{amsmath}
\usepackage{amsfonts}
\usepackage{amssymb}
\usepackage{graphicx}

\newcommand\inv{^{-1}}   
\newcommand{\Z}{\mathbb Z}
\newcommand{\R}{\mathbb R}
\newcommand{\Q}{\mathbb Q}
\newcommand{\C}{\mathbb C}
\newcommand{\N}{\mathbb N}
\newcommand{\quat}{\mathbb H}

\begin{document}

\section{Definitions}
\noindent {\bf Minimal element}

    Let $S \subseteq \R$.
    
    We say $m \in S$ is a minimal element if and only if $m \leq s$ for all $s \in S$.

\noindent {\bf Bounded Below}

    $S$ is bounded below by $n\in\R$ if and only if $n \leq s$ for all $s\in S$.
    
    If $n$ exists, we say $S$ is bounded below.

\noindent {\bf Divides}
    
    $n$ divides $m$ ($n|m$) if and only if one of the following is true:
    
    1. $\frac{m}{n}$ is an integer.
    
    2. $\exists s \in \Z$ such that $s \cdot n = m$.
    
    3. The remainder after performing the division algorithm is 0.

\noindent {\bf Division Algorithm}
    
    Let $m, n \in \Z$.
    Assume $n > 0$.
    Then $\exists q, r \in \Z$ with $0 \leq r < n$ so that $m = qn + r$.
    
\noindent {\bf Well-Ordering Principle}
    
    Every non-empty subset of the integers that is bounded below has a minimal element.

\noindent {\bf Greatest Common Divisor}

    Let $m, n$ be integers, neither one 0.
    
    GCD of $m, n$ is a unique integer $d$ such that
    
    1. $d \leq 1$
    
    2. $d|m \land d|n$
    
    3. if $k \in \Z$ such that $(k|m \land k|n)$
    
    \medskip
    
    Some more GCD facts:

    \indent\indent If integer $m \neq 0$, gcd(m, 0) = $|m|$.
    
    \indent\indent gcd(0, 0) is undefined.

\noindent{\bf Prime}

    $n \in \Z$ is prime if and only if there does not exist $q \in \Z$ such that $1 < q < n$ and $q|n$.
    
\noindent{\bf Complex Numbers}

    $\C = \{a+bi|a,b\in\R\}$.
    
    Some things that are true:
    
    $(a+bi) + (c+di) = (a+c)+(b+d)i$.
    
    $(a+bi)(c+di)=(ac-bd)+(ad+bc)i$.
    
    $i^2 = -1$.
    
    $r(a+bi) = ra+rbi$.
    
\noindent{\bf Modulus}
    
    The Modulus of $z=a+bi$ is equal to the distance from $z$ to the origin on the complex plane.
    
    $|z|=\sqrt{a^2+b^2}$.

\noindent{\bf Argument}

    The argument of $z$ is equal to the counterclockwise angle from the positive real axis to the modulus.
    
\noindent{\bf Complex Conjugate}

    $\overline{a+bi}=a-bi$.
    
\noindent{\bf The Zetas}

    $\zeta_n = \cos(\frac{2\pi}{n}) + i\sin(\frac{2\pi}{n})$.
    
\noindent{\bf The Complex Roots of Unity}

    $\mathcal{U}_n = \{1, \zeta_n, \zeta_n^2, \hdots, \zeta_n^{n-1}\}$.

\noindent {\bf Modulus}

    For a fixed $n \in \Z^+$, $n \geq 2$, we define
    
    $[i]_n = \{i+kn|k\in\Z\}$ for $i \in \Z$.
    
\noindent {\bf The Integers Mod $n$ ($\Z_n$)}

    Let $\Z_n = \{[a]_n | a\in\Z\}$ such that
    
    \indent\indent $[a]_n + [b]_n = [a+b]_n$, and
    
    \indent\indent $[a]_n \cdot [b]_n = [a\cdot b]_n$.
    
\noindent {\bf Zero Divisors}

    If $[a]_n, [b]_n \in \Z_n$, both nonzero and $[a]_n \cdot [b]_n = [0]_n$, then we call $[a]_n$ and $[b]_n$ zero divisors.
    
\noindent {\bf Multiplicative inverses in $\Z_n$}

    If $[a]_n \cdot [b]_n = [1]_n$, then we say $[a]_n$ and $[b]_n$ are multiplicative inverses of each other. Denote this by $[a]^{-1}_n = [b]_n$. 
    
\noindent {\bf Quaternions}

    $\quat = \{a + bi + cj + dk | a,b,c \in\R\}$
    
    with component-wise addition,
    
    multiplication using left distributivity,
    
    $i^2 = j^2 = k^2 = -1$,
    
    $ij = k, jk = i, ki = j$,
    
    $ji = -k, kj = -i, ik = -j$,
    
    $-i = -1 \cdot i$, etc.
    
\noindent {\bf Fields}

    A field is a set $F$ with at least 2 elements, and 2 operations, which we call $+$ and $\cdot$, such that the following properties hold:
    
    1) Additive closure
    
    2) Additive Commutativity
    
    3) Additive Associativity
    
    4) Additive identity
    
    5) Additive Inverses
    
    6) Multiplicative Closure
    
    7) Multiplicative Commutativity
    
    8) Multiplicative Associativity
    
    9) Multiplicative Identity
    
    10) Multiplicative inverses for nonzero elements.
    
    11) Left Distributivity
    
    12) Right Distributivity
    
\noindent {\bf Subfield}

    Let $F$ and $K$ be fields with $F \subseteq K$ and the same operations.
    
    Then, $F$ is a subfield of $K$, and $K$ is an extension field of $F$.

\noindent {\bf The Polynomials Over $F$}

    Let $F$ be a set with $+$ and $\cdot$.
    
    $F[x]$ is the set of all polynomials with coefficients in $F$.
    
\noindent {\bf $F$ Adjoin $n$}

    $F(n)$ is the smallest field containing all the elements in $F$ and the element $n$.
    
    $K = F(n)$ if and only if every field containing $F$ and $n$ also contains $K$.

\noindent {\bf The Polynomials Over $F$}
    
    Let $F$ be a set with addition and multiplication.
    
    Then, $F[x]$ is the set of all polynomials with coefficients in $F$.
    
\noindent {\bf $F$ Adjoin $z$}

    $K$ is the smallest field containing $F$ and $z$ if and only if every field containing $F$ and $z$ also contains $K$.

\noindent {\bf $F$ Adjoin the Roots of $p(x)$}

    Let $F$ be a field, and $p(x)$ a polynomial over $F$.

    Then, $F^{p(x)}$ is the smallest field containing the roots of $p(x)$ and $F$.

    $F^{p(x)}$ is also known as the field generated by $F$ and the roots of $p(x)$.

    Further, if $c_1, c_2, \hdots, c_n$ are the roots of $p(x)$, $F^{p(x)}$ = $F(c_1, c_2, \hdots, c_n)$.
    
\noindent {\bf Solvable by Radicals}

    Let $U$ be an extension field of $F$.
    
    We say $p(x) \in F[x]$ is solvable by radicals over $F$ if and only if there exist $r_1, r_2, \hdots, r_m \in U$ and $k_1, k_2, \hdots, k_m \in \Z^+$ such that
    
    \indent\indent 1. $F^{p(x)} \subseteq F(r_1, r_2, \hdots, r_m)$.
    
    \indent\indent 2. $r_i^{k_i}\in F$.
    
    \indent\indent 3. For all $i \in \{2,3,\hdots,m\}$, $r_i^{k_i}\in F(r_1, r_2, \hdots, r_{i-1})$.

\noindent {\bf Some Facts About $p(x) \in F[x]$}

    Let $p(x) \in f[x]$.
    
    Then, $p(x)=a_0x^0 + a_1x^1 + \hdots + a_nx^n$ for $\{a_0, a_1, \hdots, a_n\} \subseteq F$,
    
    \noindent and $p(x)\neq0 \rightarrow a_n\neq0$

    If $a \in F$ is nonzero, $p(x)=a$ is a nonzero constant polynomial, and deg$(p(x))=0$.
    
    $p(x) = 0_F$ is the zero polynomial.
    
    Suppose $p(x)\neq0_F$. Then, deg$(p(x))=n$.
    
    The zero polynomial has degree 0. % Is this true? I thought it was no degree!!!
    
\noindent {\bf Polynomial Equality}

    Two polynomials are equal if and only if deg$(p(x)) = \text{deg}(q(x))$ and coefficients are all equal, on a per-term basis.
    
\noindent {\bf How Degrees Change with Operations}

    deg$(p(x)q(x))=$deg$(p(x)) + $deg$(q(x))$.
    
    deg$(p(x) + q(x))\leq$max$\{$deg$(p(x))$, deg$(q(x))$$\}$.

\noindent {\bf Ring}

    A ring is a nonempty set $R$ with addition, multiplication, and the following properties:
    
    \indent\indent Additive closure,

    \indent\indent Additive associativity,

    \indent\indent Additive commutativity,

    \indent\indent Additive identity,

    \indent\indent Additive inverses,

    \indent\indent Multiplicative closure,

    \indent\indent Multiplicative associativity,
    
    \indent\indent Left and right distributivity

\noindent {\bf Types of Rings}

    Commutative Ring: Multiplicative commutativity
    
    Ring with Unit Element: Multiplicative identity
    
    Integral Domain: Multiplicative commutativity, multiplicative identity, and no zero divisors
    
    Division Ring: Multiplicative identity and multiplicative inverses for nonzero elements

\noindent {\bf ``Divides" for Polynomials}

Let $f(x), g(x) \in F[x]$.

We say $g(x)$ divides $f(x)$ if and only if there exists $q(x)\inF[x]$ such that

\indent\indent$f(x)=q(x)g(x)$.

\noindent{\bf Monic}

A polynomial over a field is called monic if its highest degree term has a coefficient of 1.

\noindent {\bf GCD of Polynomials}

Let $F$ be a field.

Let $f(x),g(x),d(x)$ be nonzero polynomials in F[x].

We say $d(x)=\text{gcd}(f(x),g(x))$ if the following hold:

\indent\indent1. $d(x)$ is monic.

\indent\indent2. $d(x)|f(x)$ and $d(x)|g(x)$.

\indent\indent3. If $h(x)\in F[x]$ divides $f(x)$ and $g(x)$, then $h(x)|d(x)$.

\noindent {\bf Irreducibility}

Let $F$ be a field.
Let $p(x)$ be a non-constant polynomials in $F[x]$.

We say $p(x)$ is reducible over $F$ if there exist $d(x),e(x)\in F[x]$, each of smaller degree than $p(x)$, so that $p(x)=d(x)e(x)$.

Otherwise, we say $p(x)$ is irreducible over $F$.

\noindent {\bf Subring}

Let $R$ be a ring. If a ring $S \subseteq R$, and uses same operations as $R$, then it is a subring of $R$.

Note:

\indent\indent 1. Distribution, both associativities, additive commutativity, and multiplicative commutativity (if present) are inherited.

\indent\indent 2. Additive closure and additive inverses guarantee additive identity.

\noindent{\bf Ideal}

An ideal of a ring $R$ is a subring $I$ such that for all $x\in I$, $r\in R$, $rx, xr\in I$

\noindent{\bf Principal Ideal Generated by $d$}

Let $R$ be a commutative ring.
Let $d\in R$.

$\big{(}d\big{)}=\{rd|r\in R\}$ is called the principal ideal of $R$ generated by $d$.

\noindent{\bf Principal Ideal}

Let $R$ be a commutative ring with unit element.

$I$, an ideal of $R$, is principal if and only if there exists $d\in R$ such that $\big{(}d\big{)}=I$.

\noindent{\bf Principal Ideal Domain}

An integral domain in which every ideal is a principal ideal is called a principal ideal domain.

\newpage

\noindent{\bf Algebraic}

Let $F$ be a subfield of $U$. Let $r\in U$. We say $r$ is algebraic over $F$ if there exists $p(x)\in F[x]$ such that $p(r)=0_F$.

\noindent{\bf $[V:F]$}

If $V$ is a vector space over a field $F$ then $[V:F]$ is the dimension of $V$ over $F$. (The size of any basis for $V$ over $F$)

\noindent{\bf Ring homomorphism}

Let $R_1, R_2$ be rings. A ring homomorphism is a function $f:R_1\to R_2$ so that

\indent\indent 1. for all $a,b\in R_1, f(a+b)=f(a)+f(b)$.

\indent\indent2. for all $a,b\in R_1$, $f(a\cdot b) = f(a)\cdot f(b)$.

\noindent{\bf Isomorphism}

A homomorphism $f:R_1 \to R_2$ is called an isomorphism iff it is one-to-one and onto.

\noindent{\bf Automorphism}

An isomorphism from a ring to itself is called and automorphism.

\noindent{\bf Isomorphic}

If there exists an isomorphism from $R_1\to R_2$, then $R_1$ and $R_2$ are isomorphic. This also means they have identical properties as rings.

\noindent{\bf Galois Group}

Let $K$ be an extension field of a field $F$. The set of all automorphisms $f$ of $K$ that fix $F$ is denoted Gal $K/F$ and is called the Galois Group of $K$ over $F$.

\noindent{\bf Group}

A group is a set $G$ with a binary operation $\cdot$ satisfying the following:

\indent\indent1. For all $x,y\in G, x\cdot y\in G$. (closure)

\indent\indent2. For all $x,y,z\in G, x\cdot (y\cdot z) = (x \cdot y)\cdot z$. (associativity)

\indent\indent3. There exists an element of $G$, denoted $e$, so that for all $x\in G,e\cdot x = e = x \cdot e$. (Existence of identity)

\indent\indent4. For every $x\in G$, there exists an element $y \in G$ so that $x\cdot y = e = y\cdot x$. Here $y$ is denoted $x\inv$. (Existence of inverses for all elements)

\noindent{\bf Abelian }

A group is called abelian if its operation is commutative.

\noindent{\bf Permutation}

A permutation of a set $S$ is a function $f:S\to S$ that is one-to-one and onto.

Note that $S$ does not necessarily have structure, and even if it did, a permutation does not necessarily preserve that structure.

\noindent{$\mathbf{[n]}$}

For $n\in \Z^+$, let $[n]=\{1,\hdots,n\}$.

\noindent{$\mathbf{S_n}$}

The set of permutations of $[n]$, or any set with $n$ elements.

\noindent{\bf Cycle notation}

($a_0$ $a_1$ $\hdots$ $a_{k-1}$) means $f(a_i)=a_{i+1(\text{mod }k)}$. Any element in the set not listed in the cycle is fixed. This is called a $k$-cycle. A 2-cycle is called a transposition.

\noindent{\bf Cycle inverses}

Just reverse the order. For products of cycles, apply this recursively.

\noindent{\bf Homomorphism}

A function $f:G \to H$ is called a homomorphism (or group homomorphism) if for every $x,y\in G$, $f(xy)=f(x)f(y)\in H$. The homomorphism $f$ is called an isomorphism if it is one-to-one and onto; it is called an automorphism if it is an isomorphism and $G=H$.

\noindent{\bf Subgroup}

Let $(G,\cdot)$ be a group. A subset $S$ of $G$ that is a group under the operation $\cdot$ is called a subgroup of $G$. We denote this by $S \leq G$.

\noindent{\bf Kernel, homomorphic image of $G$ under $f$.}

Let $G$ and $H$ be groups with identities $e_G$ and $e_H$ respectively. Let $f:G \to H$ be a homomorphism. The homomorphic image of $G$ under $f$ is the set $f(G) = \{f(x) | x\in G\}$. The kernel of $f$ is the set ker$(f)=\{x\in G|f(x)=e_H\}$.

\noindent{$\mathbf{S(X)}$}

Let $X$ be a nonempty subset of a group $G$. Let $X^\pm = \{x,x\inv|x\in X\}$. Define $S(X)=\{x_1x_2\hdots x_n|x_i\in X^\pm,n$ a nonnegative integer$\}$.

This is the set of all finite products of things in $X^\pm$. Notice that the empty word is there as well.

\noindent{\bf Generates, cyclic}

If $X\subseteq G$ and $S(X)=G$ we say that $X$ generates $G$.

$G$ is called cyclic if there exists $x\in G$ so that $S(x) =G$.

\noindent{\bf Cosets}

Let $S$ be a subset of a group $G$ and $x$ an arbitrary element of $G$.

$xS = \{xs~|~s\in S\}$, a left coset of $S$ in $G$.

$Sx = \{sx~|~s\in S\}$, a right coset of $S$ in $G$.

$G/S = \{xS~|~x\in G\}$, the set of all distinct left cosets of $S$ in $G$.

$|G/S|$, the index of $S$ in $G$, denoted $[G:S]$.

\noindent{\bf Order of an element}

Let $G$ be a group, $x\in G$. The order of $x$, denoted $o(x)$ is the smallest positive integer $n$ so that $x^n=e$. If there is no such integer, then $x$ is said to have infinite order.

\noindent{\bf Order of a group}

The order of a finite group, $|G|$, is the number of elements it contains. If the group has an infinite number of elements it is said to have infinite order.

\noindent{\bf Coset products}

Let $H$ be a subgroup of $G$. Let $x,y\in G$.

Define $(xH) \cdot (yH) = \{(xh_1)(yh_2)~|~h_1,h_2\in H\}$.

This is the product of all things in $xH$ with all things in $yH$.

$xHy = \{xhy~|~h\in H\}$.

\noindent{\bf Normal subgroup}

Any subgroup $H$ of $G$ that satisfies any of the conditions of Proposition $23.1$ is a normal subgroup of $G$. This is denoted by $H \lhd G$.

\noindent{\bf Well-defined Maps}

To show a map is well-defined, show that if $a=b$, $f(a)=f(b)$.

\noindent{\bf Solvable groups}

A group $G$ is solvable if there exist subgroups $G=H_0, H_1, H_2, \hdots H_k=\{e\}$ so that

$\{e\} = H_k \leq \hdots \leq H_1 \leq H_0 = G$,

and so that for all $i$

\indent\indent1. $H_i \lhd H_{i-1}$

\indent\indent2. $H_{i-1}/H_i$ is abelian.

\noindent{\bf Commutator}

Let $G$ be a group. The commutator of $x,y\in G$ is $[x,y]=x\inv y\inv x y$.

$[x,y]\inv = [y,x]$.

\noindent{\bf Commutator Subgroup}

The subgroup generated by the commutators of a group $G$ is called the commutator subgroup of $G$ and is denoted by $G'$ or $G^{(1)}$.



\end{document}