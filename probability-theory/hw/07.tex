\documentclass[12pt]{article}
\usepackage[english]{babel}
\usepackage[utf8]{inputenc}
\usepackage{amsmath, amssymb,amsthm}
\usepackage{graphicx}
\usepackage{geometry}
\usepackage{hyperref}
\usepackage{xcolor}
\hypersetup{
    colorlinks=true,
    urlcolor=blue
}
\urlstyle{same}
\graphicspath{{./images/}}
\setlength{\topmargin}{0pt}
\setlength{\headsep}{0pt}
\textheight = 600pt

\title{Probability Theory \\ Homework 7}
\author{Ben Kallus}
\date{Due Friday, October 30}

\begin{document}
\pagecolor{black}
\color{white}
\maketitle

\noindent{\bf 1.}
\noindent{\bf a.} $$P_{A,B}(a,b) =
\begin{cases}
    \frac19 & a = 0, b = 0, \\
    \frac29 & a = 0, b = 1, \\
    \frac29 & a = 1, b = 0, \\
    \frac29 & a = 1, b = 1, \\
    \frac19 & a = 2, b = 0, \\
    \frac19 & a = 0, b = 2, \\
    0 & \text{else.}
\end{cases}$$

\medskip
\noindent{\bf b.}
\begin{align*}
    \mathbb P(B < A) &=\frac29 + \frac19 \\
                     &= \frac13.
\end{align*}

\medskip
\noindent{\bf c.}
\begin{align*}
    \mathbb E[A^B] &= \sum_{a=0}^2 \sum_{b=0}^2 P_{A,B}(a,b) \cdot a^b \\
                   &= \frac19 \cdot 0^0 + \frac29 \cdot 0^1 + \frac29 \cdot 1^0 + \frac29 \cdot 1^1 + \frac19 \cdot 2^0 + \frac19 \cdot 0^2\\
                   &= \frac19 + \frac29 + \frac29 + \frac19 \\
                   &= \frac23.
\end{align*}

\medskip
\noindent{\bf d.} \[p_A(a) = \begin{cases} \frac49 & a = 0, \\
                                           \frac49 & a = 1, \\
                                           \frac19 & a = 2, \\
                                           0 & \text{else.} \end{cases}\]
\begin{align*}
    \mathbb P(A < 2) = \frac89.
\end{align*}

\medskip
\noindent{\bf e.} $$p_{B|A}(b|0) = \begin{cases} \frac14 & b = 0 \\ \frac12 & b = 1 \\ \frac14 & b = 2 \end{cases}$$
$$\mathbb P(B \geq 1|A=0) = \frac34$$

\medskip
\noindent{\bf f.}
\begin{align*}
    \mathbb P(A = 0) \mathbb P(B = 0) &= \frac49 \cdot \frac49 \\
                                      &= \frac{16}{81}. \\
    \mathbb P(A = 0, B = 0) &= \frac19.
\end{align*} Thus, $A$ and $B$ are not independent.

\newpage
\noindent{\bf 2.}

\medskip
\noindent{\bf a.}
\begin{align*}
    \int\limits_{-\infty}^\infty \int\limits_{-\infty}^\infty ky\,dy\,dx &= 1 \\
    \int\limits_0^1\int\limits_0^1ky\,dy\,dx + \int\limits_1^2\int\limits_0^2 ky\,dy\,dx &= 1 \\
    \int\limits_0^1\int\limits_0^1y\,dy\,dx + \int\limits_1^2\int\limits_0^2 y\,dy\,dx &= \frac1k \\
    \frac12 + 2 &= \frac1k \\
    k &= \frac25
\end{align*}

\medskip
\noindent{\bf b.}
\begin{align*}
    f_X(x) &= \int\limits_{-\infty}^\infty f_{X,Y}(x,y) \\
           &= \begin{cases} \frac15 & 0 \leq x < 1, \\
                            \frac45 & 1 \leq x < 2, \\
                            0 & \text{else.} \end{cases}
\end{align*} These are the areas of the triangles under the surface $z=\frac25y$ at the input $x$.

\begin{align*}
    f_Y(y) &= \int\limits_{-\infty}^\infty f_{X,Y}(x,y) \\
           &= \begin{cases} \frac45y & 0 \leq y < 1, \\
                            \frac25y & 1 \leq y < 2, \\
                            0 & \text{else.} \end{cases}
\end{align*} These are the areas of the rectangles under the surface $z=\frac25y$ at the input $y$.

\medskip
\noindent{\bf c.}
\begin{align*}
    \mathbb P(Y > 1) &= \frac12 \cdot 1 \cdot \frac25 \cdot 1 + 1 \cdot 1 \cdot \frac25 \\
                     &= \frac35.
\end{align*} This is the volume under the surface $z=\frac25y$ with $1 < x,y \leq 2$. That's equal to the volume of a triangular prism with height $\frac25$, width and length 1, added to the volume of a rectangular prism with height $\frac25$, width and length 1.

\newpage
\noindent{\bf d.}

    Let $T$ be the amount of time it takes to retrieve the ball, starting at the ball's $X$ position. Then, $$T = 20Y + 10.$$

\medskip
\noindent{\bf e.}

    In order to be barbecued by the flamethrower, I have to try to retrieve a ball at least 1 unit away from the fence. The probability that the ball ends up at least 1 unit from the fence, as we found in part \copyright, is $\frac35$. Thus, the probability that I avoid incineration is $\frac25$.

\medskip
\noindent{\bf f.} Given $X=1.2$,
\begin{align*}
    \mathbb P(Y > 1) &= \frac{\frac25 + \frac15}{\frac45} \\
                     &= \frac34.
\end{align*} The denominator is the area of the triangle under the surface $z=\frac25y$ at $x=1.2$, and the numerator is the area of that triangle for which $y > 1$. Thus, the fraction represents the proportion of the area under that curve that corresponds to the danger zone.

The answer to part (d) remains unchanged.

The probability that I survive is $\frac14$.

\newpage
\noindent{\bf 3.}

\medskip
\noindent{\bf a.} Claim: $Q$ and $R$ are independent if and only if $f_{Q|R}(q|r) = f_Q(q)$ for all $q,r$ such that $f_R(r) > 0$.
\begin{proof}
    Suppose $Q$ and $R$ are independent. Then,
    \begin{align*}
        f_{Q|R}(q|r) &= \frac{f_{Q,R}(q,r)}{f_R(r)} \\
                     &= \frac{f_Q(q)f_R(r)}{f_R(r)} \\
                     &= f_Q(q).
    \end{align*}
    Now, suppose $$f_{Q|R}(q|r) = f_Q(q).$$ Then, $$\frac{f_{Q,R}(q,r)}{f_R(r)} = f_Q(q).$$ Thus, $$f_{Q,R}(q,r) = f_Q(q)f_R(r).$$ Therefore, $Q$ and $R$ are independent.
\end{proof}

\medskip
\noindent{\bf b.} Claim: If $Q$ and $R$ are independent, then $\mathbb E[QR] = (\mathbb EQ)(\mathbb ER)$.
\begin{proof}
    Observe that
    \begin{align*}
        \mathbb E[QR] &= \int\limits_{-\infty}^\infty \int\limits_{-\infty}^\infty f_{Q,R}(q,r) qr\,dr\,dq \\
                      &= \int\limits_{-\infty}^\infty \int\limits_{-\infty}^\infty f_Q(q)f_R(r) qr\,dr\,dq \\
                      &= \int\limits_{-\infty}^\infty f_Q(q)q \int\limits_{-\infty}^\infty f_R(r) r\,dr\,dq \\
                      &= \int\limits_{-\infty}^\infty f_Q(q)q \,dq\int\limits_{-\infty}^\infty f_R(r) r\,dr \\
                      &= (\mathbb EQ)(\mathbb ER).
    \end{align*}
\end{proof}

\newpage
\noindent{\bf 4.}

\medskip
\noindent{\bf a.}

\medskip
\noindent{\bf b.}
\begin{align*}
    \mathbb P(|U-V|<1) &= \mathbb P(V \leq U + \frac12) - \mathbb P(V \leq U - \frac12) \\
                       &= (1 - e^{-U(U+\frac12)}) - (1 - e^{-U(U-\frac12)}) \\
                       &= -e^{-U(U+\frac12)} + e^{-U(U-\frac12)}
\end{align*}

\end{document}
