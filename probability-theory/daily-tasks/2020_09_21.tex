\documentclass[12pt]{article}
\usepackage[english]{babel}
\usepackage[utf8]{inputenc}
\usepackage{amsmath, amssymb,amsthm}
\usepackage{graphicx}
\usepackage{hyperref}
\usepackage{geometry}

\graphicspath{ {./images/} }

\setlength{\topmargin}{0pt}
\setlength{\headsep}{0pt}
\textheight = 600pt

\title{Probability Theory \\ Daily Task}
\author{Ben Kallus}
\date{September 21, 2020}

\begin{document}
\maketitle

\noindent{\bf 1.1}

    Observe that $\mathbb P(X = 2~\text{and}~Y=1) = \mathbb P(X=2) = \frac14$, since $Y=1$ whenever $X=2$. Thus, since $\mathbb P(X=2)\mathbb P(Y=1) = \frac14 \cdot \frac12 = \frac18$, $X$ and $Y$ are not independent.

\bigskip
\noindent{\bf 1.2}

    There are 4 equally likely outcomes for the coin flips, and any valid choice of $Y$ and $Z$ corresponds to exactly one of them. Thus,

    \begin{tabular}{| c | c | c |}
        \hline
        Outcome & $\mathbb P(Y~\text{and}~Z)$ & $\mathbb P(X) \mathbb P(Y)$ \\
        \hline
        TT & $\frac14$ & $\frac12 \cdot \frac12$ \\
        TH & $\frac14$ & $\frac12 \cdot \frac12$ \\
        HT & $\frac14$ & $\frac12 \cdot \frac12$ \\
        HH & $\frac14$ & $\frac12 \cdot \frac12$ \\
        \hline
    \end{tabular}
    
    Thus, $Y$ and $Z$ are independent.
    
\bigskip
\noindent{\bf 1.3}
    
    Let $W = -Y$. Since $X=Y+1-Z$, 
    \begin{align*}
        \mathbb VX &= \mathbb V[Y+1-Z] \\
                   &= \mathbb V[Y-Z] \\
                   &= \mathbb V[Y+ W] \\
                   &= \mathbb VY + \mathbb VW.
    \end{align*}
    Since the variance formula only deals with squares, $\mathbb VZ = \mathbb VW$. Thus, $$\mathbb VX = \mathbb VY + \mathbb VZ.$$

\newpage
\noindent{\bf 1.4}
    \begin{align*}
        \mathbb VZ &= \mathbb E[Z^2] - \mathbb E[Z]^2 \\
                   &= (\frac12 \cdot 1^2 + \frac12 \cdot 0^2) - (\frac12 \cdot 1 + \frac12 \cdot 0)^2 \\
                   &= \frac12 - \frac14 \\
                   &= \frac14.
    \end{align*}
    
    \medskip
    $\mathbb VY = \mathbb VZ = \frac14$, since $Y$ and $Z$ describe the basically the same event.
    
    \medskip
    \begin{align*}
        \mathbb VX &= \mathbb VY + \mathbb VZ \\
                   &= \frac14 + \frac14 \\
                   &= \frac12.
    \end{align*}
    
\newpage
\noindent{\bf 2.}
\begin{align*}
    \mathbb VY &= \mathbb E[Y^2] - \mathbb E[Y]^2 \\
               &= (\frac29 \cdot 0^2 + \frac23 \cdot 1^2 + \frac19 \cdot 2^2) - (\frac{8}{9})^2 \\
               &= \frac23 + \frac49 - \frac{64}{81} \\
               &= \frac{90}{81} - \frac{64}{81} \\
               &= \frac{26}{81}.
\end{align*}
\end{document}
