\documentclass[12pt]{article}
\usepackage[english]{babel}
\usepackage[utf8]{inputenc}
\usepackage{amsmath, amssymb,amsthm}
\usepackage{graphicx}
\usepackage{hyperref}
\usepackage{geometry}

\graphicspath{{./images/}}
\setlength{\topmargin}{0pt}
\setlength{\headsep}{0pt}
\textheight = 600pt

\title{Probability Theory \\ Daily Task}
\author{Ben Kallus}
\date{October 7, 2020}

\begin{document}
\maketitle

The amount of rain in a storm is uniformly distributed from 0mm to 20mm. Let $X$ be the amount of rain in the next rainstorm.

\noindent{\bf 1.} There is a $1 - \frac5{20} = \frac34$ chance that the next rainstorm drops more than 5mm of rain.

\noindent{\bf 2.} There is a $1 - \frac{\frac{50}{13}}{20} = \frac{21}{26}$ chance that there is more than 50mm of snow in the next storm.

\noindent{\bf 3.} There is a $1-\frac{\frac{\sqrt[3]{2000000}}{13}}{20} = 1-\frac{10\sqrt[3]{2}}{26}$ chance that the next storm causes more than 100 million dollars of damage.

\end{document}
