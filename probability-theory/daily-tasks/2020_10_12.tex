\documentclass[12pt]{article}
\usepackage[english]{babel}
\usepackage[utf8]{inputenc}
\usepackage{amsmath, amssymb,amsthm}
\usepackage{graphicx}
\usepackage{hyperref}
\usepackage{geometry}

\graphicspath{{./images/}}
\setlength{\topmargin}{0pt}
\setlength{\headsep}{0pt}
\textheight = 600pt

\title{Probability Theory \\ Daily Task}
\author{Ben Kallus}
\date{October 12, 2020}

\begin{document}
\maketitle

\noindent{\bf 1.} Roughly $68\%$ of ACT takers score between $15.3$ and $26.5$.

\noindent{\bf 2.} To find the fraction of ACT takers who score between 25 and 30, compute $$F\left(\frac{30 - 20.9}{5.6}\right) - F\left(\frac{25 - 20.9}{5.6}\right) \approx .18.$$

\noindent{\bf 3.} I expect roughly $16\%$ of students score less than a 15.3, since about $32\%$ are outside of one standard deviation from the mean, and we're assuming the distribution is symmetric.

\noindent{\bf 4.} I expect the ratio of $a$ to $b$ is roughly the ratio of $\frac1{\sqrt{2\pi(5.6)^2}} e^{-\frac{(25-20.9)^2}{2(5.6^2)}}$ to $\frac1{\sqrt{2\pi(5.6)^2}} e^{-\frac{(30-20.9)^2}{2(5.6^2)}}$. This is because even though our distribution is discrete, since you can't score fractional points on the ACT, it approximates a normal distribution, so using the PDF of the normal distribution should give us a sense of the amount of students getting each score.

\end{document}