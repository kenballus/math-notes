\documentclass[12pt]{article}
\usepackage[english]{babel}
\usepackage[utf8]{inputenc}
\usepackage{amsmath, amssymb,amsthm}
\usepackage{graphicx}
\usepackage{hyperref}
\usepackage{geometry}

\setlength{\topmargin}{0pt}
\setlength{\headsep}{0pt}
\textheight = 600pt

\title{Probability Theory \\ Daily Task}
\author{Ben Kallus}
\date{September 4, 2020}

\begin{document}
\maketitle

\noindent{\bf 1.}
    Let $A$ be the event that the mint is broken.

    $\mathbb P(A) = (.7)(.01) + (.3)(.05)$

\medskip
\noindent{\bf 2.}
    3. and 5. must add to 1.

\medskip
\noindent{\bf 3.}
    Let $A$ be the event that the mint was manufactured at Pennyroyal. Let $B$ be the event that the mint is broken.

    $\mathbb P(A|B) = \frac{(.7)(.01)}{(.7)(.01) + (.3)(.05)}$

\medskip
\noindent{\bf 4.}
    Let $A$ be the event that the mint was manufactured at Pennyroyal. Let $B$ be the event that the mint is not broken.
    
    $\mathbb P(A|B) = \frac{(.7)(.99)}{(.7)(.99) + (.3)(.95)}$
    
\medskip
\noindent{\bf 5.}
    Let $A$ be the event that the mint was manufactured at Calamint. Let $B$ be the event that the mint is broken.

    $\mathbb P(A|B) = \frac{(.3)(.05)}{(.7)(.01) + (.3)(.05)}$

\end{document}
