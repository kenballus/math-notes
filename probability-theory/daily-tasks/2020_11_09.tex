\documentclass[12pt]{article}
\usepackage[english]{babel}
\usepackage[utf8]{inputenc}
\usepackage{amsmath, amssymb,amsthm}
\usepackage{graphicx}
\usepackage{hyperref}
\usepackage{geometry}
\usepackage{xcolor}

\graphicspath{{./images/}}
\setlength{\topmargin}{0pt}
\setlength{\headsep}{0pt}
\textheight = 600pt

\title{Probability Theory \\ Daily Task}
\author{Ben Kallus}
\date{November 9, 2020}

\begin{document}
\color{white}
\pagecolor{black}
\maketitle

\noindent{\bf 1.} N/A

\medskip
\noindent{\bf 2.} N/A

\medskip
\noindent{\bf 3.} In see a rough triangle. This makes sense, since in order for the average of the two sample points to be 1, we'd have to select 1 twice, which is a 1\% chance. This implies that we'd expect to see more values that are close to the mean of the distribution than values that are far from it.

\medskip
\noindent{\bf 4.} At $n=3$ it looks pretty normal, and at $n=4$ it's unmistakably normal. This is a little lower than I expected.

\medskip
\noindent{\bf 5.} At $n=9$ it looks normal, but you have to make it a lot bigger before the tails become symmetric. 

\medskip
\noindent{\bf 6.} The sample mean looks normal at $n=1$.

\medskip
\noindent{\bf 7.} The CLT definitely takes the longest to work on the exponential distribution, then uniform, then normal. My guess is that it takes the longest to work on asymmetric distributions.

\end{document}