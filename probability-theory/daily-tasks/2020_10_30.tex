\documentclass[12pt]{article}
\usepackage[english]{babel}
\usepackage[utf8]{inputenc}
\usepackage{amsmath, amssymb,amsthm}
\usepackage{graphicx}
\usepackage{hyperref}
\usepackage{geometry}
\usepackage{xcolor}

\graphicspath{{./images/}}
\setlength{\topmargin}{0pt}
\setlength{\headsep}{0pt}
\textheight = 600pt

\title{Probability Theory \\ Daily Task}
\author{Ben Kallus}
\date{October 30, 2020}

\begin{document}
\color{white}
\pagecolor{black}
\maketitle

\noindent{\bf 1.} This seems like a good guess to me. It fits the formula.

\medskip
\noindent{\bf 2.}
\begin{align*}
    \mathbb E\hat\lambda &= \frac17 \cdot \mathbb EX \\
                         &= \frac17 \cdot \lambda \cdot 7 \\
                         &= \lambda.
\end{align*} Thus, this guess is correct on average.

\medskip
\noindent{\bf 3.} $$\hat\lambda = \frac{182}7 = 26$$ $$\mathbb P(\text{You receive no emails in the next hour}) = \frac{e^{-26\cdot\frac1{24}}(26\cdot \frac1{24})^0}{0!} = e^{-\frac{13}{12}}$$

\medskip
\noindent{\bf 4.} I think your guess would have been more accurate because of the weka law of large numbers.

\end{document}